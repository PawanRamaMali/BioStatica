% Options for packages loaded elsewhere
\PassOptionsToPackage{unicode}{hyperref}
\PassOptionsToPackage{hyphens}{url}
\PassOptionsToPackage{dvipsnames,svgnames,x11names}{xcolor}
%
\documentclass[
  11pt,
  letterpaper,
  oneside]{book}

\usepackage{amsmath,amssymb}
\usepackage{setspace}
\usepackage{iftex}
\ifPDFTeX
  \usepackage[T1]{fontenc}
  \usepackage[utf8]{inputenc}
  \usepackage{textcomp} % provide euro and other symbols
\else % if luatex or xetex
  \usepackage{unicode-math}
  \defaultfontfeatures{Scale=MatchLowercase}
  \defaultfontfeatures[\rmfamily]{Ligatures=TeX,Scale=1}
\fi
\usepackage[]{crimson}
\ifPDFTeX\else  
    % xetex/luatex font selection
\fi
% Use upquote if available, for straight quotes in verbatim environments
\IfFileExists{upquote.sty}{\usepackage{upquote}}{}
\IfFileExists{microtype.sty}{% use microtype if available
  \usepackage[]{microtype}
  \UseMicrotypeSet[protrusion]{basicmath} % disable protrusion for tt fonts
}{}
\makeatletter
\@ifundefined{KOMAClassName}{% if non-KOMA class
  \IfFileExists{parskip.sty}{%
    \usepackage{parskip}
  }{% else
    \setlength{\parindent}{0pt}
    \setlength{\parskip}{6pt plus 2pt minus 1pt}}
}{% if KOMA class
  \KOMAoptions{parskip=half}}
\makeatother
\usepackage{xcolor}
\usepackage[top=1in,bottom=1in,left=1.2in,right=1in]{geometry}
\setlength{\emergencystretch}{3em} % prevent overfull lines
\setcounter{secnumdepth}{5}
% Make \paragraph and \subparagraph free-standing
\makeatletter
\ifx\paragraph\undefined\else
  \let\oldparagraph\paragraph
  \renewcommand{\paragraph}{
    \@ifstar
      \xxxParagraphStar
      \xxxParagraphNoStar
  }
  \newcommand{\xxxParagraphStar}[1]{\oldparagraph*{#1}\mbox{}}
  \newcommand{\xxxParagraphNoStar}[1]{\oldparagraph{#1}\mbox{}}
\fi
\ifx\subparagraph\undefined\else
  \let\oldsubparagraph\subparagraph
  \renewcommand{\subparagraph}{
    \@ifstar
      \xxxSubParagraphStar
      \xxxSubParagraphNoStar
  }
  \newcommand{\xxxSubParagraphStar}[1]{\oldsubparagraph*{#1}\mbox{}}
  \newcommand{\xxxSubParagraphNoStar}[1]{\oldsubparagraph{#1}\mbox{}}
\fi
\makeatother

\usepackage{color}
\usepackage{fancyvrb}
\newcommand{\VerbBar}{|}
\newcommand{\VERB}{\Verb[commandchars=\\\{\}]}
\DefineVerbatimEnvironment{Highlighting}{Verbatim}{commandchars=\\\{\}}
% Add ',fontsize=\small' for more characters per line
\usepackage{framed}
\definecolor{shadecolor}{RGB}{241,243,245}
\newenvironment{Shaded}{\begin{snugshade}}{\end{snugshade}}
\newcommand{\AlertTok}[1]{\textcolor[rgb]{0.68,0.00,0.00}{#1}}
\newcommand{\AnnotationTok}[1]{\textcolor[rgb]{0.37,0.37,0.37}{#1}}
\newcommand{\AttributeTok}[1]{\textcolor[rgb]{0.40,0.45,0.13}{#1}}
\newcommand{\BaseNTok}[1]{\textcolor[rgb]{0.68,0.00,0.00}{#1}}
\newcommand{\BuiltInTok}[1]{\textcolor[rgb]{0.00,0.23,0.31}{#1}}
\newcommand{\CharTok}[1]{\textcolor[rgb]{0.13,0.47,0.30}{#1}}
\newcommand{\CommentTok}[1]{\textcolor[rgb]{0.37,0.37,0.37}{#1}}
\newcommand{\CommentVarTok}[1]{\textcolor[rgb]{0.37,0.37,0.37}{\textit{#1}}}
\newcommand{\ConstantTok}[1]{\textcolor[rgb]{0.56,0.35,0.01}{#1}}
\newcommand{\ControlFlowTok}[1]{\textcolor[rgb]{0.00,0.23,0.31}{\textbf{#1}}}
\newcommand{\DataTypeTok}[1]{\textcolor[rgb]{0.68,0.00,0.00}{#1}}
\newcommand{\DecValTok}[1]{\textcolor[rgb]{0.68,0.00,0.00}{#1}}
\newcommand{\DocumentationTok}[1]{\textcolor[rgb]{0.37,0.37,0.37}{\textit{#1}}}
\newcommand{\ErrorTok}[1]{\textcolor[rgb]{0.68,0.00,0.00}{#1}}
\newcommand{\ExtensionTok}[1]{\textcolor[rgb]{0.00,0.23,0.31}{#1}}
\newcommand{\FloatTok}[1]{\textcolor[rgb]{0.68,0.00,0.00}{#1}}
\newcommand{\FunctionTok}[1]{\textcolor[rgb]{0.28,0.35,0.67}{#1}}
\newcommand{\ImportTok}[1]{\textcolor[rgb]{0.00,0.46,0.62}{#1}}
\newcommand{\InformationTok}[1]{\textcolor[rgb]{0.37,0.37,0.37}{#1}}
\newcommand{\KeywordTok}[1]{\textcolor[rgb]{0.00,0.23,0.31}{\textbf{#1}}}
\newcommand{\NormalTok}[1]{\textcolor[rgb]{0.00,0.23,0.31}{#1}}
\newcommand{\OperatorTok}[1]{\textcolor[rgb]{0.37,0.37,0.37}{#1}}
\newcommand{\OtherTok}[1]{\textcolor[rgb]{0.00,0.23,0.31}{#1}}
\newcommand{\PreprocessorTok}[1]{\textcolor[rgb]{0.68,0.00,0.00}{#1}}
\newcommand{\RegionMarkerTok}[1]{\textcolor[rgb]{0.00,0.23,0.31}{#1}}
\newcommand{\SpecialCharTok}[1]{\textcolor[rgb]{0.37,0.37,0.37}{#1}}
\newcommand{\SpecialStringTok}[1]{\textcolor[rgb]{0.13,0.47,0.30}{#1}}
\newcommand{\StringTok}[1]{\textcolor[rgb]{0.13,0.47,0.30}{#1}}
\newcommand{\VariableTok}[1]{\textcolor[rgb]{0.07,0.07,0.07}{#1}}
\newcommand{\VerbatimStringTok}[1]{\textcolor[rgb]{0.13,0.47,0.30}{#1}}
\newcommand{\WarningTok}[1]{\textcolor[rgb]{0.37,0.37,0.37}{\textit{#1}}}

\providecommand{\tightlist}{%
  \setlength{\itemsep}{0pt}\setlength{\parskip}{0pt}}\usepackage{longtable,booktabs,array}
\usepackage{calc} % for calculating minipage widths
% Correct order of tables after \paragraph or \subparagraph
\usepackage{etoolbox}
\makeatletter
\patchcmd\longtable{\par}{\if@noskipsec\mbox{}\fi\par}{}{}
\makeatother
% Allow footnotes in longtable head/foot
\IfFileExists{footnotehyper.sty}{\usepackage{footnotehyper}}{\usepackage{footnote}}
\makesavenoteenv{longtable}
\usepackage{graphicx}
\makeatletter
\def\maxwidth{\ifdim\Gin@nat@width>\linewidth\linewidth\else\Gin@nat@width\fi}
\def\maxheight{\ifdim\Gin@nat@height>\textheight\textheight\else\Gin@nat@height\fi}
\makeatother
% Scale images if necessary, so that they will not overflow the page
% margins by default, and it is still possible to overwrite the defaults
% using explicit options in \includegraphics[width, height, ...]{}
\setkeys{Gin}{width=\maxwidth,height=\maxheight,keepaspectratio}
% Set default figure placement to htbp
\makeatletter
\def\fps@figure{htbp}
\makeatother
% definitions for citeproc citations
\NewDocumentCommand\citeproctext{}{}
\NewDocumentCommand\citeproc{mm}{%
  \begingroup\def\citeproctext{#2}\cite{#1}\endgroup}
\makeatletter
 % allow citations to break across lines
 \let\@cite@ofmt\@firstofone
 % avoid brackets around text for \cite:
 \def\@biblabel#1{}
 \def\@cite#1#2{{#1\if@tempswa , #2\fi}}
\makeatother
\newlength{\cslhangindent}
\setlength{\cslhangindent}{1.5em}
\newlength{\csllabelwidth}
\setlength{\csllabelwidth}{3em}
\newenvironment{CSLReferences}[2] % #1 hanging-indent, #2 entry-spacing
 {\begin{list}{}{%
  \setlength{\itemindent}{0pt}
  \setlength{\leftmargin}{0pt}
  \setlength{\parsep}{0pt}
  % turn on hanging indent if param 1 is 1
  \ifodd #1
   \setlength{\leftmargin}{\cslhangindent}
   \setlength{\itemindent}{-1\cslhangindent}
  \fi
  % set entry spacing
  \setlength{\itemsep}{#2\baselineskip}}}
 {\end{list}}
\usepackage{calc}
\newcommand{\CSLBlock}[1]{\hfill\break\parbox[t]{\linewidth}{\strut\ignorespaces#1\strut}}
\newcommand{\CSLLeftMargin}[1]{\parbox[t]{\csllabelwidth}{\strut#1\strut}}
\newcommand{\CSLRightInline}[1]{\parbox[t]{\linewidth - \csllabelwidth}{\strut#1\strut}}
\newcommand{\CSLIndent}[1]{\hspace{\cslhangindent}#1}

\makeatletter
\@ifpackageloaded{tcolorbox}{}{\usepackage[skins,breakable]{tcolorbox}}
\@ifpackageloaded{fontawesome5}{}{\usepackage{fontawesome5}}
\definecolor{quarto-callout-color}{HTML}{909090}
\definecolor{quarto-callout-note-color}{HTML}{0758E5}
\definecolor{quarto-callout-important-color}{HTML}{CC1914}
\definecolor{quarto-callout-warning-color}{HTML}{EB9113}
\definecolor{quarto-callout-tip-color}{HTML}{00A047}
\definecolor{quarto-callout-caution-color}{HTML}{FC5300}
\definecolor{quarto-callout-color-frame}{HTML}{acacac}
\definecolor{quarto-callout-note-color-frame}{HTML}{4582ec}
\definecolor{quarto-callout-important-color-frame}{HTML}{d9534f}
\definecolor{quarto-callout-warning-color-frame}{HTML}{f0ad4e}
\definecolor{quarto-callout-tip-color-frame}{HTML}{02b875}
\definecolor{quarto-callout-caution-color-frame}{HTML}{fd7e14}
\makeatother
\makeatletter
\@ifpackageloaded{bookmark}{}{\usepackage{bookmark}}
\makeatother
\makeatletter
\@ifpackageloaded{caption}{}{\usepackage{caption}}
\AtBeginDocument{%
\ifdefined\contentsname
  \renewcommand*\contentsname{Table of contents}
\else
  \newcommand\contentsname{Table of contents}
\fi
\ifdefined\listfigurename
  \renewcommand*\listfigurename{List of Figures}
\else
  \newcommand\listfigurename{List of Figures}
\fi
\ifdefined\listtablename
  \renewcommand*\listtablename{List of Tables}
\else
  \newcommand\listtablename{List of Tables}
\fi
\ifdefined\figurename
  \renewcommand*\figurename{Figure}
\else
  \newcommand\figurename{Figure}
\fi
\ifdefined\tablename
  \renewcommand*\tablename{Table}
\else
  \newcommand\tablename{Table}
\fi
}
\@ifpackageloaded{float}{}{\usepackage{float}}
\floatstyle{ruled}
\@ifundefined{c@chapter}{\newfloat{codelisting}{h}{lop}}{\newfloat{codelisting}{h}{lop}[chapter]}
\floatname{codelisting}{Listing}
\newcommand*\listoflistings{\listof{codelisting}{List of Listings}}
\makeatother
\makeatletter
\makeatother
\makeatletter
\@ifpackageloaded{caption}{}{\usepackage{caption}}
\@ifpackageloaded{subcaption}{}{\usepackage{subcaption}}
\makeatother

\ifLuaTeX
\usepackage[bidi=basic]{babel}
\else
\usepackage[bidi=default]{babel}
\fi
\babelprovide[main,import]{english}
% get rid of language-specific shorthands (see #6817):
\let\LanguageShortHands\languageshorthands
\def\languageshorthands#1{}
\ifLuaTeX
  \usepackage{selnolig}  % disable illegal ligatures
\fi
\usepackage{bookmark}

\IfFileExists{xurl.sty}{\usepackage{xurl}}{} % add URL line breaks if available
\urlstyle{same} % disable monospaced font for URLs
\hypersetup{
  pdftitle={BioStatica - Complete Guide to Biostatistics},
  pdfauthor={Pawan Rama Mali},
  pdflang={en},
  colorlinks=true,
  linkcolor={Maroon},
  filecolor={Maroon},
  citecolor={Blue},
  urlcolor={Blue},
  pdfcreator={LaTeX via pandoc}}


\title{BioStatica - Complete Guide to Biostatistics}
\usepackage{etoolbox}
\makeatletter
\providecommand{\subtitle}[1]{% add subtitle to \maketitle
  \apptocmd{\@title}{\par {\large #1 \par}}{}{}
}
\makeatother
\subtitle{From foundational principles to modern methodologies used in
cutting-edge research}
\author{Pawan Rama Mali}
\date{2025-09-19}

\begin{document}
\frontmatter
\maketitle

\renewcommand*\contentsname{Table of contents}
{
\hypersetup{linkcolor=}
\setcounter{tocdepth}{2}
\tableofcontents
}

\setstretch{1.2}
\mainmatter
\bookmarksetup{startatroot}

\chapter*{Preface}\label{preface}
\addcontentsline{toc}{chapter}{Preface}

\markboth{Preface}{Preface}

BioStatica is a comprehensive, beginner-friendly platform offering
tutorials, interactive modules, and real-world examples to master
biostatistics---from foundational principles to modern methodologies
used in cutting-edge research.

\section*{About This Book}\label{about-this-book}
\addcontentsline{toc}{section}{About This Book}

\markright{About This Book}

This book delivers a structured learning pathway---from statistical
fundamentals to advanced methods in biostatistics---blending clarity
with rigor for learners at all levels. Whether you're a student
beginning your journey in biostatistics or a researcher looking to
deepen your understanding, this guide provides practical insights and
real-world applications.

\section*{Prerequisites}\label{prerequisites}
\addcontentsline{toc}{section}{Prerequisites}

\markright{Prerequisites}

Before diving into this book, you should have:

\begin{itemize}
\tightlist
\item
  Basic understanding of mathematics and algebra
\item
  Familiarity with statistical software (R, Python, or similar)
\item
  Access to a computer with statistical software installed
\end{itemize}

\section*{How to Use This Book}\label{how-to-use-this-book}
\addcontentsline{toc}{section}{How to Use This Book}

\markright{How to Use This Book}

Each chapter builds upon previous concepts, so we recommend reading them
in order, especially if you're new to biostatistics. However,
experienced readers may jump to specific topics of interest.

\section*{Interactive Learning}\label{interactive-learning}
\addcontentsline{toc}{section}{Interactive Learning}

\markright{Interactive Learning}

This book includes:

\begin{itemize}
\tightlist
\item
  Interactive examples and exercises
\item
  Real-world datasets for practice
\item
  Code snippets in R and Python
\item
  Visual demonstrations of statistical concepts
\item
  Practical applications in biological and medical research
\end{itemize}

\section*{Contributing}\label{contributing}
\addcontentsline{toc}{section}{Contributing}

\markright{Contributing}

This is an open-source project. Contributions, suggestions, and feedback
are welcome. Please visit our
\href{https://github.com/PawanRamaMali/BioStatica}{GitHub repository} to
contribute.

\section*{Acknowledgments}\label{acknowledgments}
\addcontentsline{toc}{section}{Acknowledgments}

\markright{Acknowledgments}

We thank the biostatistics and open science communities for their
continued support and contributions to making statistical education
accessible to all.

\begin{center}\rule{0.5\linewidth}{0.5pt}\end{center}

\emph{Ready to begin your journey into biostatistics? Let's start with
the introduction.}

\bookmarksetup{startatroot}

\chapter{Introduction: Welcome to
Biostatistics}\label{introduction-welcome-to-biostatistics}

Biostatistics is the application of statistics to biological and
health-related data. This field bridges the gap between mathematical
theory and real-world medical research, helping us make sense of complex
biological phenomena and draw meaningful conclusions from data.

Whether you're a medical student beginning your research journey, a
healthcare professional seeking to understand clinical studies, or a
researcher diving into data analysis, this comprehensive guide will take
you from basic concepts to advanced methodologies used in cutting-edge
research.

\begin{tcolorbox}[enhanced jigsaw, toprule=.15mm, left=2mm, opacitybacktitle=0.6, colframe=quarto-callout-note-color-frame, leftrule=.75mm, titlerule=0mm, coltitle=black, colbacktitle=quarto-callout-note-color!10!white, toptitle=1mm, title=\textcolor{quarto-callout-note-color}{\faInfo}\hspace{0.5em}{Why Biostatistics Matters}, bottomtitle=1mm, arc=.35mm, rightrule=.15mm, bottomrule=.15mm, breakable, opacityback=0, colback=white]

Every major medical breakthrough---from vaccine development to treatment
protocols---relies on biostatistical methods. Understanding these
concepts enables you to:

\begin{itemize}
\tightlist
\item
  Critically evaluate medical literature
\item
  Design robust research studies
\item
  Analyze complex biological data
\item
  Make evidence-based decisions in healthcare
\end{itemize}

\end{tcolorbox}

This textbook is structured to provide a logical progression from
fundamental concepts to specialized applications. Each chapter builds
upon previous knowledge while remaining accessible to readers at
different levels of statistical background.

\bookmarksetup{startatroot}

\chapter{Foundational Concepts}\label{foundational-concepts}

Master the essential building blocks of statistical thinking that form
the foundation for all biostatistical analysis.

\section{Descriptive Statistics \&
Visualization}\label{descriptive-statistics-visualization}

Before diving into complex analyses, we must first understand how to
describe and visualize our data. Descriptive statistics provide the
first glimpse into what our data tells us.

\begin{tcolorbox}[enhanced jigsaw, toprule=.15mm, left=2mm, opacitybacktitle=0.6, colframe=quarto-callout-note-color-frame, leftrule=.75mm, titlerule=0mm, coltitle=black, colbacktitle=quarto-callout-note-color!10!white, toptitle=1mm, title=\textcolor{quarto-callout-note-color}{\faInfo}\hspace{0.5em}{Understanding Your Data}, bottomtitle=1mm, arc=.35mm, rightrule=.15mm, bottomrule=.15mm, breakable, opacityback=0, colback=white]

Every dataset tells a story. Descriptive statistics help us understand:

\begin{itemize}
\tightlist
\item
  \textbf{Central Tendency:} Where does the typical value lie?
\item
  \textbf{Variability:} How spread out are our observations?
\item
  \textbf{Shape:} Is the data symmetric or skewed?
\item
  \textbf{Outliers:} Are there unusual observations?
\end{itemize}

\end{tcolorbox}

\subsection{Measures of Central
Tendency}\label{measures-of-central-tendency}

\textbf{Mean (Average):} The arithmetic mean is the sum of all values
divided by the number of observations. While intuitive, it's sensitive
to extreme values.

\[\text{Mean} = \frac{\sum x}{n}\]

\begin{tcolorbox}[enhanced jigsaw, toprule=.15mm, left=2mm, opacitybacktitle=0.6, colframe=quarto-callout-tip-color-frame, leftrule=.75mm, titlerule=0mm, coltitle=black, colbacktitle=quarto-callout-tip-color!10!white, toptitle=1mm, title=\textcolor{quarto-callout-tip-color}{\faLightbulb}\hspace{0.5em}{Example: Blood Pressure Study}, bottomtitle=1mm, arc=.35mm, rightrule=.15mm, bottomrule=.15mm, breakable, opacityback=0, colback=white]

Consider systolic blood pressure measurements from 7 patients: 120, 125,
118, 140, 122, 135, 180

Mean = (120 + 125 + 118 + 140 + 122 + 135 + 180) / 7 = 134.3 mmHg

Notice how the outlier (180) pulls the mean upward.

\end{tcolorbox}

\textbf{Median:} The middle value when data is arranged in order. More
robust to outliers than the mean.

\textbf{Mode:} The most frequently occurring value. Useful for
categorical data or to identify peaks in distributions.

\subsection{Measures of Variability}\label{measures-of-variability}

\textbf{Range:} Simply the difference between maximum and minimum
values. Easy to calculate but affected by outliers.

\textbf{Standard Deviation:} Measures how much individual observations
deviate from the mean. This is perhaps the most important measure of
variability in statistics.

\[\text{Standard Deviation} = \sqrt{\frac{\sum(x - \text{mean})^2}{n-1}}\]

\begin{tcolorbox}[enhanced jigsaw, toprule=.15mm, left=2mm, opacitybacktitle=0.6, colframe=quarto-callout-warning-color-frame, leftrule=.75mm, titlerule=0mm, coltitle=black, colbacktitle=quarto-callout-warning-color!10!white, toptitle=1mm, title=\textcolor{quarto-callout-warning-color}{\faExclamationTriangle}\hspace{0.5em}{Common Mistake}, bottomtitle=1mm, arc=.35mm, rightrule=.15mm, bottomrule=.15mm, breakable, opacityback=0, colback=white]

Always use n-1 (not n) in the denominator when calculating sample
standard deviation. This provides an unbiased estimate of population
variability.

\end{tcolorbox}

\textbf{Variance:} The square of standard deviation. While less
intuitive (different units), it's mathematically convenient for many
statistical procedures.

\subsection{Data Visualization}\label{data-visualization}

Visual representation of data often reveals patterns invisible in raw
numbers:

\textbf{Histograms:} Show the distribution shape and identify skewness,
multiple peaks, or outliers.

\textbf{Box Plots:} Provide a five-number summary (minimum, Q1, median,
Q3, maximum) and clearly highlight outliers.

\begin{tcolorbox}[enhanced jigsaw, toprule=.15mm, left=2mm, opacitybacktitle=0.6, colframe=quarto-callout-tip-color-frame, leftrule=.75mm, titlerule=0mm, coltitle=black, colbacktitle=quarto-callout-tip-color!10!white, toptitle=1mm, title=\textcolor{quarto-callout-tip-color}{\faLightbulb}\hspace{0.5em}{Interpreting Box Plots in Medical Research}, bottomtitle=1mm, arc=.35mm, rightrule=.15mm, bottomrule=.15mm, breakable, opacityback=0, colback=white]

A box plot of recovery times after surgery might show:

\begin{itemize}
\tightlist
\item
  Median recovery: 7 days
\item
  Most patients recover between 5-10 days (IQR)
\item
  Some outliers taking 20+ days
\item
  Distribution skewed toward longer recovery times
\end{itemize}

\end{tcolorbox}

\textbf{Scatter Plots:} Essential for examining relationships between
two continuous variables.

\section{Probability \& Distributions}\label{probability-distributions}

Probability theory provides the mathematical foundation for statistical
inference. Understanding distributions helps us model real-world
phenomena and calculate the likelihood of different outcomes.

\begin{tcolorbox}[enhanced jigsaw, toprule=.15mm, left=2mm, opacitybacktitle=0.6, colframe=quarto-callout-note-color-frame, leftrule=.75mm, titlerule=0mm, coltitle=black, colbacktitle=quarto-callout-note-color!10!white, toptitle=1mm, title=\textcolor{quarto-callout-note-color}{\faInfo}\hspace{0.5em}{What is Probability?}, bottomtitle=1mm, arc=.35mm, rightrule=.15mm, bottomrule=.15mm, breakable, opacityback=0, colback=white]

Probability quantifies uncertainty. In medical research, we use
probability to:

\begin{itemize}
\tightlist
\item
  Assess the likelihood of treatment success
\item
  Calculate confidence in our estimates
\item
  Determine sample sizes needed for studies
\item
  Evaluate the strength of evidence against hypotheses
\end{itemize}

\end{tcolorbox}

\subsection{Basic Probability Rules}\label{basic-probability-rules}

\begin{enumerate}
\def\labelenumi{\arabic{enumi}.}
\tightlist
\item
  \textbf{Addition Rule:} P(A or B) = P(A) + P(B) - P(A and B)
\item
  \textbf{Multiplication Rule:} P(A and B) = P(A) × P(B\textbar A)
\item
  \textbf{Complement Rule:} P(not A) = 1 - P(A)
\end{enumerate}

\begin{tcolorbox}[enhanced jigsaw, toprule=.15mm, left=2mm, opacitybacktitle=0.6, colframe=quarto-callout-tip-color-frame, leftrule=.75mm, titlerule=0mm, coltitle=black, colbacktitle=quarto-callout-tip-color!10!white, toptitle=1mm, title=\textcolor{quarto-callout-tip-color}{\faLightbulb}\hspace{0.5em}{Medical Example: Disease Testing}, bottomtitle=1mm, arc=.35mm, rightrule=.15mm, bottomrule=.15mm, breakable, opacityback=0, colback=white]

Consider a diagnostic test for a rare disease:

\begin{itemize}
\tightlist
\item
  Disease prevalence: 1\% (P(Disease) = 0.01)
\item
  Test sensitivity: 95\% (P(Positive\textbar Disease) = 0.95)
\item
  Test specificity: 90\% (P(Negative\textbar No Disease) = 0.90)
\end{itemize}

What's the probability someone actually has the disease if they test
positive?

This requires Bayes' theorem and often surprises medical professionals!

\end{tcolorbox}

\subsection{The Normal Distribution}\label{the-normal-distribution}

The normal (Gaussian) distribution is the cornerstone of statistical
analysis. Many biological measurements naturally follow this bell-shaped
curve.

\begin{tcolorbox}[enhanced jigsaw, toprule=.15mm, left=2mm, opacitybacktitle=0.6, colframe=quarto-callout-note-color-frame, leftrule=.75mm, titlerule=0mm, coltitle=black, colbacktitle=quarto-callout-note-color!10!white, toptitle=1mm, title=\textcolor{quarto-callout-note-color}{\faInfo}\hspace{0.5em}{Properties of Normal Distribution}, bottomtitle=1mm, arc=.35mm, rightrule=.15mm, bottomrule=.15mm, breakable, opacityback=0, colback=white]

\begin{itemize}
\tightlist
\item
  Symmetric and bell-shaped
\item
  Mean = Median = Mode
\item
  68\% of data within 1 standard deviation
\item
  95\% of data within 2 standard deviations
\item
  99.7\% of data within 3 standard deviations
\end{itemize}

\end{tcolorbox}

Examples of normally distributed biological variables:

\begin{itemize}
\tightlist
\item
  Height and weight in populations
\item
  Blood pressure readings
\item
  IQ scores
\item
  Many laboratory test results
\end{itemize}

\subsection{The t-Distribution}\label{the-t-distribution}

When working with small samples (n \textless{} 30), the t-distribution
becomes crucial. It's similar to the normal distribution but with
heavier tails, accounting for additional uncertainty from small sample
sizes.

\begin{tcolorbox}[enhanced jigsaw, toprule=.15mm, left=2mm, opacitybacktitle=0.6, colframe=quarto-callout-warning-color-frame, leftrule=.75mm, titlerule=0mm, coltitle=black, colbacktitle=quarto-callout-warning-color!10!white, toptitle=1mm, title=\textcolor{quarto-callout-warning-color}{\faExclamationTriangle}\hspace{0.5em}{When to Use t vs Normal}, bottomtitle=1mm, arc=.35mm, rightrule=.15mm, bottomrule=.15mm, breakable, opacityback=0, colback=white]

Use t-distribution when:

\begin{itemize}
\tightlist
\item
  Sample size is small (n \textless{} 30)
\item
  Population standard deviation is unknown
\item
  Data is approximately normally distributed
\end{itemize}

\end{tcolorbox}

\section{Sampling \& Sampling
Distributions}\label{sampling-sampling-distributions}

Understanding how samples relate to populations is fundamental to
statistical inference. The Central Limit Theorem bridges individual
observations to population-level conclusions.

\begin{tcolorbox}[enhanced jigsaw, toprule=.15mm, left=2mm, opacitybacktitle=0.6, colframe=quarto-callout-note-color-frame, leftrule=.75mm, titlerule=0mm, coltitle=black, colbacktitle=quarto-callout-note-color!10!white, toptitle=1mm, title=\textcolor{quarto-callout-note-color}{\faInfo}\hspace{0.5em}{Population vs Sample}, bottomtitle=1mm, arc=.35mm, rightrule=.15mm, bottomrule=.15mm, breakable, opacityback=0, colback=white]

\textbf{Population:} All possible subjects of interest (e.g., all
patients with diabetes)

\textbf{Sample:} A subset of the population we actually observe (e.g.,
100 diabetic patients in our study)

We use sample statistics to estimate population parameters.

\end{tcolorbox}

\subsection{Sampling Methods}\label{sampling-methods}

\textbf{Simple Random Sampling:} Every member has equal chance of
selection.

\textbf{Stratified Sampling:} Population divided into groups (strata),
then random sampling within each group.

\begin{tcolorbox}[enhanced jigsaw, toprule=.15mm, left=2mm, opacitybacktitle=0.6, colframe=quarto-callout-tip-color-frame, leftrule=.75mm, titlerule=0mm, coltitle=black, colbacktitle=quarto-callout-tip-color!10!white, toptitle=1mm, title=\textcolor{quarto-callout-tip-color}{\faLightbulb}\hspace{0.5em}{Stratified Sampling in Clinical Trials}, bottomtitle=1mm, arc=.35mm, rightrule=.15mm, bottomrule=.15mm, breakable, opacityback=0, colback=white]

When studying a new cardiac medication, researchers might stratify by:

\begin{itemize}
\tightlist
\item
  Age groups (18-40, 41-65, 65+)
\item
  Gender
\item
  Severity of condition
\end{itemize}

This ensures adequate representation across important subgroups.

\end{tcolorbox}

\subsection{The Central Limit Theorem}\label{the-central-limit-theorem}

This fundamental theorem states that sample means will be approximately
normally distributed, regardless of the population distribution, as
sample size increases (typically n ≥ 30).

\begin{tcolorbox}[enhanced jigsaw, toprule=.15mm, left=2mm, opacitybacktitle=0.6, colframe=quarto-callout-note-color-frame, leftrule=.75mm, titlerule=0mm, coltitle=black, colbacktitle=quarto-callout-note-color!10!white, toptitle=1mm, title=\textcolor{quarto-callout-note-color}{\faInfo}\hspace{0.5em}{Implications of Central Limit Theorem}, bottomtitle=1mm, arc=.35mm, rightrule=.15mm, bottomrule=.15mm, breakable, opacityback=0, colback=white]

\begin{itemize}
\tightlist
\item
  We can use normal distribution for inference about means
\item
  Larger samples give more precise estimates
\item
  We can calculate confidence intervals and p-values
\item
  Many statistical tests are based on this principle
\end{itemize}

\end{tcolorbox}

\[\text{Standard Error of Mean} = \frac{\sigma}{\sqrt{n}}\]

The standard error quantifies how much sample means vary from the true
population mean.

\section{Point \& Interval Estimation}\label{point-interval-estimation}

Estimation allows us to use sample data to make educated guesses about
population parameters, with quantified uncertainty.

\subsection{Point Estimation}\label{point-estimation}

A point estimate is a single value that serves as our ``best guess'' for
a population parameter.

\begin{tcolorbox}[enhanced jigsaw, toprule=.15mm, left=2mm, opacitybacktitle=0.6, colframe=quarto-callout-tip-color-frame, leftrule=.75mm, titlerule=0mm, coltitle=black, colbacktitle=quarto-callout-tip-color!10!white, toptitle=1mm, title=\textcolor{quarto-callout-tip-color}{\faLightbulb}\hspace{0.5em}{Point Estimates in Medicine}, bottomtitle=1mm, arc=.35mm, rightrule=.15mm, bottomrule=.15mm, breakable, opacityback=0, colback=white]

\begin{itemize}
\tightlist
\item
  Sample mean blood pressure → Population mean blood pressure
\item
  Sample proportion cured → Population cure rate
\item
  Sample correlation → Population correlation
\end{itemize}

\end{tcolorbox}

\subsection{Interval Estimation (Confidence
Intervals)}\label{interval-estimation-confidence-intervals}

While point estimates provide a single value, confidence intervals
provide a range of plausible values for the population parameter.

\begin{tcolorbox}[enhanced jigsaw, toprule=.15mm, left=2mm, opacitybacktitle=0.6, colframe=quarto-callout-note-color-frame, leftrule=.75mm, titlerule=0mm, coltitle=black, colbacktitle=quarto-callout-note-color!10!white, toptitle=1mm, title=\textcolor{quarto-callout-note-color}{\faInfo}\hspace{0.5em}{Interpreting Confidence Intervals}, bottomtitle=1mm, arc=.35mm, rightrule=.15mm, bottomrule=.15mm, breakable, opacityback=0, colback=white]

A 95\% confidence interval means:

``If we repeated this study 100 times, approximately 95 of the resulting
confidence intervals would contain the true population parameter.''

\end{tcolorbox}

\[95\% \text{ CI for mean} = \bar{x} \pm t_{(0.025, df)} \times \frac{s}{\sqrt{n}}\]

\begin{tcolorbox}[enhanced jigsaw, toprule=.15mm, left=2mm, opacitybacktitle=0.6, colframe=quarto-callout-tip-color-frame, leftrule=.75mm, titlerule=0mm, coltitle=black, colbacktitle=quarto-callout-tip-color!10!white, toptitle=1mm, title=\textcolor{quarto-callout-tip-color}{\faLightbulb}\hspace{0.5em}{Clinical Example}, bottomtitle=1mm, arc=.35mm, rightrule=.15mm, bottomrule=.15mm, breakable, opacityback=0, colback=white]

A new pain medication reduces pain scores by an average of 3.2 points
(95\% CI: 2.1 to 4.3).

\textbf{Interpretation:} We're 95\% confident the true average reduction
is between 2.1 and 4.3 points.

\end{tcolorbox}

\begin{tcolorbox}[enhanced jigsaw, toprule=.15mm, left=2mm, opacitybacktitle=0.6, colframe=quarto-callout-warning-color-frame, leftrule=.75mm, titlerule=0mm, coltitle=black, colbacktitle=quarto-callout-warning-color!10!white, toptitle=1mm, title=\textcolor{quarto-callout-warning-color}{\faExclamationTriangle}\hspace{0.5em}{Common Misinterpretation}, bottomtitle=1mm, arc=.35mm, rightrule=.15mm, bottomrule=.15mm, breakable, opacityback=0, colback=white]

❌ ``There's a 95\% chance the true mean is in this interval''

✅ ``95\% of such intervals would contain the true mean''

\end{tcolorbox}

\section{Hypothesis Testing
Framework}\label{hypothesis-testing-framework}

Hypothesis testing provides a structured approach to making decisions
under uncertainty, fundamental to scientific research.

\begin{tcolorbox}[enhanced jigsaw, toprule=.15mm, left=2mm, opacitybacktitle=0.6, colframe=quarto-callout-note-color-frame, leftrule=.75mm, titlerule=0mm, coltitle=black, colbacktitle=quarto-callout-note-color!10!white, toptitle=1mm, title=\textcolor{quarto-callout-note-color}{\faInfo}\hspace{0.5em}{The Logic of Hypothesis Testing}, bottomtitle=1mm, arc=.35mm, rightrule=.15mm, bottomrule=.15mm, breakable, opacityback=0, colback=white]

We start with a skeptical position (null hypothesis) and ask: ``Is our
observed data surprising enough to abandon this position?''

\end{tcolorbox}

\subsection{Setting Up Hypotheses}\label{setting-up-hypotheses}

\textbf{Null Hypothesis (H₀):} The ``no effect'' or ``status quo''
hypothesis. What we assume is true until proven otherwise.

\textbf{Alternative Hypothesis (H₁ or Hₐ):} The research hypothesis
we're trying to establish.

\begin{tcolorbox}[enhanced jigsaw, toprule=.15mm, left=2mm, opacitybacktitle=0.6, colframe=quarto-callout-tip-color-frame, leftrule=.75mm, titlerule=0mm, coltitle=black, colbacktitle=quarto-callout-tip-color!10!white, toptitle=1mm, title=\textcolor{quarto-callout-tip-color}{\faLightbulb}\hspace{0.5em}{Hypothesis Example: New Treatment}, bottomtitle=1mm, arc=.35mm, rightrule=.15mm, bottomrule=.15mm, breakable, opacityback=0, colback=white]

Research question: ``Does new drug X reduce blood pressure more than
placebo?''

\begin{itemize}
\tightlist
\item
  \textbf{H₀:} Drug X has no effect (μ\_drug = μ\_placebo)
\item
  \textbf{H₁:} Drug X reduces blood pressure (μ\_drug \textless{}
  μ\_placebo)
\end{itemize}

\end{tcolorbox}

\subsection{Types of Errors in Statistical
Testing}\label{types-of-errors-in-statistical-testing}

Statistical testing is not perfect---we can make two types of errors
when making decisions about hypotheses. Understanding these errors is
crucial for interpreting research results and making informed decisions.

\begin{tcolorbox}[enhanced jigsaw, toprule=.15mm, left=2mm, opacitybacktitle=0.6, colframe=quarto-callout-note-color-frame, leftrule=.75mm, titlerule=0mm, coltitle=black, colbacktitle=quarto-callout-note-color!10!white, toptitle=1mm, title=\textcolor{quarto-callout-note-color}{\faInfo}\hspace{0.5em}{The Truth Table of Statistical Decisions}, bottomtitle=1mm, arc=.35mm, rightrule=.15mm, bottomrule=.15mm, breakable, opacityback=0, colback=white]

When conducting a hypothesis test, there are four possible outcomes:

\begin{longtable}[]{@{}lll@{}}
\toprule\noalign{}
& H₀ is True & H₀ is False \\
\midrule\noalign{}
\endhead
\bottomrule\noalign{}
\endlastfoot
Reject H₀ & Type I Error (α) & Correct Decision \\
Fail to Reject H₀ & Correct Decision & Type II Error (β) \\
\end{longtable}

\end{tcolorbox}

\subsection{Type I Error (α): False
Positive}\label{type-i-error-ux3b1-false-positive}

\textbf{Definition:} Rejecting a true null hypothesis---concluding there
is an effect when there really isn't one.

\textbf{Probability:} The significance level (α) represents the maximum
Type I error rate we're willing to accept, typically set at 0.05 (5\%).

\begin{tcolorbox}[enhanced jigsaw, toprule=.15mm, left=2mm, opacitybacktitle=0.6, colframe=quarto-callout-tip-color-frame, leftrule=.75mm, titlerule=0mm, coltitle=black, colbacktitle=quarto-callout-tip-color!10!white, toptitle=1mm, title=\textcolor{quarto-callout-tip-color}{\faLightbulb}\hspace{0.5em}{Type I Error Examples}, bottomtitle=1mm, arc=.35mm, rightrule=.15mm, bottomrule=.15mm, breakable, opacityback=0, colback=white]

\textbf{Example 1: Drug Testing} - \textbf{H₀:} New drug has no effect
(same as placebo) - \textbf{Truth:} Drug actually has no effect -
\textbf{Type I Error:} Study concludes drug is effective -
\textbf{Consequences:} Ineffective drug approved, wasting resources and
potentially exposing patients to unnecessary side effects

\textbf{Example 2: Diagnostic Testing} - \textbf{H₀:} Patient does not
have disease - \textbf{Truth:} Patient is actually healthy -
\textbf{Type I Error:} Test indicates patient has disease (false
positive) - \textbf{Consequences:} Unnecessary anxiety, additional
testing, and potentially harmful treatments

\textbf{Example 3: Quality Control} - \textbf{H₀:} Manufacturing process
is working correctly - \textbf{Truth:} Process is actually working fine
- \textbf{Type I Error:} Conclude process is faulty -
\textbf{Consequences:} Unnecessary production shutdown, wasted time and
money investigating non-existent problems

\end{tcolorbox}

\begin{tcolorbox}[enhanced jigsaw, toprule=.15mm, left=2mm, opacitybacktitle=0.6, colframe=quarto-callout-warning-color-frame, leftrule=.75mm, titlerule=0mm, coltitle=black, colbacktitle=quarto-callout-warning-color!10!white, toptitle=1mm, title=\textcolor{quarto-callout-warning-color}{\faExclamationTriangle}\hspace{0.5em}{Controlling Type I Error}, bottomtitle=1mm, arc=.35mm, rightrule=.15mm, bottomrule=.15mm, breakable, opacityback=0, colback=white]

We control Type I error by setting the significance level (α) before
conducting our test:

\begin{itemize}
\tightlist
\item
  \textbf{α = 0.05:} 5\% chance of Type I error (most common)
\item
  \textbf{α = 0.01:} 1\% chance of Type I error (more stringent)
\item
  \textbf{α = 0.10:} 10\% chance of Type I error (more lenient)
\end{itemize}

\emph{Lowering α reduces Type I errors but increases Type II errors.}

\end{tcolorbox}

\subsection{Type II Error (β): False
Negative}\label{type-ii-error-ux3b2-false-negative}

\textbf{Definition:} Failing to reject a false null hypothesis---missing
a real effect that actually exists.

\textbf{Statistical Power:} Power = 1 - β, represents the probability of
correctly detecting an effect when it truly exists.

\begin{tcolorbox}[enhanced jigsaw, toprule=.15mm, left=2mm, opacitybacktitle=0.6, colframe=quarto-callout-tip-color-frame, leftrule=.75mm, titlerule=0mm, coltitle=black, colbacktitle=quarto-callout-tip-color!10!white, toptitle=1mm, title=\textcolor{quarto-callout-tip-color}{\faLightbulb}\hspace{0.5em}{Type II Error Examples}, bottomtitle=1mm, arc=.35mm, rightrule=.15mm, bottomrule=.15mm, breakable, opacityback=0, colback=white]

\textbf{Example 1: Drug Testing} - \textbf{H₀:} New drug has no effect -
\textbf{Truth:} Drug is actually effective - \textbf{Type II Error:}
Study fails to detect drug's effectiveness - \textbf{Consequences:}
Effective treatment not approved, patients continue suffering from
treatable condition

\textbf{Example 2: Diagnostic Testing} - \textbf{H₀:} Patient does not
have disease - \textbf{Truth:} Patient actually has the disease -
\textbf{Type II Error:} Test fails to detect disease (false negative) -
\textbf{Consequences:} Delayed treatment, disease progression,
potentially fatal outcomes

\textbf{Example 3: Environmental Monitoring} - \textbf{H₀:} Pollution
levels are safe - \textbf{Truth:} Pollution levels are actually
dangerous - \textbf{Type II Error:} Fail to detect dangerous pollution -
\textbf{Consequences:} Continued exposure to harmful substances, public
health risks

\end{tcolorbox}

\begin{tcolorbox}[enhanced jigsaw, toprule=.15mm, left=2mm, opacitybacktitle=0.6, colframe=quarto-callout-note-color-frame, leftrule=.75mm, titlerule=0mm, coltitle=black, colbacktitle=quarto-callout-note-color!10!white, toptitle=1mm, title=\textcolor{quarto-callout-note-color}{\faInfo}\hspace{0.5em}{Factors Affecting Type II Error (β)}, bottomtitle=1mm, arc=.35mm, rightrule=.15mm, bottomrule=.15mm, breakable, opacityback=0, colback=white]

Several factors influence the probability of Type II error:

\begin{itemize}
\tightlist
\item
  \textbf{Effect Size:} Larger true effects are easier to detect (lower
  β)
\item
  \textbf{Sample Size:} Larger samples reduce β
\item
  \textbf{Significance Level (α):} Higher α reduces β
\item
  \textbf{Variability:} Less noisy data reduces β
\item
  \textbf{Study Design:} Better designs reduce β
\end{itemize}

\end{tcolorbox}

\subsection{Real-World Clinical Example: COVID-19
Testing}\label{real-world-clinical-example-covid-19-testing}

\begin{tcolorbox}[enhanced jigsaw, toprule=.15mm, left=2mm, opacitybacktitle=0.6, colframe=quarto-callout-tip-color-frame, leftrule=.75mm, titlerule=0mm, coltitle=black, colbacktitle=quarto-callout-tip-color!10!white, toptitle=1mm, title=\textcolor{quarto-callout-tip-color}{\faLightbulb}\hspace{0.5em}{Understanding Both Error Types in Practice}, bottomtitle=1mm, arc=.35mm, rightrule=.15mm, bottomrule=.15mm, breakable, opacityback=0, colback=white]

\textbf{Scenario:} Testing for COVID-19 infection \textbf{H₀:} Person is
not infected with COVID-19 \textbf{H₁:} Person is infected with COVID-19

\textbf{Type I Error (False Positive):} - Test says ``infected'' when
person is actually healthy - Rate: Depends on test specificity (1 -
specificity) - Consequences: Unnecessary quarantine, contact tracing,
anxiety - Public health impact: Resource waste, reduced public
confidence

\textbf{Type II Error (False Negative):} - Test says ``not infected''
when person is actually infected - Rate: Depends on test sensitivity (1
- sensitivity) - Consequences: Continued spread, no treatment, false
security - Public health impact: Disease transmission, outbreak
expansion

\textbf{The Trade-off:} During a pandemic, false negatives (Type II
errors) are often considered more dangerous than false positives (Type I
errors) because missing infected individuals leads to continued disease
spread.

\end{tcolorbox}

\subsection{The Relationship Between Error
Types}\label{the-relationship-between-error-types}

\begin{tcolorbox}[enhanced jigsaw, toprule=.15mm, left=2mm, opacitybacktitle=0.6, colframe=quarto-callout-note-color-frame, leftrule=.75mm, titlerule=0mm, coltitle=black, colbacktitle=quarto-callout-note-color!10!white, toptitle=1mm, title=\textcolor{quarto-callout-note-color}{\faInfo}\hspace{0.5em}{Key Insights About Statistical Errors}, bottomtitle=1mm, arc=.35mm, rightrule=.15mm, bottomrule=.15mm, breakable, opacityback=0, colback=white]

\begin{itemize}
\tightlist
\item
  \textbf{Inverse Relationship:} Generally, as Type I error decreases,
  Type II error increases
\item
  \textbf{Sample Size Effect:} Larger samples reduce both types of
  errors
\item
  \textbf{Effect Size Matters:} Larger true effects make Type II errors
  less likely
\item
  \textbf{Context Dependent:} The relative cost of each error type
  varies by situation
\item
  \textbf{Power Analysis:} Helps balance these errors when designing
  studies
\end{itemize}

\end{tcolorbox}

\textbf{Key Relationships:} - Power = 1 - β - P(Type I Error) = α -
P(Type II Error) = β - As n ↑, both α and β ↓ (for fixed effect size)

\subsection{Practical Guidelines for
Researchers}\label{practical-guidelines-for-researchers}

\begin{tcolorbox}[enhanced jigsaw, toprule=.15mm, left=2mm, opacitybacktitle=0.6, colframe=quarto-callout-warning-color-frame, leftrule=.75mm, titlerule=0mm, coltitle=black, colbacktitle=quarto-callout-warning-color!10!white, toptitle=1mm, title=\textcolor{quarto-callout-warning-color}{\faExclamationTriangle}\hspace{0.5em}{Choosing Error Tolerances}, bottomtitle=1mm, arc=.35mm, rightrule=.15mm, bottomrule=.15mm, breakable, opacityback=0, colback=white]

\textbf{When Type I errors are more serious:} - Drug approval studies
(don't approve ineffective drugs) - Diagnostic tests for serious
conditions - Use lower α (0.01 or 0.001)

\textbf{When Type II errors are more serious:} - Screening for treatable
conditions - Safety monitoring studies - Use higher α (0.10) or ensure
high power (0.90+)

\end{tcolorbox}

\begin{tcolorbox}[enhanced jigsaw, toprule=.15mm, left=2mm, opacitybacktitle=0.6, colframe=quarto-callout-note-color-frame, leftrule=.75mm, titlerule=0mm, coltitle=black, colbacktitle=quarto-callout-note-color!10!white, toptitle=1mm, title=\textcolor{quarto-callout-note-color}{\faInfo}\hspace{0.5em}{Medical Context Summary}, bottomtitle=1mm, arc=.35mm, rightrule=.15mm, bottomrule=.15mm, breakable, opacityback=0, colback=white]

\textbf{Type I Error (False Positive):} Concluding a treatment works
when it doesn't, a diagnostic test is positive when disease is absent,
or an intervention has an effect when it doesn't.

\textbf{Type II Error (False Negative):} Failing to detect a treatment
that actually works, missing a disease that is present, or failing to
identify a real intervention effect.

\textbf{Clinical Impact:} Both errors have serious consequences in
healthcare---unnecessary treatments and missed diagnoses can both harm
patients.

\end{tcolorbox}

\subsection{P-values}\label{p-values}

The p-value answers: ``If the null hypothesis were true, what's the
probability of observing data as extreme or more extreme than what we
actually observed?''

\begin{tcolorbox}[enhanced jigsaw, toprule=.15mm, left=2mm, opacitybacktitle=0.6, colframe=quarto-callout-warning-color-frame, leftrule=.75mm, titlerule=0mm, coltitle=black, colbacktitle=quarto-callout-warning-color!10!white, toptitle=1mm, title=\textcolor{quarto-callout-warning-color}{\faExclamationTriangle}\hspace{0.5em}{P-value Misconceptions}, bottomtitle=1mm, arc=.35mm, rightrule=.15mm, bottomrule=.15mm, breakable, opacityback=0, colback=white]

❌ ``P-value is the probability the null hypothesis is true''

❌ ``P-value is the probability of making a mistake''

✅ ``P-value is the probability of observing such data if null
hypothesis is true''

\end{tcolorbox}

\textbf{Statistical Significance:} Typically, p \textless{} 0.05 is
considered statistically significant, meaning we reject the null
hypothesis.

\section{One-Sample \& Two-Sample
t-Tests}\label{one-sample-two-sample-t-tests}

T-tests are among the most commonly used statistical tests in biomedical
research, allowing us to compare means between groups or against known
values.

\subsection{One-Sample t-Test}\label{one-sample-t-test}

Compares a sample mean to a known population value or theoretical
expectation.

\begin{tcolorbox}[enhanced jigsaw, toprule=.15mm, left=2mm, opacitybacktitle=0.6, colframe=quarto-callout-tip-color-frame, leftrule=.75mm, titlerule=0mm, coltitle=black, colbacktitle=quarto-callout-tip-color!10!white, toptitle=1mm, title=\textcolor{quarto-callout-tip-color}{\faLightbulb}\hspace{0.5em}{One-Sample Example}, bottomtitle=1mm, arc=.35mm, rightrule=.15mm, bottomrule=.15mm, breakable, opacityback=0, colback=white]

Normal body temperature is supposedly 98.6°F. We measure 25 healthy
adults and want to test if the population mean differs from 98.6°F.

\begin{itemize}
\tightlist
\item
  \textbf{H₀:} μ = 98.6°F
\item
  \textbf{H₁:} μ ≠ 98.6°F
\end{itemize}

\end{tcolorbox}

\[t = \frac{\bar{x} - \mu_0}{s/\sqrt{n}}\] \[df = n - 1\]

\subsection{Two-Sample t-Test (Equal
Variances)}\label{two-sample-t-test-equal-variances}

Compares means between two independent groups, assuming equal population
variances.

\begin{tcolorbox}[enhanced jigsaw, toprule=.15mm, left=2mm, opacitybacktitle=0.6, colframe=quarto-callout-note-color-frame, leftrule=.75mm, titlerule=0mm, coltitle=black, colbacktitle=quarto-callout-note-color!10!white, toptitle=1mm, title=\textcolor{quarto-callout-note-color}{\faInfo}\hspace{0.5em}{Assumptions for Two-Sample t-Test}, bottomtitle=1mm, arc=.35mm, rightrule=.15mm, bottomrule=.15mm, breakable, opacityback=0, colback=white]

\begin{itemize}
\tightlist
\item
  Both samples from normal distributions
\item
  Independent observations
\item
  Equal population variances (homoscedasticity)
\end{itemize}

\end{tcolorbox}

The \textbf{pooled standard deviation} combines information from both
samples:

\[s_{pooled} = \sqrt{\frac{(n_1-1)s_1^2 + (n_2-1)s_2^2}{n_1+n_2-2}}\]

\[t = \frac{\bar{x}_1 - \bar{x}_2}{s_{pooled} \times \sqrt{\frac{1}{n_1} + \frac{1}{n_2}}}\]

\begin{tcolorbox}[enhanced jigsaw, toprule=.15mm, left=2mm, opacitybacktitle=0.6, colframe=quarto-callout-tip-color-frame, leftrule=.75mm, titlerule=0mm, coltitle=black, colbacktitle=quarto-callout-tip-color!10!white, toptitle=1mm, title=\textcolor{quarto-callout-tip-color}{\faLightbulb}\hspace{0.5em}{Clinical Trial Example}, bottomtitle=1mm, arc=.35mm, rightrule=.15mm, bottomrule=.15mm, breakable, opacityback=0, colback=white]

Comparing pain scores between treatment and control groups:

\begin{itemize}
\tightlist
\item
  \textbf{Treatment group:} n=20, mean=4.2, sd=1.8
\item
  \textbf{Control group:} n=18, mean=6.1, sd=2.1
\item
  \textbf{Question:} Does treatment reduce pain scores?
\end{itemize}

\end{tcolorbox}

\subsection{When to Use Pooled vs Unpooled (Welch's)
t-Test}\label{when-to-use-pooled-vs-unpooled-welchs-t-test}

\begin{tcolorbox}[enhanced jigsaw, toprule=.15mm, left=2mm, opacitybacktitle=0.6, colframe=quarto-callout-warning-color-frame, leftrule=.75mm, titlerule=0mm, coltitle=black, colbacktitle=quarto-callout-warning-color!10!white, toptitle=1mm, title=\textcolor{quarto-callout-warning-color}{\faExclamationTriangle}\hspace{0.5em}{Testing Equal Variances}, bottomtitle=1mm, arc=.35mm, rightrule=.15mm, bottomrule=.15mm, breakable, opacityback=0, colback=white]

Use F-test or Levene's test to check equal variance assumption. If p
\textgreater{} 0.05, variances are likely equal and pooled t-test is
appropriate.

\end{tcolorbox}

The pooled t-test is more powerful when variances are truly equal, but
Welch's t-test is more robust when they're not.

\bookmarksetup{startatroot}

\chapter{Comparative Inference}\label{comparative-inference}

Advanced techniques for comparing groups when traditional assumptions
don't hold, ensuring robust statistical inference across diverse
research scenarios.

\section{Welch's t-Test \& Paired
t-Test}\label{welchs-t-test-paired-t-test}

When the equal variance assumption is violated or when data comes in
natural pairs, we need specialized approaches to maintain valid
statistical inference.

\subsection{Welch's t-Test (Unequal
Variances)}\label{welchs-t-test-unequal-variances}

When group variances differ substantially, the pooled t-test can give
misleading results. Welch's t-test adjusts for unequal variances.

\[t = \frac{\bar{x}_1 - \bar{x}_2}{\sqrt{\frac{s_1^2}{n_1} + \frac{s_2^2}{n_2}}}\]

\[df = \frac{\left(\frac{s_1^2}{n_1} + \frac{s_2^2}{n_2}\right)^2}{\frac{\left(\frac{s_1^2}{n_1}\right)^2}{n_1-1} + \frac{\left(\frac{s_2^2}{n_2}\right)^2}{n_2-1}}\]

\begin{tcolorbox}[enhanced jigsaw, toprule=.15mm, left=2mm, opacitybacktitle=0.6, colframe=quarto-callout-tip-color-frame, leftrule=.75mm, titlerule=0mm, coltitle=black, colbacktitle=quarto-callout-tip-color!10!white, toptitle=1mm, title=\textcolor{quarto-callout-tip-color}{\faLightbulb}\hspace{0.5em}{When Variances Differ}, bottomtitle=1mm, arc=.35mm, rightrule=.15mm, bottomrule=.15mm, breakable, opacityback=0, colback=white]

Comparing recovery times between two surgical procedures:

\begin{itemize}
\tightlist
\item
  \textbf{Procedure A:} mean=7 days, sd=2 days (low variability)
\item
  \textbf{Procedure B:} mean=8 days, sd=5 days (high variability)
\end{itemize}

The large difference in standard deviations suggests using Welch's
t-test.

\end{tcolorbox}

\emph{Note: This chapter appears to be incomplete in the source HTML
file. Additional content for paired t-tests and other comparative
inference methods would typically be included here.}

\bookmarksetup{startatroot}

\chapter{Multi-Group Analysis}\label{multi-group-analysis}

When research involves comparing three or more groups, specialized
techniques prevent inflated error rates while maintaining statistical
power.

\section{Introduction to ANOVA (Analysis of
Variance)}\label{introduction-to-anova-analysis-of-variance}

When comparing means across three or more groups, using multiple t-tests
creates a serious statistical problem: \textbf{multiple comparisons}.
ANOVA provides an elegant solution by testing all groups simultaneously
while controlling error rates.

\begin{tcolorbox}[enhanced jigsaw, toprule=.15mm, left=2mm, opacitybacktitle=0.6, colframe=quarto-callout-warning-color-frame, leftrule=.75mm, titlerule=0mm, coltitle=black, colbacktitle=quarto-callout-warning-color!10!white, toptitle=1mm, title=\textcolor{quarto-callout-warning-color}{\faExclamationTriangle}\hspace{0.5em}{Why Not Multiple t-Tests?}, bottomtitle=1mm, arc=.35mm, rightrule=.15mm, bottomrule=.15mm, breakable, opacityback=0, colback=white]

With 3 groups, you'd need 3 t-tests. With α = 0.05 for each test:

\begin{itemize}
\tightlist
\item
  \textbf{Individual error rate:} 5\% per test
\item
  \textbf{Family-wise error rate:} 1 - (0.95)³ = 14.3\%
\item
  \textbf{Problem:} Much higher chance of false positives!
\end{itemize}

With 5 groups, you'd need 10 tests, inflating error rate to
\textasciitilde40\%!

\end{tcolorbox}

\begin{tcolorbox}[enhanced jigsaw, toprule=.15mm, left=2mm, opacitybacktitle=0.6, colframe=quarto-callout-note-color-frame, leftrule=.75mm, titlerule=0mm, coltitle=black, colbacktitle=quarto-callout-note-color!10!white, toptitle=1mm, title=\textcolor{quarto-callout-note-color}{\faInfo}\hspace{0.5em}{The Logic of ANOVA}, bottomtitle=1mm, arc=.35mm, rightrule=.15mm, bottomrule=.15mm, breakable, opacityback=0, colback=white]

ANOVA compares two sources of variation:

\begin{itemize}
\tightlist
\item
  \textbf{Between-groups variation:} Differences due to treatment
  effects
\item
  \textbf{Within-groups variation:} Random variation (noise)
\end{itemize}

\textbf{Key Insight:} If treatment has no effect, both sources should be
similar. If there's a real effect, between-groups variation will be much
larger.

\end{tcolorbox}

\subsection{ANOVA Hypotheses}\label{anova-hypotheses}

\textbf{Null Hypothesis (H₀):} All group means are equal H₀: μ₁ = μ₂ =
μ₃ = \ldots{} = μₖ

\textbf{Alternative Hypothesis (H₁):} At least one group mean differs
H₁: Not all μᵢ are equal

\begin{tcolorbox}[enhanced jigsaw, toprule=.15mm, left=2mm, opacitybacktitle=0.6, colframe=quarto-callout-tip-color-frame, leftrule=.75mm, titlerule=0mm, coltitle=black, colbacktitle=quarto-callout-tip-color!10!white, toptitle=1mm, title=\textcolor{quarto-callout-tip-color}{\faLightbulb}\hspace{0.5em}{Medical Research Example}, bottomtitle=1mm, arc=.35mm, rightrule=.15mm, bottomrule=.15mm, breakable, opacityback=0, colback=white]

\textbf{Research Question:} Do three different pain medications have
different effectiveness?

\begin{itemize}
\tightlist
\item
  \textbf{Group 1:} Medication A (n=25)
\item
  \textbf{Group 2:} Medication B (n=23)
\item
  \textbf{Group 3:} Placebo (n=27)
\item
  \textbf{Outcome:} Pain reduction score (0-10 scale)
\end{itemize}

\textbf{H₀:} μₐ = μᵦ = μₚₗₐᶜₑᵦₒ (all medications equally effective)
\textbf{H₁:} At least one medication differs in effectiveness

\end{tcolorbox}

\subsection{The F-Statistic}\label{the-f-statistic}

ANOVA uses the F-statistic to compare variances:

\[F = \frac{\text{MSB}}{\text{MSW}}\]

where: - MSB = Mean Square Between groups - MSW = Mean Square Within
groups

\begin{tcolorbox}[enhanced jigsaw, toprule=.15mm, left=2mm, opacitybacktitle=0.6, colframe=quarto-callout-note-color-frame, leftrule=.75mm, titlerule=0mm, coltitle=black, colbacktitle=quarto-callout-note-color!10!white, toptitle=1mm, title=\textcolor{quarto-callout-note-color}{\faInfo}\hspace{0.5em}{Understanding the F-Ratio}, bottomtitle=1mm, arc=.35mm, rightrule=.15mm, bottomrule=.15mm, breakable, opacityback=0, colback=white]

\begin{itemize}
\tightlist
\item
  \textbf{F ≈ 1:} Between-group variation similar to within-group
  variation → No treatment effect
\item
  \textbf{F \textgreater\textgreater{} 1:} Between-group variation much
  larger → Likely treatment effect
\item
  \textbf{F \textless{} 1:} Rare, suggests less variation between groups
  than within (unusual)
\end{itemize}

\end{tcolorbox}

\subsection{ANOVA Assumptions}\label{anova-assumptions}

\begin{tcolorbox}[enhanced jigsaw, toprule=.15mm, left=2mm, opacitybacktitle=0.6, colframe=quarto-callout-warning-color-frame, leftrule=.75mm, titlerule=0mm, coltitle=black, colbacktitle=quarto-callout-warning-color!10!white, toptitle=1mm, title=\textcolor{quarto-callout-warning-color}{\faExclamationTriangle}\hspace{0.5em}{Critical Assumptions}, bottomtitle=1mm, arc=.35mm, rightrule=.15mm, bottomrule=.15mm, breakable, opacityback=0, colback=white]

\begin{itemize}
\tightlist
\item
  \textbf{Independence:} Observations must be independent
\item
  \textbf{Normality:} Data in each group should be approximately normal
\item
  \textbf{Homoscedasticity:} Equal variances across all groups
\end{itemize}

\textbf{Violation Consequences:} Increased Type I error, reduced power,
invalid conclusions

\end{tcolorbox}

\begin{tcolorbox}[enhanced jigsaw, toprule=.15mm, left=2mm, opacitybacktitle=0.6, colframe=quarto-callout-tip-color-frame, leftrule=.75mm, titlerule=0mm, coltitle=black, colbacktitle=quarto-callout-tip-color!10!white, toptitle=1mm, title=\textcolor{quarto-callout-tip-color}{\faLightbulb}\hspace{0.5em}{Checking Assumptions in Practice}, bottomtitle=1mm, arc=.35mm, rightrule=.15mm, bottomrule=.15mm, breakable, opacityback=0, colback=white]

\textbf{Independence:} Ensure proper randomization and no clustering

\textbf{Normality:} - Visual: Q-Q plots, histograms for each group -
Statistical: Shapiro-Wilk test (if n \textless{} 50) - Robust: ANOVA
reasonably robust with n \textgreater{} 30 per group

\textbf{Equal Variances:} - Visual: Box plots, side-by-side -
Statistical: Levene's test (preferred), Bartlett's test - Rule of thumb:
Largest SD / Smallest SD \textless{} 2

\end{tcolorbox}

\section{One-Way ANOVA: Step-by-Step}\label{one-way-anova-step-by-step}

One-way ANOVA compares means across groups defined by a single
categorical variable (factor).

\subsection{ANOVA Table Components}\label{anova-table-components}

\begin{tcolorbox}[enhanced jigsaw, toprule=.15mm, left=2mm, opacitybacktitle=0.6, colframe=quarto-callout-note-color-frame, leftrule=.75mm, titlerule=0mm, coltitle=black, colbacktitle=quarto-callout-note-color!10!white, toptitle=1mm, title=\textcolor{quarto-callout-note-color}{\faInfo}\hspace{0.5em}{Understanding ANOVA Table}, bottomtitle=1mm, arc=.35mm, rightrule=.15mm, bottomrule=.15mm, breakable, opacityback=0, colback=white]

\begin{longtable}[]{@{}lllll@{}}
\toprule\noalign{}
Source & df & SS & MS & F \\
\midrule\noalign{}
\endhead
\bottomrule\noalign{}
\endlastfoot
Between & k-1 & SSB & MSB & MSB/MSW \\
Within & N-k & SSW & MSW & - \\
Total & N-1 & SST & - & - \\
\end{longtable}

\textbf{Where:} k = number of groups, N = total sample size

\end{tcolorbox}

\subsection{Calculating Sum of
Squares}\label{calculating-sum-of-squares}

\textbf{Total Sum of Squares (SST):}
\[\text{SST} = \sum_i(x_i - \bar{x})^2\]

\textbf{Between Groups Sum of Squares (SSB):}
\[\text{SSB} = \sum_j n_j(\bar{x}_j - \bar{x})^2\]

\textbf{Within Groups Sum of Squares (SSW):}
\[\text{SSW} = \text{SST} - \text{SSB}\]

\begin{tcolorbox}[enhanced jigsaw, toprule=.15mm, left=2mm, opacitybacktitle=0.6, colframe=quarto-callout-tip-color-frame, leftrule=.75mm, titlerule=0mm, coltitle=black, colbacktitle=quarto-callout-tip-color!10!white, toptitle=1mm, title=\textcolor{quarto-callout-tip-color}{\faLightbulb}\hspace{0.5em}{Worked Example: Pain Medication Study}, bottomtitle=1mm, arc=.35mm, rightrule=.15mm, bottomrule=.15mm, breakable, opacityback=0, colback=white]

\textbf{Data:} - \textbf{Medication A:} 7, 8, 6, 9, 7 (n₁=5, x̄₁=7.4) -
\textbf{Medication B:} 5, 6, 4, 7, 5 (n₂=5, x̄₂=5.4) - \textbf{Placebo:}
3, 4, 2, 5, 3 (n₃=5, x̄₃=3.4)

\textbf{Step 1: Calculate overall mean} x̄ = (7.4×5 + 5.4×5 + 3.4×5) / 15
= 81/15 = 5.4

\textbf{Step 2: Calculate SSB} SSB = 5×(7.4-5.4)² + 5×(5.4-5.4)² +
5×(3.4-5.4)² SSB = 5×4 + 5×0 + 5×4 = 40

\textbf{Step 3: Calculate SST} SST = (7-5.4)² + (8-5.4)² + \ldots{} +
(3-5.4)² = 56

\textbf{Step 4: Calculate SSW} SSW = SST - SSB = 56 - 40 = 16

\textbf{Step 5: Calculate Mean Squares} MSB = SSB/(k-1) = 40/(3-1) = 20
MSW = SSW/(N-k) = 16/(15-3) = 1.33

\textbf{Step 6: Calculate F-statistic} F = MSB/MSW = 20/1.33 = 15.0

\textbf{Step 7: Compare to critical value} F₀.₀₅,₂,₁₂ = 3.89. Since 15.0
\textgreater{} 3.89, p \textless{} 0.05 \textbf{Conclusion:} Reject H₀.
At least one medication differs significantly.

\end{tcolorbox}

\section{Post-Hoc Tests and Multiple
Comparisons}\label{post-hoc-tests-and-multiple-comparisons}

When ANOVA shows significant differences, \textbf{post-hoc tests}
determine which specific groups differ from each other.

\begin{tcolorbox}[enhanced jigsaw, toprule=.15mm, left=2mm, opacitybacktitle=0.6, colframe=quarto-callout-note-color-frame, leftrule=.75mm, titlerule=0mm, coltitle=black, colbacktitle=quarto-callout-note-color!10!white, toptitle=1mm, title=\textcolor{quarto-callout-note-color}{\faInfo}\hspace{0.5em}{Why Post-Hoc Tests?}, bottomtitle=1mm, arc=.35mm, rightrule=.15mm, bottomrule=.15mm, breakable, opacityback=0, colback=white]

ANOVA only tells us that ``at least one group differs'' but not:

\begin{itemize}
\tightlist
\item
  Which specific groups are different?
\item
  How many groups are different?
\item
  The magnitude of differences?
\end{itemize}

Post-hoc tests provide pairwise comparisons while controlling
family-wise error rate.

\end{tcolorbox}

\subsection{Common Post-Hoc Tests}\label{common-post-hoc-tests}

\begin{tcolorbox}[enhanced jigsaw, toprule=.15mm, left=2mm, opacitybacktitle=0.6, colframe=quarto-callout-tip-color-frame, leftrule=.75mm, titlerule=0mm, coltitle=black, colbacktitle=quarto-callout-tip-color!10!white, toptitle=1mm, title=\textcolor{quarto-callout-tip-color}{\faLightbulb}\hspace{0.5em}{Tukey's HSD (Honestly Significant Difference)}, bottomtitle=1mm, arc=.35mm, rightrule=.15mm, bottomrule=.15mm, breakable, opacityback=0, colback=white]

\textbf{When to use:} Most common, good balance of power and control
\textbf{Formula:} HSD = q × √(MSW/n) \textbf{Advantage:} Controls
family-wise error rate at α \textbf{Disadvantage:} Requires equal sample
sizes (or harmonic mean)

\textbf{Interpretation:} If \textbar x̄ᵢ - x̄ⱼ\textbar{} \textgreater{}
HSD, groups i and j differ significantly

\end{tcolorbox}

\begin{tcolorbox}[enhanced jigsaw, toprule=.15mm, left=2mm, opacitybacktitle=0.6, colframe=quarto-callout-tip-color-frame, leftrule=.75mm, titlerule=0mm, coltitle=black, colbacktitle=quarto-callout-tip-color!10!white, toptitle=1mm, title=\textcolor{quarto-callout-tip-color}{\faLightbulb}\hspace{0.5em}{Bonferroni Correction}, bottomtitle=1mm, arc=.35mm, rightrule=.15mm, bottomrule=.15mm, breakable, opacityback=0, colback=white]

\textbf{When to use:} Conservative approach, few planned comparisons
\textbf{Method:} Use α/c for each comparison (c = number of comparisons)
\textbf{Example:} With 3 groups, 3 comparisons, use α = 0.05/3 = 0.017
\textbf{Advantage:} Simple, very conservative \textbf{Disadvantage:} Can
be overly conservative, reduced power

\end{tcolorbox}

\begin{tcolorbox}[enhanced jigsaw, toprule=.15mm, left=2mm, opacitybacktitle=0.6, colframe=quarto-callout-tip-color-frame, leftrule=.75mm, titlerule=0mm, coltitle=black, colbacktitle=quarto-callout-tip-color!10!white, toptitle=1mm, title=\textcolor{quarto-callout-tip-color}{\faLightbulb}\hspace{0.5em}{Scheffé's Test}, bottomtitle=1mm, arc=.35mm, rightrule=.15mm, bottomrule=.15mm, breakable, opacityback=0, colback=white]

\textbf{When to use:} Any possible comparison, not just pairwise
\textbf{Advantage:} Most flexible, allows complex contrasts
\textbf{Disadvantage:} Most conservative for simple pairwise comparisons

\end{tcolorbox}

\subsection{Continuing Our Pain Medication
Example}\label{continuing-our-pain-medication-example}

\begin{tcolorbox}[enhanced jigsaw, toprule=.15mm, left=2mm, opacitybacktitle=0.6, colframe=quarto-callout-tip-color-frame, leftrule=.75mm, titlerule=0mm, coltitle=black, colbacktitle=quarto-callout-tip-color!10!white, toptitle=1mm, title=\textcolor{quarto-callout-tip-color}{\faLightbulb}\hspace{0.5em}{Post-Hoc Analysis}, bottomtitle=1mm, arc=.35mm, rightrule=.15mm, bottomrule=.15mm, breakable, opacityback=0, colback=white]

We found F = 15.0, p \textless{} 0.05. Now let's determine which
medications differ:

\textbf{Tukey's HSD Calculation:} HSD = q₀.₀₅,₃,₁₂ × √(MSW/n) = 3.77 ×
√(1.33/5) = 1.94

\textbf{Pairwise Comparisons:} - \textbar Med A - Med B\textbar{} =
\textbar7.4 - 5.4\textbar{} = 2.0 \textgreater{} 1.94 ✓
\textbf{Significant} - \textbar Med A - Placebo\textbar{} = \textbar7.4
- 3.4\textbar{} = 4.0 \textgreater{} 1.94 ✓ \textbf{Significant} -
\textbar Med B - Placebo\textbar{} = \textbar5.4 - 3.4\textbar{} = 2.0
\textgreater{} 1.94 ✓ \textbf{Significant}

\textbf{Conclusion:} All three treatments differ significantly from each
other. \textbf{Clinical Interpretation:} Med A \textgreater{} Med B
\textgreater{} Placebo for pain reduction.

\end{tcolorbox}

\begin{tcolorbox}[enhanced jigsaw, toprule=.15mm, left=2mm, opacitybacktitle=0.6, colframe=quarto-callout-warning-color-frame, leftrule=.75mm, titlerule=0mm, coltitle=black, colbacktitle=quarto-callout-warning-color!10!white, toptitle=1mm, title=\textcolor{quarto-callout-warning-color}{\faExclamationTriangle}\hspace{0.5em}{Common Mistakes in Post-Hoc Testing}, bottomtitle=1mm, arc=.35mm, rightrule=.15mm, bottomrule=.15mm, breakable, opacityback=0, colback=white]

\begin{itemize}
\tightlist
\item
  \textbf{❌ Running post-hocs when ANOVA is not significant} → Fishing
  for significance
\item
  \textbf{❌ Using multiple different post-hoc tests} → Inflated error
  rates
\item
  \textbf{❌ Ignoring practical significance} → Statistical ≠ clinical
  significance
\item
  \textbf{✅ Choose post-hoc test before analysis} → Avoid bias
\item
  \textbf{✅ Report effect sizes} → Practical importance
\end{itemize}

\end{tcolorbox}

\section{Two-Way ANOVA}\label{two-way-anova}

Two-way ANOVA examines the effects of two categorical variables
(factors) simultaneously, including their potential interaction.

\begin{tcolorbox}[enhanced jigsaw, toprule=.15mm, left=2mm, opacitybacktitle=0.6, colframe=quarto-callout-note-color-frame, leftrule=.75mm, titlerule=0mm, coltitle=black, colbacktitle=quarto-callout-note-color!10!white, toptitle=1mm, title=\textcolor{quarto-callout-note-color}{\faInfo}\hspace{0.5em}{Advantages of Two-Way ANOVA}, bottomtitle=1mm, arc=.35mm, rightrule=.15mm, bottomrule=.15mm, breakable, opacityback=0, colback=white]

\begin{itemize}
\tightlist
\item
  \textbf{Efficiency:} Tests multiple factors in one analysis
\item
  \textbf{Interaction Detection:} Identifies when factors work together
\item
  \textbf{Control:} Accounts for additional sources of variation
\item
  \textbf{Power:} Often more powerful than separate one-way ANOVAs
\end{itemize}

\end{tcolorbox}

\subsection{Three Research Questions}\label{three-research-questions}

Two-way ANOVA addresses three hypotheses simultaneously:

\textbf{Main Effect of Factor A:} H₀: μ₁• = μ₂• = \ldots{} = μₐ•
\textbf{Main Effect of Factor B:} H₀: μ•₁ = μ•₂ = \ldots{} = μ•ᵦ
\textbf{Interaction Effect:} H₀: No A×B interaction

\begin{tcolorbox}[enhanced jigsaw, toprule=.15mm, left=2mm, opacitybacktitle=0.6, colframe=quarto-callout-tip-color-frame, leftrule=.75mm, titlerule=0mm, coltitle=black, colbacktitle=quarto-callout-tip-color!10!white, toptitle=1mm, title=\textcolor{quarto-callout-tip-color}{\faLightbulb}\hspace{0.5em}{Clinical Trial Example: Drug and Exercise}, bottomtitle=1mm, arc=.35mm, rightrule=.15mm, bottomrule=.15mm, breakable, opacityback=0, colback=white]

\textbf{Research Question:} Do cholesterol-lowering drugs and exercise
programs interact?

\textbf{Factor A (Drug):} New drug vs.~Placebo \textbf{Factor B
(Exercise):} High intensity vs.~Low intensity \textbf{Outcome:}
Cholesterol reduction (mg/dL)

\textbf{Design:} 2×2 factorial design - Group 1: New drug + High
exercise (n=20) - Group 2: New drug + Low exercise (n=20) - Group 3:
Placebo + High exercise (n=20) - Group 4: Placebo + Low exercise (n=20)

\end{tcolorbox}

\subsection{Understanding
Interactions}\label{understanding-interactions}

\begin{tcolorbox}[enhanced jigsaw, toprule=.15mm, left=2mm, opacitybacktitle=0.6, colframe=quarto-callout-note-color-frame, leftrule=.75mm, titlerule=0mm, coltitle=black, colbacktitle=quarto-callout-note-color!10!white, toptitle=1mm, title=\textcolor{quarto-callout-note-color}{\faInfo}\hspace{0.5em}{Types of Interaction Effects}, bottomtitle=1mm, arc=.35mm, rightrule=.15mm, bottomrule=.15mm, breakable, opacityback=0, colback=white]

\textbf{No Interaction:} Effect of Factor A is the same at all levels of
Factor B \textbf{Ordinal Interaction:} Effect of Factor A varies in
magnitude but not direction \textbf{Disordinal Interaction:} Effect of
Factor A changes direction at different levels of Factor B

\end{tcolorbox}

\begin{tcolorbox}[enhanced jigsaw, toprule=.15mm, left=2mm, opacitybacktitle=0.6, colframe=quarto-callout-tip-color-frame, leftrule=.75mm, titlerule=0mm, coltitle=black, colbacktitle=quarto-callout-tip-color!10!white, toptitle=1mm, title=\textcolor{quarto-callout-tip-color}{\faLightbulb}\hspace{0.5em}{Interpreting Interaction: Hypothetical Results}, bottomtitle=1mm, arc=.35mm, rightrule=.15mm, bottomrule=.15mm, breakable, opacityback=0, colback=white]

\textbf{Scenario 1: No Interaction}

\begin{longtable}[]{@{}llll@{}}
\toprule\noalign{}
& High Exercise & Low Exercise & Difference \\
\midrule\noalign{}
\endhead
\bottomrule\noalign{}
\endlastfoot
New Drug & 40 mg/dL & 30 mg/dL & 10 mg/dL \\
Placebo & 20 mg/dL & 10 mg/dL & 10 mg/dL \\
\end{longtable}

\textbf{Interpretation:} Exercise always improves cholesterol by 10
mg/dL, regardless of drug

\textbf{Scenario 2: Significant Interaction}

\begin{longtable}[]{@{}llll@{}}
\toprule\noalign{}
& High Exercise & Low Exercise & Difference \\
\midrule\noalign{}
\endhead
\bottomrule\noalign{}
\endlastfoot
New Drug & 50 mg/dL & 25 mg/dL & 25 mg/dL \\
Placebo & 15 mg/dL & 10 mg/dL & 5 mg/dL \\
\end{longtable}

\textbf{Interpretation:} The drug works much better with high exercise
(synergistic effect)

\end{tcolorbox}

\begin{tcolorbox}[enhanced jigsaw, toprule=.15mm, left=2mm, opacitybacktitle=0.6, colframe=quarto-callout-warning-color-frame, leftrule=.75mm, titlerule=0mm, coltitle=black, colbacktitle=quarto-callout-warning-color!10!white, toptitle=1mm, title=\textcolor{quarto-callout-warning-color}{\faExclamationTriangle}\hspace{0.5em}{Interpreting Two-Way ANOVA Results}, bottomtitle=1mm, arc=.35mm, rightrule=.15mm, bottomrule=.15mm, breakable, opacityback=0, colback=white]

\textbf{If interaction is significant:} - Focus primarily on interaction
interpretation - Main effects may be misleading - Use simple effects
analysis or interaction contrasts

\textbf{If interaction is not significant:} - Interpret main effects
independently - Each factor's effect is consistent across levels of the
other

\end{tcolorbox}

\section{ANOVA in Practice: Assumptions and
Alternatives}\label{anova-in-practice-assumptions-and-alternatives}

\subsection{When ANOVA Assumptions
Fail}\label{when-anova-assumptions-fail}

\begin{tcolorbox}[enhanced jigsaw, toprule=.15mm, left=2mm, opacitybacktitle=0.6, colframe=quarto-callout-warning-color-frame, leftrule=.75mm, titlerule=0mm, coltitle=black, colbacktitle=quarto-callout-warning-color!10!white, toptitle=1mm, title=\textcolor{quarto-callout-warning-color}{\faExclamationTriangle}\hspace{0.5em}{Assumption Violations and Solutions}, bottomtitle=1mm, arc=.35mm, rightrule=.15mm, bottomrule=.15mm, breakable, opacityback=0, colback=white]

\textbf{Non-normality:} - \textbf{Mild violation:} ANOVA is robust
(central limit theorem) - \textbf{Severe violation:} Use Kruskal-Wallis
test (non-parametric) - \textbf{Transformation:} Log, square root, or
Box-Cox transformations

\textbf{Unequal variances:} - \textbf{Welch's ANOVA:} Doesn't assume
equal variances - \textbf{Brown-Forsythe test:} Robust to variance
differences - \textbf{Transformation:} Often stabilizes variances

\textbf{Non-independence:} - \textbf{Most serious violation} → Inflated
Type I error - \textbf{Solutions:} Mixed-effects models, cluster-robust
standard errors - \textbf{Design fix:} Proper randomization, account for
clustering

\end{tcolorbox}

\subsection{Effect Size and Practical
Significance}\label{effect-size-and-practical-significance}

\begin{tcolorbox}[enhanced jigsaw, toprule=.15mm, left=2mm, opacitybacktitle=0.6, colframe=quarto-callout-note-color-frame, leftrule=.75mm, titlerule=0mm, coltitle=black, colbacktitle=quarto-callout-note-color!10!white, toptitle=1mm, title=\textcolor{quarto-callout-note-color}{\faInfo}\hspace{0.5em}{Effect Size Measures for ANOVA}, bottomtitle=1mm, arc=.35mm, rightrule=.15mm, bottomrule=.15mm, breakable, opacityback=0, colback=white]

\textbf{Eta-squared (η²):} Proportion of total variance explained η² =
SSB / SST

\textbf{Partial Eta-squared:} More common in complex designs ηₚ² = SSB /
(SSB + SSW)

\textbf{Cohen's Guidelines:} - Small effect: η² = 0.01 - Medium effect:
η² = 0.06 - Large effect: η² = 0.14

\end{tcolorbox}

\begin{tcolorbox}[enhanced jigsaw, toprule=.15mm, left=2mm, opacitybacktitle=0.6, colframe=quarto-callout-tip-color-frame, leftrule=.75mm, titlerule=0mm, coltitle=black, colbacktitle=quarto-callout-tip-color!10!white, toptitle=1mm, title=\textcolor{quarto-callout-tip-color}{\faLightbulb}\hspace{0.5em}{Reporting ANOVA Results}, bottomtitle=1mm, arc=.35mm, rightrule=.15mm, bottomrule=.15mm, breakable, opacityback=0, colback=white]

\textbf{Complete ANOVA Report:}

``A one-way ANOVA was conducted to compare the effectiveness of three
pain medications. The analysis revealed a statistically significant
difference between groups, F(2, 12) = 15.0, p \textless{} 0.001, η² =
0.71, indicating a large effect size.

Post-hoc comparisons using Tukey's HSD test indicated that all pairwise
comparisons were statistically significant (p \textless{} 0.05).
Medication A (M = 7.4, SD = 1.1) was more effective than Medication B (M
= 5.4, SD = 1.1), which was more effective than Placebo (M = 3.4, SD =
1.1).''

\end{tcolorbox}

\subsection{Power Analysis for ANOVA}\label{power-analysis-for-anova}

\begin{tcolorbox}[enhanced jigsaw, toprule=.15mm, left=2mm, opacitybacktitle=0.6, colframe=quarto-callout-note-color-frame, leftrule=.75mm, titlerule=0mm, coltitle=black, colbacktitle=quarto-callout-note-color!10!white, toptitle=1mm, title=\textcolor{quarto-callout-note-color}{\faInfo}\hspace{0.5em}{Sample Size Planning}, bottomtitle=1mm, arc=.35mm, rightrule=.15mm, bottomrule=.15mm, breakable, opacityback=0, colback=white]

Power analysis helps determine appropriate sample sizes before
conducting research:

\begin{itemize}
\tightlist
\item
  \textbf{Specify:} Effect size, significance level (α), desired power
\item
  \textbf{Effect size:} Based on pilot data or literature
\item
  \textbf{Common power:} 0.80 (80\% chance of detecting true effect)
\item
  \textbf{Balance:} Equal sample sizes maximize power
\end{itemize}

\end{tcolorbox}

\begin{tcolorbox}[enhanced jigsaw, toprule=.15mm, left=2mm, opacitybacktitle=0.6, colframe=quarto-callout-tip-color-frame, leftrule=.75mm, titlerule=0mm, coltitle=black, colbacktitle=quarto-callout-tip-color!10!white, toptitle=1mm, title=\textcolor{quarto-callout-tip-color}{\faLightbulb}\hspace{0.5em}{Power Analysis Example}, bottomtitle=1mm, arc=.35mm, rightrule=.15mm, bottomrule=.15mm, breakable, opacityback=0, colback=white]

\textbf{Research Planning:} Comparing 4 treatments for depression

\begin{itemize}
\tightlist
\item
  \textbf{Expected effect size:} f = 0.25 (medium effect)
\item
  \textbf{Significance level:} α = 0.05
\item
  \textbf{Desired power:} 0.80
\item
  \textbf{Result:} Need n = 45 per group (180 total)
\end{itemize}

\textbf{Interpretation:} With 45 participants per group, we have an 80\%
chance of detecting a medium-sized difference if it truly exists.

\end{tcolorbox}

\bookmarksetup{startatroot}

\chapter{Interrelationships \&
Modeling}\label{interrelationships-modeling}

Move beyond simple group comparisons to understand relationships between
variables and build predictive models.

\section{Introduction}\label{introduction}

In biostatistics, we often need to understand relationships between
variables and make predictions. This chapter covers fundamental
techniques for exploring associations, building predictive models, and
evaluating their performance in medical and biological contexts.

\section{Correlation Analysis}\label{correlation-analysis}

\subsection{Understanding Correlation}\label{understanding-correlation}

Correlation measures the strength and direction of a linear relationship
between two continuous variables. In biostatistics, we might examine
correlations between:

\begin{itemize}
\tightlist
\item
  Blood pressure and age
\item
  BMI and cholesterol levels
\item
  Gene expression levels between related genes
\end{itemize}

\subsection{Pearson Correlation
Coefficient}\label{pearson-correlation-coefficient}

The Pearson correlation coefficient (\(r\)) ranges from -1 to +1:

\[r = \frac{\sum_{i=1}^{n}(x_i - \bar{x})(y_i - \bar{y})}{\sqrt{\sum_{i=1}^{n}(x_i - \bar{x})^2 \sum_{i=1}^{n}(y_i - \bar{y})^2}}\]

Where: - \(r = 1\): Perfect positive linear relationship - \(r = 0\): No
linear relationship - \(r = -1\): Perfect negative linear relationship

\begin{tcolorbox}[enhanced jigsaw, toprule=.15mm, left=2mm, opacitybacktitle=0.6, colframe=quarto-callout-tip-color-frame, leftrule=.75mm, titlerule=0mm, coltitle=black, colbacktitle=quarto-callout-tip-color!10!white, toptitle=1mm, title=\textcolor{quarto-callout-tip-color}{\faLightbulb}\hspace{0.5em}{Interpreting Correlation Strength}, bottomtitle=1mm, arc=.35mm, rightrule=.15mm, bottomrule=.15mm, breakable, opacityback=0, colback=white]

\begin{itemize}
\tightlist
\item
  \textbar r\textbar{} \textless{} 0.3: Weak correlation
\item
  0.3 ≤ \textbar r\textbar{} \textless{} 0.7: Moderate correlation
\item
  \textbar r\textbar{} ≥ 0.7: Strong correlation
\end{itemize}

\end{tcolorbox}

\begin{Shaded}
\begin{Highlighting}[]
\CommentTok{\# Example: Correlation between age and blood pressure}
\FunctionTok{library}\NormalTok{(ggplot2)}

\CommentTok{\# Simulate data}
\FunctionTok{set.seed}\NormalTok{(}\DecValTok{123}\NormalTok{)}
\NormalTok{n }\OtherTok{\textless{}{-}} \DecValTok{100}
\NormalTok{age }\OtherTok{\textless{}{-}} \FunctionTok{rnorm}\NormalTok{(n, }\AttributeTok{mean =} \DecValTok{45}\NormalTok{, }\AttributeTok{sd =} \DecValTok{15}\NormalTok{)}
\NormalTok{bp\_systolic }\OtherTok{\textless{}{-}} \DecValTok{90} \SpecialCharTok{+} \FloatTok{0.8} \SpecialCharTok{*}\NormalTok{ age }\SpecialCharTok{+} \FunctionTok{rnorm}\NormalTok{(n, }\AttributeTok{sd =} \DecValTok{10}\NormalTok{)}

\CommentTok{\# Calculate correlation}
\NormalTok{cor\_result }\OtherTok{\textless{}{-}} \FunctionTok{cor.test}\NormalTok{(age, bp\_systolic)}
\FunctionTok{print}\NormalTok{(}\FunctionTok{paste}\NormalTok{(}\StringTok{"Correlation:"}\NormalTok{, }\FunctionTok{round}\NormalTok{(cor\_result}\SpecialCharTok{$}\NormalTok{estimate, }\DecValTok{3}\NormalTok{)))}
\FunctionTok{print}\NormalTok{(}\FunctionTok{paste}\NormalTok{(}\StringTok{"P{-}value:"}\NormalTok{, }\FunctionTok{round}\NormalTok{(cor\_result}\SpecialCharTok{$}\NormalTok{p.value, }\DecValTok{4}\NormalTok{)))}

\CommentTok{\# Visualize}
\FunctionTok{ggplot}\NormalTok{(}\FunctionTok{data.frame}\NormalTok{(age, bp\_systolic), }\FunctionTok{aes}\NormalTok{(}\AttributeTok{x =}\NormalTok{ age, }\AttributeTok{y =}\NormalTok{ bp\_systolic)) }\SpecialCharTok{+}
  \FunctionTok{geom\_point}\NormalTok{(}\AttributeTok{alpha =} \FloatTok{0.6}\NormalTok{) }\SpecialCharTok{+}
  \FunctionTok{geom\_smooth}\NormalTok{(}\AttributeTok{method =} \StringTok{"lm"}\NormalTok{, }\AttributeTok{color =} \StringTok{"red"}\NormalTok{) }\SpecialCharTok{+}
  \FunctionTok{labs}\NormalTok{(}\AttributeTok{title =} \StringTok{"Age vs Systolic Blood Pressure"}\NormalTok{,}
       \AttributeTok{x =} \StringTok{"Age (years)"}\NormalTok{, }\AttributeTok{y =} \StringTok{"Systolic BP (mmHg)"}\NormalTok{) }\SpecialCharTok{+}
  \FunctionTok{theme\_minimal}\NormalTok{()}
\end{Highlighting}
\end{Shaded}

\section{Simple Linear Regression}\label{simple-linear-regression}

\subsection{The Linear Model}\label{the-linear-model}

Simple linear regression models the relationship between a predictor
variable (X) and an outcome variable (Y):

\[Y_i = \beta_0 + \beta_1 X_i + \epsilon_i\]

Where: - \(\beta_0\): Intercept (value of Y when X = 0) - \(\beta_1\):
Slope (change in Y per unit change in X) - \(\epsilon_i\): Random error
term

\subsection{Example: Predicting Blood Pressure from
Age}\label{example-predicting-blood-pressure-from-age}

\begin{Shaded}
\begin{Highlighting}[]
\CommentTok{\# Fit linear regression model}
\NormalTok{model }\OtherTok{\textless{}{-}} \FunctionTok{lm}\NormalTok{(bp\_systolic }\SpecialCharTok{\textasciitilde{}}\NormalTok{ age)}
\FunctionTok{summary}\NormalTok{(model)}

\CommentTok{\# The model: bp\_systolic = β₀ + β₁ × age}
\CommentTok{\# Interpretation: For each additional year of age,}
\CommentTok{\# systolic BP increases by β₁ mmHg on average}
\end{Highlighting}
\end{Shaded}

\section{Logistic Regression}\label{logistic-regression}

\subsection{When to Use Logistic
Regression}\label{when-to-use-logistic-regression}

When your outcome variable is binary (diseased/healthy, survived/died,
positive/negative test), logistic regression is the appropriate method.

\subsection{The Logistic Model}\label{the-logistic-model}

Instead of predicting the outcome directly, logistic regression predicts
the log-odds:

\[\text{logit}(p) = \ln\left(\frac{p}{1-p}\right) = \beta_0 + \beta_1 X_1 + \beta_2 X_2 + ... + \beta_k X_k\]

Where \(p\) is the probability of the event occurring.

The probability is then:

\[p = \frac{e^{\beta_0 + \beta_1 X_1 + ... + \beta_k X_k}}{1 + e^{\beta_0 + \beta_1 X_1 + ... + \beta_k X_k}}\]

\subsection{Example: Predicting Heart Disease
Risk}\label{example-predicting-heart-disease-risk}

\begin{Shaded}
\begin{Highlighting}[]
\CommentTok{\# Simulate heart disease data}
\FunctionTok{set.seed}\NormalTok{(}\DecValTok{456}\NormalTok{)}
\NormalTok{n }\OtherTok{\textless{}{-}} \DecValTok{500}
\NormalTok{age }\OtherTok{\textless{}{-}} \FunctionTok{rnorm}\NormalTok{(n, }\DecValTok{55}\NormalTok{, }\DecValTok{12}\NormalTok{)}
\NormalTok{cholesterol }\OtherTok{\textless{}{-}} \FunctionTok{rnorm}\NormalTok{(n, }\DecValTok{200}\NormalTok{, }\DecValTok{40}\NormalTok{)}
\NormalTok{smoking }\OtherTok{\textless{}{-}} \FunctionTok{rbinom}\NormalTok{(n, }\DecValTok{1}\NormalTok{, }\FloatTok{0.3}\NormalTok{)}

\CommentTok{\# Create log{-}odds for heart disease}
\NormalTok{log\_odds }\OtherTok{\textless{}{-}} \SpecialCharTok{{-}}\DecValTok{5} \SpecialCharTok{+} \FloatTok{0.05}\SpecialCharTok{*}\NormalTok{age }\SpecialCharTok{+} \FloatTok{0.01}\SpecialCharTok{*}\NormalTok{cholesterol }\SpecialCharTok{+} \FloatTok{1.2}\SpecialCharTok{*}\NormalTok{smoking}
\NormalTok{prob\_disease }\OtherTok{\textless{}{-}} \FunctionTok{exp}\NormalTok{(log\_odds)}\SpecialCharTok{/}\NormalTok{(}\DecValTok{1} \SpecialCharTok{+} \FunctionTok{exp}\NormalTok{(log\_odds))}
\NormalTok{heart\_disease }\OtherTok{\textless{}{-}} \FunctionTok{rbinom}\NormalTok{(n, }\DecValTok{1}\NormalTok{, prob\_disease)}

\CommentTok{\# Fit logistic regression}
\NormalTok{logit\_model }\OtherTok{\textless{}{-}} \FunctionTok{glm}\NormalTok{(heart\_disease }\SpecialCharTok{\textasciitilde{}}\NormalTok{ age }\SpecialCharTok{+}\NormalTok{ cholesterol }\SpecialCharTok{+}\NormalTok{ smoking,}
                   \AttributeTok{family =}\NormalTok{ binomial)}
\FunctionTok{summary}\NormalTok{(logit\_model)}

\CommentTok{\# Interpretation: exp(coefficient) gives the odds ratio}
\FunctionTok{exp}\NormalTok{(}\FunctionTok{coef}\NormalTok{(logit\_model))}
\end{Highlighting}
\end{Shaded}

\section{Model Evaluation for
Classification}\label{model-evaluation-for-classification}

\subsection{The Confusion Matrix}\label{the-confusion-matrix}

A confusion matrix is a table that shows how well our classification
model performs. It's like a report card for prediction models.

\begin{tcolorbox}[enhanced jigsaw, toprule=.15mm, left=2mm, opacitybacktitle=0.6, colframe=quarto-callout-note-color-frame, leftrule=.75mm, titlerule=0mm, coltitle=black, colbacktitle=quarto-callout-note-color!10!white, toptitle=1mm, title=\textcolor{quarto-callout-note-color}{\faInfo}\hspace{0.5em}{Think of it this way}, bottomtitle=1mm, arc=.35mm, rightrule=.15mm, bottomrule=.15mm, breakable, opacityback=0, colback=white]

Imagine you're a doctor using a test to diagnose disease. The confusion
matrix shows: - How many sick patients you correctly identified (True
Positives) - How many healthy patients you correctly identified (True
Negatives) - How many times you missed the disease (False Negatives) -
How many times you incorrectly diagnosed disease (False Positives)

\end{tcolorbox}

\subsection{2×2 Confusion Matrix}\label{confusion-matrix}

\begin{longtable}[]{@{}lcc@{}}
\toprule\noalign{}
& \textbf{Predicted} & \\
\midrule\noalign{}
\endhead
\bottomrule\noalign{}
\endlastfoot
\textbf{Actual} & \textbf{Positive} & \textbf{Negative} \\
\textbf{Positive} & TP & FN \\
\textbf{Negative} & FP & TN \\
\end{longtable}

Where: - \textbf{TP (True Positives)}: Correctly predicted positive
cases - \textbf{TN (True Negatives)}: Correctly predicted negative cases
- \textbf{FP (False Positives)}: Incorrectly predicted positive (Type I
error) - \textbf{FN (False Negatives)}: Incorrectly predicted negative
(Type II error)

\subsection{Example: COVID-19 Test
Evaluation}\label{example-covid-19-test-evaluation}

Let's say we have a COVID-19 test and we tested 1000 people:

\begin{Shaded}
\begin{Highlighting}[]
\CommentTok{\# Create confusion matrix data}
\FunctionTok{library}\NormalTok{(caret)}

\CommentTok{\# Simulate COVID test results}
\FunctionTok{set.seed}\NormalTok{(}\DecValTok{789}\NormalTok{)}
\NormalTok{n }\OtherTok{\textless{}{-}} \DecValTok{1000}
\NormalTok{true\_status }\OtherTok{\textless{}{-}} \FunctionTok{c}\NormalTok{(}\FunctionTok{rep}\NormalTok{(}\StringTok{"Positive"}\NormalTok{, }\DecValTok{100}\NormalTok{), }\FunctionTok{rep}\NormalTok{(}\StringTok{"Negative"}\NormalTok{, }\DecValTok{900}\NormalTok{))  }\CommentTok{\# 10\% prevalence}
\NormalTok{test\_sensitivity }\OtherTok{\textless{}{-}} \FloatTok{0.95}  \CommentTok{\# 95\% of positive cases detected}
\NormalTok{test\_specificity }\OtherTok{\textless{}{-}} \FloatTok{0.98}  \CommentTok{\# 98\% of negative cases correctly identified}

\NormalTok{predicted\_status }\OtherTok{\textless{}{-}} \FunctionTok{ifelse}\NormalTok{(}
\NormalTok{  true\_status }\SpecialCharTok{==} \StringTok{"Positive"}\NormalTok{,}
  \FunctionTok{ifelse}\NormalTok{(}\FunctionTok{runif}\NormalTok{(}\FunctionTok{sum}\NormalTok{(true\_status }\SpecialCharTok{==} \StringTok{"Positive"}\NormalTok{)) }\SpecialCharTok{\textless{}}\NormalTok{ test\_sensitivity, }\StringTok{"Positive"}\NormalTok{, }\StringTok{"Negative"}\NormalTok{),}
  \FunctionTok{ifelse}\NormalTok{(}\FunctionTok{runif}\NormalTok{(}\FunctionTok{sum}\NormalTok{(true\_status }\SpecialCharTok{==} \StringTok{"Negative"}\NormalTok{)) }\SpecialCharTok{\textless{}}\NormalTok{ test\_specificity, }\StringTok{"Negative"}\NormalTok{, }\StringTok{"Positive"}\NormalTok{)}
\NormalTok{)}

\CommentTok{\# Create confusion matrix}
\NormalTok{cm }\OtherTok{\textless{}{-}} \FunctionTok{confusionMatrix}\NormalTok{(}\FunctionTok{as.factor}\NormalTok{(predicted\_status), }\FunctionTok{as.factor}\NormalTok{(true\_status), }\AttributeTok{positive =} \StringTok{"Positive"}\NormalTok{)}
\FunctionTok{print}\NormalTok{(cm)}

\CommentTok{\# Visualize confusion matrix}
\FunctionTok{library}\NormalTok{(ggplot2)}
\NormalTok{cm\_table }\OtherTok{\textless{}{-}} \FunctionTok{as.data.frame}\NormalTok{(cm}\SpecialCharTok{$}\NormalTok{table)}
\FunctionTok{ggplot}\NormalTok{(cm\_table, }\FunctionTok{aes}\NormalTok{(}\AttributeTok{x =}\NormalTok{ Reference, }\AttributeTok{y =}\NormalTok{ Prediction, }\AttributeTok{fill =}\NormalTok{ Freq)) }\SpecialCharTok{+}
  \FunctionTok{geom\_tile}\NormalTok{(}\AttributeTok{color =} \StringTok{"white"}\NormalTok{) }\SpecialCharTok{+}
  \FunctionTok{geom\_text}\NormalTok{(}\FunctionTok{aes}\NormalTok{(}\AttributeTok{label =}\NormalTok{ Freq), }\AttributeTok{size =} \DecValTok{12}\NormalTok{, }\AttributeTok{color =} \StringTok{"white"}\NormalTok{) }\SpecialCharTok{+}
  \FunctionTok{scale\_fill\_gradient}\NormalTok{(}\AttributeTok{low =} \StringTok{"lightblue"}\NormalTok{, }\AttributeTok{high =} \StringTok{"darkblue"}\NormalTok{) }\SpecialCharTok{+}
  \FunctionTok{labs}\NormalTok{(}\AttributeTok{title =} \StringTok{"COVID{-}19 Test Confusion Matrix"}\NormalTok{,}
       \AttributeTok{x =} \StringTok{"Actual Status"}\NormalTok{, }\AttributeTok{y =} \StringTok{"Predicted Status"}\NormalTok{) }\SpecialCharTok{+}
  \FunctionTok{theme\_minimal}\NormalTok{() }\SpecialCharTok{+}
  \FunctionTok{theme}\NormalTok{(}\AttributeTok{text =} \FunctionTok{element\_text}\NormalTok{(}\AttributeTok{size =} \DecValTok{12}\NormalTok{))}
\end{Highlighting}
\end{Shaded}

\section{Diagnostic Test Performance
Metrics}\label{diagnostic-test-performance-metrics}

From the confusion matrix, we can calculate several important metrics:

\subsection{Sensitivity (True Positive
Rate)}\label{sensitivity-true-positive-rate}

\textbf{What it means}: Of all the people who actually have the disease,
what percentage does our test catch?

\[\text{Sensitivity} = \frac{TP}{TP + FN}\]

\textbf{Example}: If sensitivity = 0.95, then our test catches 95\% of
COVID-positive patients.

\subsection{Specificity (True Negative
Rate)}\label{specificity-true-negative-rate}

\textbf{What it means}: Of all the people who don't have the disease,
what percentage does our test correctly identify as negative?

\[\text{Specificity} = \frac{TN}{TN + FP}\]

\textbf{Example}: If specificity = 0.98, then our test correctly
identifies 98\% of COVID-negative patients.

\subsection{Positive Predictive Value
(PPV)}\label{positive-predictive-value-ppv}

\textbf{What it means}: If the test is positive, what's the probability
the person actually has the disease?

\[\text{PPV} = \frac{TP}{TP + FP}\]

\subsection{Negative Predictive Value
(NPV)}\label{negative-predictive-value-npv}

\textbf{What it means}: If the test is negative, what's the probability
the person actually doesn't have the disease?

\[\text{NPV} = \frac{TN}{TN + FN}\]

\begin{tcolorbox}[enhanced jigsaw, toprule=.15mm, left=2mm, opacitybacktitle=0.6, colframe=quarto-callout-warning-color-frame, leftrule=.75mm, titlerule=0mm, coltitle=black, colbacktitle=quarto-callout-warning-color!10!white, toptitle=1mm, title=\textcolor{quarto-callout-warning-color}{\faExclamationTriangle}\hspace{0.5em}{Important: PPV and NPV depend on disease prevalence!}, bottomtitle=1mm, arc=.35mm, rightrule=.15mm, bottomrule=.15mm, breakable, opacityback=0, colback=white]

In a population where disease is rare, even a good test will have many
false positives, lowering PPV. In a high-prevalence population, the same
test will have better PPV.

\end{tcolorbox}

\begin{Shaded}
\begin{Highlighting}[]
\CommentTok{\# Calculate metrics manually}
\NormalTok{calculate\_metrics }\OtherTok{\textless{}{-}} \ControlFlowTok{function}\NormalTok{(tp, tn, fp, fn) \{}
\NormalTok{  sensitivity }\OtherTok{\textless{}{-}}\NormalTok{ tp }\SpecialCharTok{/}\NormalTok{ (tp }\SpecialCharTok{+}\NormalTok{ fn)}
\NormalTok{  specificity }\OtherTok{\textless{}{-}}\NormalTok{ tn }\SpecialCharTok{/}\NormalTok{ (tn }\SpecialCharTok{+}\NormalTok{ fp)}
\NormalTok{  ppv }\OtherTok{\textless{}{-}}\NormalTok{ tp }\SpecialCharTok{/}\NormalTok{ (tp }\SpecialCharTok{+}\NormalTok{ fp)}
\NormalTok{  npv }\OtherTok{\textless{}{-}}\NormalTok{ tn }\SpecialCharTok{/}\NormalTok{ (tn }\SpecialCharTok{+}\NormalTok{ fn)}
\NormalTok{  accuracy }\OtherTok{\textless{}{-}}\NormalTok{ (tp }\SpecialCharTok{+}\NormalTok{ tn) }\SpecialCharTok{/}\NormalTok{ (tp }\SpecialCharTok{+}\NormalTok{ tn }\SpecialCharTok{+}\NormalTok{ fp }\SpecialCharTok{+}\NormalTok{ fn)}

  \FunctionTok{return}\NormalTok{(}\FunctionTok{list}\NormalTok{(}
    \AttributeTok{Sensitivity =}\NormalTok{ sensitivity,}
    \AttributeTok{Specificity =}\NormalTok{ specificity,}
    \AttributeTok{PPV =}\NormalTok{ ppv,}
    \AttributeTok{NPV =}\NormalTok{ npv,}
    \AttributeTok{Accuracy =}\NormalTok{ accuracy}
\NormalTok{  ))}
\NormalTok{\}}

\CommentTok{\# Example with our COVID data}
\NormalTok{metrics }\OtherTok{\textless{}{-}} \FunctionTok{calculate\_metrics}\NormalTok{(}\AttributeTok{tp =} \DecValTok{95}\NormalTok{, }\AttributeTok{tn =} \DecValTok{882}\NormalTok{, }\AttributeTok{fp =} \DecValTok{18}\NormalTok{, }\AttributeTok{fn =} \DecValTok{5}\NormalTok{)}
\FunctionTok{print}\NormalTok{(metrics)}
\end{Highlighting}
\end{Shaded}

\section{ROC Curves and AUC}\label{roc-curves-and-auc}

\subsection{What is an ROC Curve?}\label{what-is-an-roc-curve}

ROC stands for ``Receiver Operating Characteristic.'' Think of it as a
way to visualize the trade-off between catching true cases (sensitivity)
and avoiding false alarms (1 - specificity).

\textbf{Real-world analogy}: Imagine you're a security guard with a
metal detector. You can adjust the sensitivity: - High sensitivity:
Catches all weapons but sets off many false alarms - Low sensitivity:
Fewer false alarms but might miss some weapons

The ROC curve shows this trade-off at different threshold settings.

\subsection{Mathematical Definition}\label{mathematical-definition}

The ROC curve plots: - \textbf{Y-axis}: True Positive Rate (Sensitivity)
= \(\frac{TP}{TP + FN}\) - \textbf{X-axis}: False Positive Rate (1 -
Specificity) = \(\frac{FP}{FP + TN}\)

\subsection{Creating ROC Curves}\label{creating-roc-curves}

\begin{Shaded}
\begin{Highlighting}[]
\FunctionTok{library}\NormalTok{(pROC)}
\FunctionTok{library}\NormalTok{(ggplot2)}

\CommentTok{\# Simulate biomarker data for disease diagnosis}
\FunctionTok{set.seed}\NormalTok{(}\DecValTok{101}\NormalTok{)}
\NormalTok{n\_healthy }\OtherTok{\textless{}{-}} \DecValTok{500}
\NormalTok{n\_diseased }\OtherTok{\textless{}{-}} \DecValTok{200}

\CommentTok{\# Healthy individuals: lower biomarker levels}
\NormalTok{healthy\_biomarker }\OtherTok{\textless{}{-}} \FunctionTok{rnorm}\NormalTok{(n\_healthy, }\AttributeTok{mean =} \DecValTok{10}\NormalTok{, }\AttributeTok{sd =} \DecValTok{3}\NormalTok{)}

\CommentTok{\# Diseased individuals: higher biomarker levels}
\NormalTok{diseased\_biomarker }\OtherTok{\textless{}{-}} \FunctionTok{rnorm}\NormalTok{(n\_diseased, }\AttributeTok{mean =} \DecValTok{15}\NormalTok{, }\AttributeTok{sd =} \DecValTok{3}\NormalTok{)}

\CommentTok{\# Combine data}
\NormalTok{biomarker\_values }\OtherTok{\textless{}{-}} \FunctionTok{c}\NormalTok{(healthy\_biomarker, diseased\_biomarker)}
\NormalTok{true\_status }\OtherTok{\textless{}{-}} \FunctionTok{c}\NormalTok{(}\FunctionTok{rep}\NormalTok{(}\DecValTok{0}\NormalTok{, n\_healthy), }\FunctionTok{rep}\NormalTok{(}\DecValTok{1}\NormalTok{, n\_diseased))}

\CommentTok{\# Create ROC curve}
\NormalTok{roc\_obj }\OtherTok{\textless{}{-}} \FunctionTok{roc}\NormalTok{(true\_status, biomarker\_values)}

\CommentTok{\# Plot ROC curve}
\FunctionTok{ggroc}\NormalTok{(roc\_obj, }\AttributeTok{color =} \StringTok{"blue"}\NormalTok{, }\AttributeTok{size =} \DecValTok{1}\NormalTok{) }\SpecialCharTok{+}
  \FunctionTok{geom\_abline}\NormalTok{(}\AttributeTok{intercept =} \DecValTok{1}\NormalTok{, }\AttributeTok{slope =} \DecValTok{1}\NormalTok{, }\AttributeTok{linetype =} \StringTok{"dashed"}\NormalTok{, }\AttributeTok{color =} \StringTok{"red"}\NormalTok{) }\SpecialCharTok{+}
  \FunctionTok{labs}\NormalTok{(}\AttributeTok{title =} \StringTok{"ROC Curve for Biomarker Test"}\NormalTok{,}
       \AttributeTok{x =} \StringTok{"False Positive Rate (1 {-} Specificity)"}\NormalTok{,}
       \AttributeTok{y =} \StringTok{"True Positive Rate (Sensitivity)"}\NormalTok{) }\SpecialCharTok{+}
  \FunctionTok{theme\_minimal}\NormalTok{() }\SpecialCharTok{+}
  \FunctionTok{annotate}\NormalTok{(}\StringTok{"text"}\NormalTok{, }\AttributeTok{x =} \FloatTok{0.7}\NormalTok{, }\AttributeTok{y =} \FloatTok{0.3}\NormalTok{,}
           \AttributeTok{label =} \FunctionTok{paste}\NormalTok{(}\StringTok{"AUC ="}\NormalTok{, }\FunctionTok{round}\NormalTok{(}\FunctionTok{auc}\NormalTok{(roc\_obj), }\DecValTok{3}\NormalTok{)), }\AttributeTok{size =} \DecValTok{5}\NormalTok{)}
\end{Highlighting}
\end{Shaded}

\subsection{Understanding AUC (Area Under the
Curve)}\label{understanding-auc-area-under-the-curve}

The AUC summarizes the ROC curve into a single number between 0 and 1:

\[\text{AUC} = \int_0^1 \text{TPR}(t) \, d(\text{FPR}(t))\]

\textbf{Interpretation}: - \textbf{AUC = 0.5}: Random guessing (coin
flip) - \textbf{AUC = 0.7}: Acceptable discrimination - \textbf{AUC =
0.8}: Excellent discrimination - \textbf{AUC = 0.9}: Outstanding
discrimination - \textbf{AUC = 1.0}: Perfect discrimination

\begin{tcolorbox}[enhanced jigsaw, toprule=.15mm, left=2mm, opacitybacktitle=0.6, colframe=quarto-callout-tip-color-frame, leftrule=.75mm, titlerule=0mm, coltitle=black, colbacktitle=quarto-callout-tip-color!10!white, toptitle=1mm, title=\textcolor{quarto-callout-tip-color}{\faLightbulb}\hspace{0.5em}{Intuitive AUC Interpretation}, bottomtitle=1mm, arc=.35mm, rightrule=.15mm, bottomrule=.15mm, breakable, opacityback=0, colback=white]

AUC represents the probability that a randomly selected positive case
will have a higher predicted probability than a randomly selected
negative case.

If AUC = 0.8, then 80\% of the time, a diseased person will have a
higher biomarker value than a healthy person.

\end{tcolorbox}

\subsection{Comparing Multiple Tests}\label{comparing-multiple-tests}

\begin{Shaded}
\begin{Highlighting}[]
\CommentTok{\# Simulate three different biomarkers}
\NormalTok{biomarker1 }\OtherTok{\textless{}{-}} \FunctionTok{c}\NormalTok{(}\FunctionTok{rnorm}\NormalTok{(n\_healthy, }\DecValTok{10}\NormalTok{, }\DecValTok{3}\NormalTok{), }\FunctionTok{rnorm}\NormalTok{(n\_diseased, }\DecValTok{15}\NormalTok{, }\DecValTok{3}\NormalTok{))  }\CommentTok{\# Good test}
\NormalTok{biomarker2 }\OtherTok{\textless{}{-}} \FunctionTok{c}\NormalTok{(}\FunctionTok{rnorm}\NormalTok{(n\_healthy, }\DecValTok{12}\NormalTok{, }\DecValTok{4}\NormalTok{), }\FunctionTok{rnorm}\NormalTok{(n\_diseased, }\DecValTok{14}\NormalTok{, }\DecValTok{4}\NormalTok{))  }\CommentTok{\# Moderate test}
\NormalTok{biomarker3 }\OtherTok{\textless{}{-}} \FunctionTok{c}\NormalTok{(}\FunctionTok{rnorm}\NormalTok{(n\_healthy, }\DecValTok{11}\NormalTok{, }\DecValTok{5}\NormalTok{), }\FunctionTok{rnorm}\NormalTok{(n\_diseased, }\DecValTok{13}\NormalTok{, }\DecValTok{5}\NormalTok{))  }\CommentTok{\# Poor test}

\CommentTok{\# Create ROC curves}
\NormalTok{roc1 }\OtherTok{\textless{}{-}} \FunctionTok{roc}\NormalTok{(true\_status, biomarker1)}
\NormalTok{roc2 }\OtherTok{\textless{}{-}} \FunctionTok{roc}\NormalTok{(true\_status, biomarker2)}
\NormalTok{roc3 }\OtherTok{\textless{}{-}} \FunctionTok{roc}\NormalTok{(true\_status, biomarker3)}

\CommentTok{\# Plot comparison}
\FunctionTok{library}\NormalTok{(patchwork)}
\NormalTok{p1 }\OtherTok{\textless{}{-}} \FunctionTok{ggroc}\NormalTok{(}\FunctionTok{list}\NormalTok{(}\StringTok{"Biomarker 1"} \OtherTok{=}\NormalTok{ roc1, }\StringTok{"Biomarker 2"} \OtherTok{=}\NormalTok{ roc2, }\StringTok{"Biomarker 3"} \OtherTok{=}\NormalTok{ roc3)) }\SpecialCharTok{+}
  \FunctionTok{scale\_color\_manual}\NormalTok{(}\AttributeTok{values =} \FunctionTok{c}\NormalTok{(}\StringTok{"blue"}\NormalTok{, }\StringTok{"green"}\NormalTok{, }\StringTok{"red"}\NormalTok{)) }\SpecialCharTok{+}
  \FunctionTok{geom\_abline}\NormalTok{(}\AttributeTok{intercept =} \DecValTok{1}\NormalTok{, }\AttributeTok{slope =} \DecValTok{1}\NormalTok{, }\AttributeTok{linetype =} \StringTok{"dashed"}\NormalTok{, }\AttributeTok{color =} \StringTok{"black"}\NormalTok{) }\SpecialCharTok{+}
  \FunctionTok{labs}\NormalTok{(}\AttributeTok{title =} \StringTok{"Comparison of Three Biomarkers"}\NormalTok{,}
       \AttributeTok{x =} \StringTok{"False Positive Rate"}\NormalTok{, }\AttributeTok{y =} \StringTok{"True Positive Rate"}\NormalTok{) }\SpecialCharTok{+}
  \FunctionTok{theme\_minimal}\NormalTok{()}

\CommentTok{\# Print AUC values}
\FunctionTok{cat}\NormalTok{(}\StringTok{"AUC Values:}\SpecialCharTok{\textbackslash{}n}\StringTok{"}\NormalTok{)}
\FunctionTok{cat}\NormalTok{(}\StringTok{"Biomarker 1:"}\NormalTok{, }\FunctionTok{round}\NormalTok{(}\FunctionTok{auc}\NormalTok{(roc1), }\DecValTok{3}\NormalTok{), }\StringTok{"}\SpecialCharTok{\textbackslash{}n}\StringTok{"}\NormalTok{)}
\FunctionTok{cat}\NormalTok{(}\StringTok{"Biomarker 2:"}\NormalTok{, }\FunctionTok{round}\NormalTok{(}\FunctionTok{auc}\NormalTok{(roc2), }\DecValTok{3}\NormalTok{), }\StringTok{"}\SpecialCharTok{\textbackslash{}n}\StringTok{"}\NormalTok{)}
\FunctionTok{cat}\NormalTok{(}\StringTok{"Biomarker 3:"}\NormalTok{, }\FunctionTok{round}\NormalTok{(}\FunctionTok{auc}\NormalTok{(roc3), }\DecValTok{3}\NormalTok{), }\StringTok{"}\SpecialCharTok{\textbackslash{}n}\StringTok{"}\NormalTok{)}

\FunctionTok{print}\NormalTok{(p1)}
\end{Highlighting}
\end{Shaded}

\subsection{Optimal Threshold
Selection}\label{optimal-threshold-selection}

\begin{Shaded}
\begin{Highlighting}[]
\CommentTok{\# Find optimal threshold using Youden\textquotesingle{}s J statistic}
\NormalTok{coords\_all }\OtherTok{\textless{}{-}} \FunctionTok{coords}\NormalTok{(roc1, }\StringTok{"all"}\NormalTok{, }\AttributeTok{ret =} \FunctionTok{c}\NormalTok{(}\StringTok{"threshold"}\NormalTok{, }\StringTok{"sensitivity"}\NormalTok{, }\StringTok{"specificity"}\NormalTok{))}
\NormalTok{coords\_all}\SpecialCharTok{$}\NormalTok{youden }\OtherTok{\textless{}{-}}\NormalTok{ coords\_all}\SpecialCharTok{$}\NormalTok{sensitivity }\SpecialCharTok{+}\NormalTok{ coords\_all}\SpecialCharTok{$}\NormalTok{specificity }\SpecialCharTok{{-}} \DecValTok{1}
\NormalTok{optimal\_threshold }\OtherTok{\textless{}{-}}\NormalTok{ coords\_all[}\FunctionTok{which.max}\NormalTok{(coords\_all}\SpecialCharTok{$}\NormalTok{youden), ]}

\FunctionTok{cat}\NormalTok{(}\StringTok{"Optimal Threshold Analysis:}\SpecialCharTok{\textbackslash{}n}\StringTok{"}\NormalTok{)}
\FunctionTok{cat}\NormalTok{(}\StringTok{"Threshold:"}\NormalTok{, }\FunctionTok{round}\NormalTok{(optimal\_threshold}\SpecialCharTok{$}\NormalTok{threshold, }\DecValTok{2}\NormalTok{), }\StringTok{"}\SpecialCharTok{\textbackslash{}n}\StringTok{"}\NormalTok{)}
\FunctionTok{cat}\NormalTok{(}\StringTok{"Sensitivity:"}\NormalTok{, }\FunctionTok{round}\NormalTok{(optimal\_threshold}\SpecialCharTok{$}\NormalTok{sensitivity, }\DecValTok{3}\NormalTok{), }\StringTok{"}\SpecialCharTok{\textbackslash{}n}\StringTok{"}\NormalTok{)}
\FunctionTok{cat}\NormalTok{(}\StringTok{"Specificity:"}\NormalTok{, }\FunctionTok{round}\NormalTok{(optimal\_threshold}\SpecialCharTok{$}\NormalTok{specificity, }\DecValTok{3}\NormalTok{), }\StringTok{"}\SpecialCharTok{\textbackslash{}n}\StringTok{"}\NormalTok{)}
\FunctionTok{cat}\NormalTok{(}\StringTok{"Youden\textquotesingle{}s J:"}\NormalTok{, }\FunctionTok{round}\NormalTok{(optimal\_threshold}\SpecialCharTok{$}\NormalTok{youden, }\DecValTok{3}\NormalTok{), }\StringTok{"}\SpecialCharTok{\textbackslash{}n}\StringTok{"}\NormalTok{)}

\CommentTok{\# Plot threshold selection}
\NormalTok{threshold\_plot }\OtherTok{\textless{}{-}} \FunctionTok{ggplot}\NormalTok{(coords\_all, }\FunctionTok{aes}\NormalTok{(}\AttributeTok{x =}\NormalTok{ threshold)) }\SpecialCharTok{+}
  \FunctionTok{geom\_line}\NormalTok{(}\FunctionTok{aes}\NormalTok{(}\AttributeTok{y =}\NormalTok{ sensitivity, }\AttributeTok{color =} \StringTok{"Sensitivity"}\NormalTok{), }\AttributeTok{size =} \DecValTok{1}\NormalTok{) }\SpecialCharTok{+}
  \FunctionTok{geom\_line}\NormalTok{(}\FunctionTok{aes}\NormalTok{(}\AttributeTok{y =}\NormalTok{ specificity, }\AttributeTok{color =} \StringTok{"Specificity"}\NormalTok{), }\AttributeTok{size =} \DecValTok{1}\NormalTok{) }\SpecialCharTok{+}
  \FunctionTok{geom\_vline}\NormalTok{(}\AttributeTok{xintercept =}\NormalTok{ optimal\_threshold}\SpecialCharTok{$}\NormalTok{threshold, }\AttributeTok{linetype =} \StringTok{"dashed"}\NormalTok{) }\SpecialCharTok{+}
  \FunctionTok{scale\_color\_manual}\NormalTok{(}\AttributeTok{values =} \FunctionTok{c}\NormalTok{(}\StringTok{"blue"}\NormalTok{, }\StringTok{"red"}\NormalTok{)) }\SpecialCharTok{+}
  \FunctionTok{labs}\NormalTok{(}\AttributeTok{title =} \StringTok{"Sensitivity and Specificity vs Threshold"}\NormalTok{,}
       \AttributeTok{x =} \StringTok{"Biomarker Threshold"}\NormalTok{, }\AttributeTok{y =} \StringTok{"Rate"}\NormalTok{, }\AttributeTok{color =} \StringTok{"Metric"}\NormalTok{) }\SpecialCharTok{+}
  \FunctionTok{theme\_minimal}\NormalTok{()}

\FunctionTok{print}\NormalTok{(threshold\_plot)}
\end{Highlighting}
\end{Shaded}

\section{Practical Applications in
Biostatistics}\label{practical-applications-in-biostatistics}

\subsection{1. Biomarker Validation}\label{biomarker-validation}

When developing diagnostic biomarkers: - Use ROC analysis to assess
discriminative ability - Calculate AUC to compare different biomarkers -
Determine optimal cutoff points balancing sensitivity and specificity

\subsection{2. Risk Prediction Models}\label{risk-prediction-models}

In clinical risk assessment: - Logistic regression to model disease
probability - Confusion matrices to evaluate model performance - ROC
curves to assess calibration across risk thresholds

\subsection{3. Treatment Response
Prediction}\label{treatment-response-prediction}

For personalized medicine: - Classification models to predict treatment
response - Performance metrics to validate model utility - Threshold
optimization for clinical decision-making

\begin{tcolorbox}[enhanced jigsaw, toprule=.15mm, left=2mm, opacitybacktitle=0.6, colframe=quarto-callout-note-color-frame, leftrule=.75mm, titlerule=0mm, coltitle=black, colbacktitle=quarto-callout-note-color!10!white, toptitle=1mm, title=\textcolor{quarto-callout-note-color}{\faInfo}\hspace{0.5em}{Key Takeaways}, bottomtitle=1mm, arc=.35mm, rightrule=.15mm, bottomrule=.15mm, breakable, opacityback=0, colback=white]

\begin{enumerate}
\def\labelenumi{\arabic{enumi}.}
\tightlist
\item
  \textbf{Confusion matrices} provide a complete picture of
  classification performance
\item
  \textbf{Sensitivity and specificity} are fundamental test
  characteristics
\item
  \textbf{PPV and NPV} depend on disease prevalence in your population
\item
  \textbf{ROC curves} visualize the sensitivity-specificity trade-off
\item
  \textbf{AUC} provides a single metric for overall discriminative
  ability
\item
  \textbf{Threshold selection} should consider clinical consequences of
  false positives vs false negatives
\end{enumerate}

\end{tcolorbox}

\section{Python Implementation}\label{python-implementation}

For those preferring Python, here's equivalent code:

\begin{Shaded}
\begin{Highlighting}[]
\ImportTok{import}\NormalTok{ numpy }\ImportTok{as}\NormalTok{ np}
\ImportTok{import}\NormalTok{ pandas }\ImportTok{as}\NormalTok{ pd}
\ImportTok{import}\NormalTok{ matplotlib.pyplot }\ImportTok{as}\NormalTok{ plt}
\ImportTok{import}\NormalTok{ seaborn }\ImportTok{as}\NormalTok{ sns}
\ImportTok{from}\NormalTok{ sklearn.metrics }\ImportTok{import}\NormalTok{ confusion\_matrix, roc\_curve, roc\_auc\_score, classification\_report}
\ImportTok{from}\NormalTok{ sklearn.linear\_model }\ImportTok{import}\NormalTok{ LogisticRegression}

\CommentTok{\# Simulate data}
\NormalTok{np.random.seed(}\DecValTok{123}\NormalTok{)}
\NormalTok{n }\OperatorTok{=} \DecValTok{1000}
\NormalTok{age }\OperatorTok{=}\NormalTok{ np.random.normal(}\DecValTok{55}\NormalTok{, }\DecValTok{12}\NormalTok{, n)}
\NormalTok{cholesterol }\OperatorTok{=}\NormalTok{ np.random.normal(}\DecValTok{200}\NormalTok{, }\DecValTok{40}\NormalTok{, n)}
\NormalTok{smoking }\OperatorTok{=}\NormalTok{ np.random.binomial(}\DecValTok{1}\NormalTok{, }\FloatTok{0.3}\NormalTok{, n)}

\CommentTok{\# Create outcome}
\NormalTok{log\_odds }\OperatorTok{=} \OperatorTok{{-}}\DecValTok{5} \OperatorTok{+} \FloatTok{0.05}\OperatorTok{*}\NormalTok{age }\OperatorTok{+} \FloatTok{0.01}\OperatorTok{*}\NormalTok{cholesterol }\OperatorTok{+} \FloatTok{1.2}\OperatorTok{*}\NormalTok{smoking}
\NormalTok{prob\_disease }\OperatorTok{=}\NormalTok{ np.exp(log\_odds)}\OperatorTok{/}\NormalTok{(}\DecValTok{1} \OperatorTok{+}\NormalTok{ np.exp(log\_odds))}
\NormalTok{heart\_disease }\OperatorTok{=}\NormalTok{ np.random.binomial(}\DecValTok{1}\NormalTok{, prob\_disease, n)}

\CommentTok{\# Fit logistic regression}
\NormalTok{X }\OperatorTok{=}\NormalTok{ np.column\_stack([age, cholesterol, smoking])}
\NormalTok{model }\OperatorTok{=}\NormalTok{ LogisticRegression()}
\NormalTok{model.fit(X, heart\_disease)}

\CommentTok{\# Predictions}
\NormalTok{y\_pred\_proba }\OperatorTok{=}\NormalTok{ model.predict\_proba(X)[:, }\DecValTok{1}\NormalTok{]}
\NormalTok{y\_pred }\OperatorTok{=}\NormalTok{ model.predict(X)}

\CommentTok{\# Confusion matrix}
\NormalTok{cm }\OperatorTok{=}\NormalTok{ confusion\_matrix(heart\_disease, y\_pred)}
\NormalTok{plt.figure(figsize}\OperatorTok{=}\NormalTok{(}\DecValTok{8}\NormalTok{, }\DecValTok{6}\NormalTok{))}
\NormalTok{sns.heatmap(cm, annot}\OperatorTok{=}\VariableTok{True}\NormalTok{, fmt}\OperatorTok{=}\StringTok{\textquotesingle{}d\textquotesingle{}}\NormalTok{, cmap}\OperatorTok{=}\StringTok{\textquotesingle{}Blues\textquotesingle{}}\NormalTok{)}
\NormalTok{plt.title(}\StringTok{\textquotesingle{}Confusion Matrix\textquotesingle{}}\NormalTok{)}
\NormalTok{plt.ylabel(}\StringTok{\textquotesingle{}Actual\textquotesingle{}}\NormalTok{)}
\NormalTok{plt.xlabel(}\StringTok{\textquotesingle{}Predicted\textquotesingle{}}\NormalTok{)}
\NormalTok{plt.show()}

\CommentTok{\# ROC curve}
\NormalTok{fpr, tpr, thresholds }\OperatorTok{=}\NormalTok{ roc\_curve(heart\_disease, y\_pred\_proba)}
\NormalTok{auc\_score }\OperatorTok{=}\NormalTok{ roc\_auc\_score(heart\_disease, y\_pred\_proba)}

\NormalTok{plt.figure(figsize}\OperatorTok{=}\NormalTok{(}\DecValTok{8}\NormalTok{, }\DecValTok{6}\NormalTok{))}
\NormalTok{plt.plot(fpr, tpr, color}\OperatorTok{=}\StringTok{\textquotesingle{}blue\textquotesingle{}}\NormalTok{, label}\OperatorTok{=}\SpecialStringTok{f\textquotesingle{}ROC Curve (AUC = }\SpecialCharTok{\{}\NormalTok{auc\_score}\SpecialCharTok{:.3f\}}\SpecialStringTok{)\textquotesingle{}}\NormalTok{)}
\NormalTok{plt.plot([}\DecValTok{0}\NormalTok{, }\DecValTok{1}\NormalTok{], [}\DecValTok{0}\NormalTok{, }\DecValTok{1}\NormalTok{], color}\OperatorTok{=}\StringTok{\textquotesingle{}red\textquotesingle{}}\NormalTok{, linestyle}\OperatorTok{=}\StringTok{\textquotesingle{}{-}{-}\textquotesingle{}}\NormalTok{, label}\OperatorTok{=}\StringTok{\textquotesingle{}Random\textquotesingle{}}\NormalTok{)}
\NormalTok{plt.xlabel(}\StringTok{\textquotesingle{}False Positive Rate\textquotesingle{}}\NormalTok{)}
\NormalTok{plt.ylabel(}\StringTok{\textquotesingle{}True Positive Rate\textquotesingle{}}\NormalTok{)}
\NormalTok{plt.title(}\StringTok{\textquotesingle{}ROC Curve\textquotesingle{}}\NormalTok{)}
\NormalTok{plt.legend()}
\NormalTok{plt.show()}

\CommentTok{\# Classification report}
\BuiltInTok{print}\NormalTok{(classification\_report(heart\_disease, y\_pred))}
\end{Highlighting}
\end{Shaded}

\bookmarksetup{startatroot}

\chapter{Design \& Power
Considerations}\label{design-power-considerations}

Plan studies effectively by determining appropriate sample sizes and
control error rates when testing multiple hypotheses.

\section{Introduction}\label{introduction-1}

In biostatistics, proper study design and statistical power
considerations are crucial for obtaining reliable and interpretable
results. This chapter focuses on key concepts including sample size
determination, statistical power, and critically important methods for
controlling error rates when testing multiple hypotheses---a common
scenario in modern biological research.

\section{Statistical Power and Sample
Size}\label{statistical-power-and-sample-size}

\subsection{Understanding Statistical
Power}\label{understanding-statistical-power}

Statistical power is the probability of correctly rejecting a false null
hypothesis (avoiding Type II error). In simple terms, it's the
probability that your study will detect an effect if there really is
one.

\[\text{Power} = 1 - \beta = P(\text{reject } H_0 | H_0 \text{ is false})\]

Where \(\beta\) is the probability of Type II error.

\subsection{The Four Connected
Elements}\label{the-four-connected-elements}

For any statistical test, four elements are mathematically related: 1.
\textbf{Effect size} (how big the difference/relationship is) 2.
\textbf{Sample size} (how many observations) 3. \textbf{Significance
level} (\(\alpha\), usually 0.05) 4. \textbf{Statistical power} (usually
aim for 0.80 or 80\%)

\begin{tcolorbox}[enhanced jigsaw, toprule=.15mm, left=2mm, opacitybacktitle=0.6, colframe=quarto-callout-tip-color-frame, leftrule=.75mm, titlerule=0mm, coltitle=black, colbacktitle=quarto-callout-tip-color!10!white, toptitle=1mm, title=\textcolor{quarto-callout-tip-color}{\faLightbulb}\hspace{0.5em}{Power Analysis Rule}, bottomtitle=1mm, arc=.35mm, rightrule=.15mm, bottomrule=.15mm, breakable, opacityback=0, colback=white]

If you know any three of these elements, you can calculate the fourth.
This is the foundation of power analysis and sample size planning.

\end{tcolorbox}

\subsection{Sample Size Calculation
Example}\label{sample-size-calculation-example}

\begin{Shaded}
\begin{Highlighting}[]
\CommentTok{\# Sample size for comparing two means}
\FunctionTok{library}\NormalTok{(pwr)}

\CommentTok{\# Scenario: Compare blood pressure between two treatments}
\CommentTok{\# We expect a difference of 5 mmHg (effect size)}
\CommentTok{\# Pooled standard deviation is estimated at 12 mmHg}
\NormalTok{effect\_size }\OtherTok{\textless{}{-}} \DecValTok{5} \SpecialCharTok{/} \DecValTok{12}  \CommentTok{\# Cohen\textquotesingle{}s d}

\CommentTok{\# Calculate sample size needed}
\NormalTok{power\_result }\OtherTok{\textless{}{-}} \FunctionTok{pwr.t.test}\NormalTok{(}
  \AttributeTok{d =}\NormalTok{ effect\_size,           }\CommentTok{\# Effect size (Cohen\textquotesingle{}s d)}
  \AttributeTok{sig.level =} \FloatTok{0.05}\NormalTok{,         }\CommentTok{\# Alpha level}
  \AttributeTok{power =} \FloatTok{0.80}\NormalTok{,             }\CommentTok{\# Desired power}
  \AttributeTok{type =} \StringTok{"two.sample"}       \CommentTok{\# Two{-}sample t{-}test}
\NormalTok{)}

\FunctionTok{print}\NormalTok{(power\_result)}
\FunctionTok{cat}\NormalTok{(}\StringTok{"Required sample size per group:"}\NormalTok{, }\FunctionTok{ceiling}\NormalTok{(power\_result}\SpecialCharTok{$}\NormalTok{n))}
\end{Highlighting}
\end{Shaded}

\section{Multiple Testing Problem}\label{multiple-testing-problem}

\subsection{The Challenge}\label{the-challenge}

When conducting multiple statistical tests, the probability of making at
least one Type I error (false positive) increases dramatically. This is
called the \textbf{multiple testing problem} or \textbf{multiple
comparisons problem}.

\textbf{Simple example}: - Single test with α = 0.05: 5\% chance of
false positive - 10 independent tests: Probability of at least one false
positive = \(1 - (0.95)^{10} = 0.401\) (40.1\%) - 100 tests: Probability
= \(1 - (0.95)^{100} = 0.994\) (99.4\%)

\begin{tcolorbox}[enhanced jigsaw, toprule=.15mm, left=2mm, opacitybacktitle=0.6, colframe=quarto-callout-warning-color-frame, leftrule=.75mm, titlerule=0mm, coltitle=black, colbacktitle=quarto-callout-warning-color!10!white, toptitle=1mm, title=\textcolor{quarto-callout-warning-color}{\faExclamationTriangle}\hspace{0.5em}{The Multiple Testing Crisis}, bottomtitle=1mm, arc=.35mm, rightrule=.15mm, bottomrule=.15mm, breakable, opacityback=0, colback=white]

In genomics, researchers often test thousands or millions of hypotheses
simultaneously (e.g., testing each gene for differential expression).
Without proper correction, almost every study would report significant
results, even if no true effects exist!

\end{tcolorbox}

\subsection{Family-Wise Error Rate
(FWER)}\label{family-wise-error-rate-fwer}

The \textbf{Family-Wise Error Rate} is the probability of making one or
more Type I errors in a family of tests:

\[\text{FWER} = P(\text{at least one Type I error})\]

For \(m\) independent tests each with significance level \(\alpha\):

\[\text{FWER} = 1 - (1-\alpha)^m\]

\subsection{Bonferroni Correction}\label{bonferroni-correction}

The \textbf{Bonferroni correction} is the simplest method to control
FWER. Instead of using \(\alpha\) for each test, use \(\alpha/m\):

\[\alpha_{\text{adjusted}} = \frac{\alpha}{m}\]

\textbf{Example}: Testing 20 hypotheses with overall α = 0.05 -
Bonferroni-adjusted α = 0.05/20 = 0.0025 - Reject \(H_0\) only if
p-value ≤ 0.0025

\begin{Shaded}
\begin{Highlighting}[]
\CommentTok{\# Example: Multiple t{-}tests with Bonferroni correction}
\FunctionTok{set.seed}\NormalTok{(}\DecValTok{123}\NormalTok{)}

\CommentTok{\# Simulate data for 20 different comparisons}
\NormalTok{n\_tests }\OtherTok{\textless{}{-}} \DecValTok{20}
\NormalTok{p\_values }\OtherTok{\textless{}{-}} \FunctionTok{numeric}\NormalTok{(n\_tests)}

\ControlFlowTok{for}\NormalTok{(i }\ControlFlowTok{in} \DecValTok{1}\SpecialCharTok{:}\NormalTok{n\_tests) \{}
  \CommentTok{\# Generate random data (no real differences)}
\NormalTok{  group1 }\OtherTok{\textless{}{-}} \FunctionTok{rnorm}\NormalTok{(}\DecValTok{30}\NormalTok{, }\AttributeTok{mean =} \DecValTok{0}\NormalTok{, }\AttributeTok{sd =} \DecValTok{1}\NormalTok{)}
\NormalTok{  group2 }\OtherTok{\textless{}{-}} \FunctionTok{rnorm}\NormalTok{(}\DecValTok{30}\NormalTok{, }\AttributeTok{mean =} \DecValTok{0}\NormalTok{, }\AttributeTok{sd =} \DecValTok{1}\NormalTok{)}

  \CommentTok{\# Perform t{-}test}
\NormalTok{  test\_result }\OtherTok{\textless{}{-}} \FunctionTok{t.test}\NormalTok{(group1, group2)}
\NormalTok{  p\_values[i] }\OtherTok{\textless{}{-}}\NormalTok{ test\_result}\SpecialCharTok{$}\NormalTok{p.value}
\NormalTok{\}}

\CommentTok{\# Apply Bonferroni correction}
\NormalTok{alpha }\OtherTok{\textless{}{-}} \FloatTok{0.05}
\NormalTok{bonferroni\_alpha }\OtherTok{\textless{}{-}}\NormalTok{ alpha }\SpecialCharTok{/}\NormalTok{ n\_tests}
\NormalTok{bonferroni\_adjusted }\OtherTok{\textless{}{-}} \FunctionTok{p.adjust}\NormalTok{(p\_values, }\AttributeTok{method =} \StringTok{"bonferroni"}\NormalTok{)}

\CommentTok{\# Results}
\NormalTok{results\_df }\OtherTok{\textless{}{-}} \FunctionTok{data.frame}\NormalTok{(}
  \AttributeTok{Test =} \DecValTok{1}\SpecialCharTok{:}\NormalTok{n\_tests,}
  \AttributeTok{Raw\_P\_Value =} \FunctionTok{round}\NormalTok{(p\_values, }\DecValTok{4}\NormalTok{),}
  \AttributeTok{Bonferroni\_Adjusted =} \FunctionTok{round}\NormalTok{(bonferroni\_adjusted, }\DecValTok{4}\NormalTok{),}
  \AttributeTok{Raw\_Significant =}\NormalTok{ p\_values }\SpecialCharTok{\textless{}}\NormalTok{ alpha,}
  \AttributeTok{Bonferroni\_Significant =}\NormalTok{ bonferroni\_adjusted }\SpecialCharTok{\textless{}}\NormalTok{ alpha}
\NormalTok{)}

\FunctionTok{print}\NormalTok{(results\_df)}
\FunctionTok{cat}\NormalTok{(}\StringTok{"Raw significant tests:"}\NormalTok{, }\FunctionTok{sum}\NormalTok{(results\_df}\SpecialCharTok{$}\NormalTok{Raw\_Significant), }\StringTok{"}\SpecialCharTok{\textbackslash{}n}\StringTok{"}\NormalTok{)}
\FunctionTok{cat}\NormalTok{(}\StringTok{"Bonferroni significant tests:"}\NormalTok{, }\FunctionTok{sum}\NormalTok{(results\_df}\SpecialCharTok{$}\NormalTok{Bonferroni\_Significant), }\StringTok{"}\SpecialCharTok{\textbackslash{}n}\StringTok{"}\NormalTok{)}
\end{Highlighting}
\end{Shaded}

\section{False Discovery Rate (FDR)}\label{false-discovery-rate-fdr}

\subsection{What is FDR?}\label{what-is-fdr}

While FWER controls the probability of any false positives, the
\textbf{False Discovery Rate (FDR)} controls the expected proportion of
false positives among the rejected hypotheses.

\textbf{Simple explanation}: If you call 100 tests ``significant'' and
set FDR = 0.05, you expect about 5 of those 100 to be false positives.

\[\text{FDR} = E\left[\frac{\text{Number of false positives}}{\text{Number of rejected hypotheses}}\right]\]

\subsection{Why FDR is Often Better Than
FWER}\label{why-fdr-is-often-better-than-fwer}

\begin{enumerate}
\def\labelenumi{\arabic{enumi}.}
\tightlist
\item
  \textbf{Less conservative}: Allows more discoveries while controlling
  false discovery proportion
\item
  \textbf{More appropriate for exploratory research}: Especially in
  genomics and high-throughput studies
\item
  \textbf{Adaptive to data}: Better performance when many true effects
  exist
\end{enumerate}

\subsection{Benjamini-Hochberg (BH)
Procedure}\label{benjamini-hochberg-bh-procedure}

The most popular FDR control method:

\textbf{Step-by-step algorithm}:

\begin{enumerate}
\def\labelenumi{\arabic{enumi}.}
\tightlist
\item
  Order p-values: \(p_{(1)} \leq p_{(2)} \leq ... \leq p_{(m)}\)
\item
  For \(i = m, m-1, ..., 1\), find the largest \(i\) such that:
  \[p_{(i)} \leq \frac{i}{m} \times \alpha\]
\item
  Reject all \(H_{(j)}\) for \(j = 1, 2, ..., i\)
\end{enumerate}

\textbf{Mathematical foundation}: The BH procedure controls FDR at level
\(\alpha\) when tests are independent or have certain dependence
structures.

\subsection{FDR Example: Gene Expression
Analysis}\label{fdr-example-gene-expression-analysis}

\begin{Shaded}
\begin{Highlighting}[]
\CommentTok{\# Simulate gene expression data}
\FunctionTok{set.seed}\NormalTok{(}\DecValTok{456}\NormalTok{)}
\NormalTok{n\_genes }\OtherTok{\textless{}{-}} \DecValTok{1000}
\NormalTok{n\_samples\_per\_group }\OtherTok{\textless{}{-}} \DecValTok{20}

\CommentTok{\# 950 genes with no difference (null hypotheses)}
\CommentTok{\# 50 genes with real differences (alternative hypotheses)}
\NormalTok{true\_effects }\OtherTok{\textless{}{-}} \FunctionTok{c}\NormalTok{(}\FunctionTok{rep}\NormalTok{(}\DecValTok{0}\NormalTok{, }\DecValTok{950}\NormalTok{), }\FunctionTok{rep}\NormalTok{(}\DecValTok{2}\NormalTok{, }\DecValTok{50}\NormalTok{))  }\CommentTok{\# Effect sizes}

\NormalTok{p\_values }\OtherTok{\textless{}{-}} \FunctionTok{numeric}\NormalTok{(n\_genes)}

\ControlFlowTok{for}\NormalTok{(i }\ControlFlowTok{in} \DecValTok{1}\SpecialCharTok{:}\NormalTok{n\_genes) \{}
  \CommentTok{\# Control group}
\NormalTok{  control }\OtherTok{\textless{}{-}} \FunctionTok{rnorm}\NormalTok{(n\_samples\_per\_group, }\AttributeTok{mean =} \DecValTok{0}\NormalTok{, }\AttributeTok{sd =} \DecValTok{1}\NormalTok{)}

  \CommentTok{\# Treatment group (with potential effect)}
\NormalTok{  treatment }\OtherTok{\textless{}{-}} \FunctionTok{rnorm}\NormalTok{(n\_samples\_per\_group, }\AttributeTok{mean =}\NormalTok{ true\_effects[i], }\AttributeTok{sd =} \DecValTok{1}\NormalTok{)}

  \CommentTok{\# Perform t{-}test}
\NormalTok{  test\_result }\OtherTok{\textless{}{-}} \FunctionTok{t.test}\NormalTok{(control, treatment)}
\NormalTok{  p\_values[i] }\OtherTok{\textless{}{-}}\NormalTok{ test\_result}\SpecialCharTok{$}\NormalTok{p.value}
\NormalTok{\}}

\CommentTok{\# Apply different correction methods}
\NormalTok{alpha }\OtherTok{\textless{}{-}} \FloatTok{0.05}
\NormalTok{raw\_significant }\OtherTok{\textless{}{-}}\NormalTok{ p\_values }\SpecialCharTok{\textless{}}\NormalTok{ alpha}
\NormalTok{bonferroni\_significant }\OtherTok{\textless{}{-}} \FunctionTok{p.adjust}\NormalTok{(p\_values, }\AttributeTok{method =} \StringTok{"bonferroni"}\NormalTok{) }\SpecialCharTok{\textless{}}\NormalTok{ alpha}
\NormalTok{fdr\_significant }\OtherTok{\textless{}{-}} \FunctionTok{p.adjust}\NormalTok{(p\_values, }\AttributeTok{method =} \StringTok{"BH"}\NormalTok{) }\SpecialCharTok{\textless{}}\NormalTok{ alpha}

\CommentTok{\# Calculate performance metrics}
\NormalTok{true\_positives\_raw }\OtherTok{\textless{}{-}} \FunctionTok{sum}\NormalTok{(raw\_significant }\SpecialCharTok{\&}\NormalTok{ true\_effects }\SpecialCharTok{\textgreater{}} \DecValTok{0}\NormalTok{)}
\NormalTok{false\_positives\_raw }\OtherTok{\textless{}{-}} \FunctionTok{sum}\NormalTok{(raw\_significant }\SpecialCharTok{\&}\NormalTok{ true\_effects }\SpecialCharTok{==} \DecValTok{0}\NormalTok{)}

\NormalTok{true\_positives\_bonf }\OtherTok{\textless{}{-}} \FunctionTok{sum}\NormalTok{(bonferroni\_significant }\SpecialCharTok{\&}\NormalTok{ true\_effects }\SpecialCharTok{\textgreater{}} \DecValTok{0}\NormalTok{)}
\NormalTok{false\_positives\_bonf }\OtherTok{\textless{}{-}} \FunctionTok{sum}\NormalTok{(bonferroni\_significant }\SpecialCharTok{\&}\NormalTok{ true\_effects }\SpecialCharTok{==} \DecValTok{0}\NormalTok{)}

\NormalTok{true\_positives\_fdr }\OtherTok{\textless{}{-}} \FunctionTok{sum}\NormalTok{(fdr\_significant }\SpecialCharTok{\&}\NormalTok{ true\_effects }\SpecialCharTok{\textgreater{}} \DecValTok{0}\NormalTok{)}
\NormalTok{false\_positives\_fdr }\OtherTok{\textless{}{-}} \FunctionTok{sum}\NormalTok{(fdr\_significant }\SpecialCharTok{\&}\NormalTok{ true\_effects }\SpecialCharTok{==} \DecValTok{0}\NormalTok{)}

\CommentTok{\# Results summary}
\FunctionTok{cat}\NormalTok{(}\StringTok{"Results Summary:}\SpecialCharTok{\textbackslash{}n}\StringTok{"}\NormalTok{)}
\FunctionTok{cat}\NormalTok{(}\StringTok{"================}\SpecialCharTok{\textbackslash{}n}\StringTok{"}\NormalTok{)}
\FunctionTok{cat}\NormalTok{(}\StringTok{"Raw p{-}values (α = 0.05):}\SpecialCharTok{\textbackslash{}n}\StringTok{"}\NormalTok{)}
\FunctionTok{cat}\NormalTok{(}\StringTok{"  Discoveries:"}\NormalTok{, }\FunctionTok{sum}\NormalTok{(raw\_significant), }\StringTok{"}\SpecialCharTok{\textbackslash{}n}\StringTok{"}\NormalTok{)}
\FunctionTok{cat}\NormalTok{(}\StringTok{"  True positives:"}\NormalTok{, true\_positives\_raw, }\StringTok{"}\SpecialCharTok{\textbackslash{}n}\StringTok{"}\NormalTok{)}
\FunctionTok{cat}\NormalTok{(}\StringTok{"  False positives:"}\NormalTok{, false\_positives\_raw, }\StringTok{"}\SpecialCharTok{\textbackslash{}n}\StringTok{"}\NormalTok{)}
\FunctionTok{cat}\NormalTok{(}\StringTok{"  FDR:"}\NormalTok{, }\FunctionTok{round}\NormalTok{(false\_positives\_raw }\SpecialCharTok{/} \FunctionTok{max}\NormalTok{(}\DecValTok{1}\NormalTok{, }\FunctionTok{sum}\NormalTok{(raw\_significant)), }\DecValTok{3}\NormalTok{), }\StringTok{"}\SpecialCharTok{\textbackslash{}n\textbackslash{}n}\StringTok{"}\NormalTok{)}

\FunctionTok{cat}\NormalTok{(}\StringTok{"Bonferroni correction:}\SpecialCharTok{\textbackslash{}n}\StringTok{"}\NormalTok{)}
\FunctionTok{cat}\NormalTok{(}\StringTok{"  Discoveries:"}\NormalTok{, }\FunctionTok{sum}\NormalTok{(bonferroni\_significant), }\StringTok{"}\SpecialCharTok{\textbackslash{}n}\StringTok{"}\NormalTok{)}
\FunctionTok{cat}\NormalTok{(}\StringTok{"  True positives:"}\NormalTok{, true\_positives\_bonf, }\StringTok{"}\SpecialCharTok{\textbackslash{}n}\StringTok{"}\NormalTok{)}
\FunctionTok{cat}\NormalTok{(}\StringTok{"  False positives:"}\NormalTok{, false\_positives\_bonf, }\StringTok{"}\SpecialCharTok{\textbackslash{}n}\StringTok{"}\NormalTok{)}
\FunctionTok{cat}\NormalTok{(}\StringTok{"  FDR:"}\NormalTok{, }\FunctionTok{round}\NormalTok{(false\_positives\_bonf }\SpecialCharTok{/} \FunctionTok{max}\NormalTok{(}\DecValTok{1}\NormalTok{, }\FunctionTok{sum}\NormalTok{(bonferroni\_significant)), }\DecValTok{3}\NormalTok{), }\StringTok{"}\SpecialCharTok{\textbackslash{}n\textbackslash{}n}\StringTok{"}\NormalTok{)}

\FunctionTok{cat}\NormalTok{(}\StringTok{"FDR (Benjamini{-}Hochberg):}\SpecialCharTok{\textbackslash{}n}\StringTok{"}\NormalTok{)}
\FunctionTok{cat}\NormalTok{(}\StringTok{"  Discoveries:"}\NormalTok{, }\FunctionTok{sum}\NormalTok{(fdr\_significant), }\StringTok{"}\SpecialCharTok{\textbackslash{}n}\StringTok{"}\NormalTok{)}
\FunctionTok{cat}\NormalTok{(}\StringTok{"  True positives:"}\NormalTok{, true\_positives\_fdr, }\StringTok{"}\SpecialCharTok{\textbackslash{}n}\StringTok{"}\NormalTok{)}
\FunctionTok{cat}\NormalTok{(}\StringTok{"  False positives:"}\NormalTok{, false\_positives\_fdr, }\StringTok{"}\SpecialCharTok{\textbackslash{}n}\StringTok{"}\NormalTok{)}
\FunctionTok{cat}\NormalTok{(}\StringTok{"  FDR:"}\NormalTok{, }\FunctionTok{round}\NormalTok{(false\_positives\_fdr }\SpecialCharTok{/} \FunctionTok{max}\NormalTok{(}\DecValTok{1}\NormalTok{, }\FunctionTok{sum}\NormalTok{(fdr\_significant)), }\DecValTok{3}\NormalTok{), }\StringTok{"}\SpecialCharTok{\textbackslash{}n}\StringTok{"}\NormalTok{)}
\end{Highlighting}
\end{Shaded}

\subsection{Visualizing FDR vs FWER}\label{visualizing-fdr-vs-fwer}

\begin{Shaded}
\begin{Highlighting}[]
\FunctionTok{library}\NormalTok{(ggplot2)}
\FunctionTok{library}\NormalTok{(dplyr)}

\CommentTok{\# Create comparison data}
\NormalTok{comparison\_data }\OtherTok{\textless{}{-}} \FunctionTok{data.frame}\NormalTok{(}
  \AttributeTok{Method =} \FunctionTok{c}\NormalTok{(}\StringTok{"Raw p{-}values"}\NormalTok{, }\StringTok{"Bonferroni"}\NormalTok{, }\StringTok{"FDR (BH)"}\NormalTok{),}
  \AttributeTok{Discoveries =} \FunctionTok{c}\NormalTok{(}\FunctionTok{sum}\NormalTok{(raw\_significant), }\FunctionTok{sum}\NormalTok{(bonferroni\_significant), }\FunctionTok{sum}\NormalTok{(fdr\_significant)),}
  \AttributeTok{True\_Positives =} \FunctionTok{c}\NormalTok{(true\_positives\_raw, true\_positives\_bonf, true\_positives\_fdr),}
  \AttributeTok{False\_Positives =} \FunctionTok{c}\NormalTok{(false\_positives\_raw, false\_positives\_bonf, false\_positives\_fdr)}
\NormalTok{)}

\NormalTok{comparison\_data}\SpecialCharTok{$}\NormalTok{Power }\OtherTok{\textless{}{-}}\NormalTok{ comparison\_data}\SpecialCharTok{$}\NormalTok{True\_Positives }\SpecialCharTok{/} \DecValTok{50}  \CommentTok{\# 50 true effects}
\NormalTok{comparison\_data}\SpecialCharTok{$}\NormalTok{FDR }\OtherTok{\textless{}{-}}\NormalTok{ comparison\_data}\SpecialCharTok{$}\NormalTok{False\_Positives }\SpecialCharTok{/} \FunctionTok{pmax}\NormalTok{(}\DecValTok{1}\NormalTok{, comparison\_data}\SpecialCharTok{$}\NormalTok{Discoveries)}

\CommentTok{\# Plot results}
\NormalTok{p1 }\OtherTok{\textless{}{-}} \FunctionTok{ggplot}\NormalTok{(comparison\_data, }\FunctionTok{aes}\NormalTok{(}\AttributeTok{x =}\NormalTok{ Method, }\AttributeTok{y =}\NormalTok{ Discoveries, }\AttributeTok{fill =}\NormalTok{ Method)) }\SpecialCharTok{+}
  \FunctionTok{geom\_col}\NormalTok{() }\SpecialCharTok{+}
  \FunctionTok{geom\_text}\NormalTok{(}\FunctionTok{aes}\NormalTok{(}\AttributeTok{label =}\NormalTok{ Discoveries), }\AttributeTok{vjust =} \SpecialCharTok{{-}}\FloatTok{0.5}\NormalTok{) }\SpecialCharTok{+}
  \FunctionTok{labs}\NormalTok{(}\AttributeTok{title =} \StringTok{"Number of Discoveries by Method"}\NormalTok{, }\AttributeTok{y =} \StringTok{"Discoveries"}\NormalTok{) }\SpecialCharTok{+}
  \FunctionTok{theme\_minimal}\NormalTok{() }\SpecialCharTok{+}
  \FunctionTok{theme}\NormalTok{(}\AttributeTok{legend.position =} \StringTok{"none"}\NormalTok{)}

\NormalTok{p2 }\OtherTok{\textless{}{-}} \FunctionTok{ggplot}\NormalTok{(comparison\_data, }\FunctionTok{aes}\NormalTok{(}\AttributeTok{x =}\NormalTok{ Method, }\AttributeTok{y =}\NormalTok{ Power, }\AttributeTok{fill =}\NormalTok{ Method)) }\SpecialCharTok{+}
  \FunctionTok{geom\_col}\NormalTok{() }\SpecialCharTok{+}
  \FunctionTok{geom\_text}\NormalTok{(}\FunctionTok{aes}\NormalTok{(}\AttributeTok{label =} \FunctionTok{round}\NormalTok{(Power, }\DecValTok{2}\NormalTok{)), }\AttributeTok{vjust =} \SpecialCharTok{{-}}\FloatTok{0.5}\NormalTok{) }\SpecialCharTok{+}
  \FunctionTok{labs}\NormalTok{(}\AttributeTok{title =} \StringTok{"Statistical Power by Method"}\NormalTok{, }\AttributeTok{y =} \StringTok{"Power (True Positive Rate)"}\NormalTok{) }\SpecialCharTok{+}
  \FunctionTok{ylim}\NormalTok{(}\DecValTok{0}\NormalTok{, }\DecValTok{1}\NormalTok{) }\SpecialCharTok{+}
  \FunctionTok{theme\_minimal}\NormalTok{() }\SpecialCharTok{+}
  \FunctionTok{theme}\NormalTok{(}\AttributeTok{legend.position =} \StringTok{"none"}\NormalTok{)}

\FunctionTok{library}\NormalTok{(patchwork)}
\FunctionTok{print}\NormalTok{(p1 }\SpecialCharTok{/}\NormalTok{ p2)}
\end{Highlighting}
\end{Shaded}

\section{q-values: A Modern Approach to
FDR}\label{q-values-a-modern-approach-to-fdr}

\subsection{What are q-values?}\label{what-are-q-values}

A \textbf{q-value} is the minimum FDR at which a hypothesis would be
rejected. It's analogous to how a p-value is the minimum significance
level at which a hypothesis would be rejected.

\textbf{Interpretation}: A q-value of 0.05 means that 5\% of tests with
q-values ≤ 0.05 are expected to be false positives.

\subsection{Calculating q-values}\label{calculating-q-values}

\begin{Shaded}
\begin{Highlighting}[]
\CommentTok{\# Calculate q{-}values using the qvalue package}
\CommentTok{\# Note: You may need to install from Bioconductor}
\CommentTok{\# if (!requireNamespace("BiocManager", quietly = TRUE))}
\CommentTok{\#     install.packages("BiocManager")}
\CommentTok{\# BiocManager::install("qvalue")}

\FunctionTok{library}\NormalTok{(qvalue)}

\CommentTok{\# Calculate q{-}values from our p{-}values}
\NormalTok{q\_values }\OtherTok{\textless{}{-}} \FunctionTok{qvalue}\NormalTok{(p\_values)}

\CommentTok{\# Plot q{-}value object}
\FunctionTok{plot}\NormalTok{(q\_values)}

\CommentTok{\# Summary of q{-}values}
\FunctionTok{summary}\NormalTok{(q\_values)}

\CommentTok{\# Number of significant findings at different FDR levels}
\FunctionTok{cat}\NormalTok{(}\StringTok{"Significant findings:}\SpecialCharTok{\textbackslash{}n}\StringTok{"}\NormalTok{)}
\FunctionTok{cat}\NormalTok{(}\StringTok{"FDR \textless{} 0.01:"}\NormalTok{, }\FunctionTok{sum}\NormalTok{(q\_values}\SpecialCharTok{$}\NormalTok{qvalues }\SpecialCharTok{\textless{}} \FloatTok{0.01}\NormalTok{), }\StringTok{"}\SpecialCharTok{\textbackslash{}n}\StringTok{"}\NormalTok{)}
\FunctionTok{cat}\NormalTok{(}\StringTok{"FDR \textless{} 0.05:"}\NormalTok{, }\FunctionTok{sum}\NormalTok{(q\_values}\SpecialCharTok{$}\NormalTok{qvalues }\SpecialCharTok{\textless{}} \FloatTok{0.05}\NormalTok{), }\StringTok{"}\SpecialCharTok{\textbackslash{}n}\StringTok{"}\NormalTok{)}
\FunctionTok{cat}\NormalTok{(}\StringTok{"FDR \textless{} 0.10:"}\NormalTok{, }\FunctionTok{sum}\NormalTok{(q\_values}\SpecialCharTok{$}\NormalTok{qvalues }\SpecialCharTok{\textless{}} \FloatTok{0.10}\NormalTok{), }\StringTok{"}\SpecialCharTok{\textbackslash{}n}\StringTok{"}\NormalTok{)}
\end{Highlighting}
\end{Shaded}

\section{Practical Guidelines for Multiple
Testing}\label{practical-guidelines-for-multiple-testing}

\subsection{When to Use Which Method}\label{when-to-use-which-method}

\textbf{Use Bonferroni (FWER control) when}: - Small number of tests
(\textless{} 20) - Each test is critically important - Any false
positive would be seriously problematic - Confirmatory studies

\textbf{Use FDR control when}: - Large number of tests (hundreds to
millions) - Exploratory research - Discovery-oriented studies -
Follow-up validation will be performed

\subsection{Best Practices}\label{best-practices}

\begin{enumerate}
\def\labelenumi{\arabic{enumi}.}
\tightlist
\item
  \textbf{Plan your analysis}: Decide on multiple testing correction
  before seeing the data
\item
  \textbf{Consider the biological context}: Balance statistical rigor
  with biological interpretation
\item
  \textbf{Report both raw and adjusted p-values}: Transparency in
  reporting
\item
  \textbf{Use appropriate FDR level}: Common choices are 0.05, 0.10, or
  0.20 depending on context
\end{enumerate}

\begin{tcolorbox}[enhanced jigsaw, toprule=.15mm, left=2mm, opacitybacktitle=0.6, colframe=quarto-callout-note-color-frame, leftrule=.75mm, titlerule=0mm, coltitle=black, colbacktitle=quarto-callout-note-color!10!white, toptitle=1mm, title=\textcolor{quarto-callout-note-color}{\faInfo}\hspace{0.5em}{Key Concepts Summary}, bottomtitle=1mm, arc=.35mm, rightrule=.15mm, bottomrule=.15mm, breakable, opacityback=0, colback=white]

\begin{enumerate}
\def\labelenumi{\arabic{enumi}.}
\tightlist
\item
  \textbf{Multiple testing increases false positive rates} exponentially
\item
  \textbf{FWER control} (Bonferroni): Controls probability of any false
  positive
\item
  \textbf{FDR control} (Benjamini-Hochberg): Controls proportion of
  false positives among discoveries
\item
  \textbf{q-values} provide interpretable measure of false discovery
  rate
\item
  \textbf{Method choice} depends on study goals, number of tests, and
  consequences of false positives
\end{enumerate}

\end{tcolorbox}

\section{Real-World Example: COVID-19 Biomarker
Discovery}\label{real-world-example-covid-19-biomarker-discovery}

Let's apply these concepts to a realistic scenario:

\begin{Shaded}
\begin{Highlighting}[]
\CommentTok{\# Simulate COVID{-}19 biomarker study}
\FunctionTok{set.seed}\NormalTok{(}\DecValTok{789}\NormalTok{)}

\CommentTok{\# Study parameters}
\NormalTok{n\_biomarkers }\OtherTok{\textless{}{-}} \DecValTok{500}  \CommentTok{\# Testing 500 potential biomarkers}
\NormalTok{n\_covid }\OtherTok{\textless{}{-}} \DecValTok{100}       \CommentTok{\# 100 COVID{-}19 patients}
\NormalTok{n\_control }\OtherTok{\textless{}{-}} \DecValTok{100}     \CommentTok{\# 100 healthy controls}

\CommentTok{\# Simulate biomarker data}
\CommentTok{\# 475 biomarkers: no real difference}
\CommentTok{\# 25 biomarkers: real differences (effect size = 1)}
\NormalTok{true\_biomarkers }\OtherTok{\textless{}{-}} \FunctionTok{c}\NormalTok{(}\FunctionTok{rep}\NormalTok{(}\ConstantTok{FALSE}\NormalTok{, }\DecValTok{475}\NormalTok{), }\FunctionTok{rep}\NormalTok{(}\ConstantTok{TRUE}\NormalTok{, }\DecValTok{25}\NormalTok{))}

\NormalTok{biomarker\_pvalues }\OtherTok{\textless{}{-}} \FunctionTok{numeric}\NormalTok{(n\_biomarkers)}

\ControlFlowTok{for}\NormalTok{(i }\ControlFlowTok{in} \DecValTok{1}\SpecialCharTok{:}\NormalTok{n\_biomarkers) \{}
  \CommentTok{\# Control group}
\NormalTok{  control\_values }\OtherTok{\textless{}{-}} \FunctionTok{rnorm}\NormalTok{(n\_control, }\AttributeTok{mean =} \DecValTok{0}\NormalTok{, }\AttributeTok{sd =} \DecValTok{1}\NormalTok{)}

  \CommentTok{\# COVID group (with potential elevation)}
\NormalTok{  covid\_mean }\OtherTok{\textless{}{-}} \FunctionTok{ifelse}\NormalTok{(true\_biomarkers[i], }\DecValTok{1}\NormalTok{, }\DecValTok{0}\NormalTok{)}
\NormalTok{  covid\_values }\OtherTok{\textless{}{-}} \FunctionTok{rnorm}\NormalTok{(n\_covid, }\AttributeTok{mean =}\NormalTok{ covid\_mean, }\AttributeTok{sd =} \DecValTok{1}\NormalTok{)}

  \CommentTok{\# Statistical test}
\NormalTok{  test\_result }\OtherTok{\textless{}{-}} \FunctionTok{t.test}\NormalTok{(covid\_values, control\_values)}
\NormalTok{  biomarker\_pvalues[i] }\OtherTok{\textless{}{-}}\NormalTok{ test\_result}\SpecialCharTok{$}\NormalTok{p.value}
\NormalTok{\}}

\CommentTok{\# Apply corrections}
\NormalTok{alpha }\OtherTok{\textless{}{-}} \FloatTok{0.05}
\NormalTok{bonferroni\_results }\OtherTok{\textless{}{-}} \FunctionTok{p.adjust}\NormalTok{(biomarker\_pvalues, }\AttributeTok{method =} \StringTok{"bonferroni"}\NormalTok{)}
\NormalTok{fdr\_results }\OtherTok{\textless{}{-}} \FunctionTok{p.adjust}\NormalTok{(biomarker\_pvalues, }\AttributeTok{method =} \StringTok{"BH"}\NormalTok{)}

\CommentTok{\# Create results dataframe}
\NormalTok{covid\_results }\OtherTok{\textless{}{-}} \FunctionTok{data.frame}\NormalTok{(}
  \AttributeTok{Biomarker =} \FunctionTok{paste0}\NormalTok{(}\StringTok{"Biomarker\_"}\NormalTok{, }\DecValTok{1}\SpecialCharTok{:}\NormalTok{n\_biomarkers),}
  \AttributeTok{True\_Effect =}\NormalTok{ true\_biomarkers,}
  \AttributeTok{Raw\_P =}\NormalTok{ biomarker\_pvalues,}
  \AttributeTok{Bonferroni\_P =}\NormalTok{ bonferroni\_results,}
  \AttributeTok{FDR\_Q =}\NormalTok{ fdr\_results,}
  \AttributeTok{Raw\_Sig =}\NormalTok{ biomarker\_pvalues }\SpecialCharTok{\textless{}}\NormalTok{ alpha,}
  \AttributeTok{Bonferroni\_Sig =}\NormalTok{ bonferroni\_results }\SpecialCharTok{\textless{}}\NormalTok{ alpha,}
  \AttributeTok{FDR\_Sig =}\NormalTok{ fdr\_results }\SpecialCharTok{\textless{}}\NormalTok{ alpha}
\NormalTok{)}

\CommentTok{\# Performance summary}
\NormalTok{performance\_summary }\OtherTok{\textless{}{-}} \FunctionTok{data.frame}\NormalTok{(}
  \AttributeTok{Method =} \FunctionTok{c}\NormalTok{(}\StringTok{"Raw P{-}values"}\NormalTok{, }\StringTok{"Bonferroni"}\NormalTok{, }\StringTok{"FDR"}\NormalTok{),}
  \AttributeTok{Total\_Discoveries =} \FunctionTok{c}\NormalTok{(}
    \FunctionTok{sum}\NormalTok{(covid\_results}\SpecialCharTok{$}\NormalTok{Raw\_Sig),}
    \FunctionTok{sum}\NormalTok{(covid\_results}\SpecialCharTok{$}\NormalTok{Bonferroni\_Sig),}
    \FunctionTok{sum}\NormalTok{(covid\_results}\SpecialCharTok{$}\NormalTok{FDR\_Sig)}
\NormalTok{  ),}
  \AttributeTok{True\_Discoveries =} \FunctionTok{c}\NormalTok{(}
    \FunctionTok{sum}\NormalTok{(covid\_results}\SpecialCharTok{$}\NormalTok{Raw\_Sig }\SpecialCharTok{\&}\NormalTok{ covid\_results}\SpecialCharTok{$}\NormalTok{True\_Effect),}
    \FunctionTok{sum}\NormalTok{(covid\_results}\SpecialCharTok{$}\NormalTok{Bonferroni\_Sig }\SpecialCharTok{\&}\NormalTok{ covid\_results}\SpecialCharTok{$}\NormalTok{True\_Effect),}
    \FunctionTok{sum}\NormalTok{(covid\_results}\SpecialCharTok{$}\NormalTok{FDR\_Sig }\SpecialCharTok{\&}\NormalTok{ covid\_results}\SpecialCharTok{$}\NormalTok{True\_Effect)}
\NormalTok{  ),}
  \AttributeTok{False\_Discoveries =} \FunctionTok{c}\NormalTok{(}
    \FunctionTok{sum}\NormalTok{(covid\_results}\SpecialCharTok{$}\NormalTok{Raw\_Sig }\SpecialCharTok{\&} \SpecialCharTok{!}\NormalTok{covid\_results}\SpecialCharTok{$}\NormalTok{True\_Effect),}
    \FunctionTok{sum}\NormalTok{(covid\_results}\SpecialCharTok{$}\NormalTok{Bonferroni\_Sig }\SpecialCharTok{\&} \SpecialCharTok{!}\NormalTok{covid\_results}\SpecialCharTok{$}\NormalTok{True\_Effect),}
    \FunctionTok{sum}\NormalTok{(covid\_results}\SpecialCharTok{$}\NormalTok{FDR\_Sig }\SpecialCharTok{\&} \SpecialCharTok{!}\NormalTok{covid\_results}\SpecialCharTok{$}\NormalTok{True\_Effect)}
\NormalTok{  )}
\NormalTok{)}

\NormalTok{performance\_summary}\SpecialCharTok{$}\NormalTok{Sensitivity }\OtherTok{\textless{}{-}}\NormalTok{ performance\_summary}\SpecialCharTok{$}\NormalTok{True\_Discoveries }\SpecialCharTok{/} \DecValTok{25}
\NormalTok{performance\_summary}\SpecialCharTok{$}\NormalTok{FDR\_Actual }\OtherTok{\textless{}{-}}\NormalTok{ performance\_summary}\SpecialCharTok{$}\NormalTok{False\_Discoveries }\SpecialCharTok{/}
  \FunctionTok{pmax}\NormalTok{(}\DecValTok{1}\NormalTok{, performance\_summary}\SpecialCharTok{$}\NormalTok{Total\_Discoveries)}

\FunctionTok{print}\NormalTok{(performance\_summary)}

\CommentTok{\# Visualization}
\FunctionTok{library}\NormalTok{(ggplot2)}
\NormalTok{volcano\_data }\OtherTok{\textless{}{-}} \FunctionTok{data.frame}\NormalTok{(}
  \AttributeTok{Log\_P =} \SpecialCharTok{{-}}\FunctionTok{log10}\NormalTok{(covid\_results}\SpecialCharTok{$}\NormalTok{Raw\_P),}
  \AttributeTok{Effect\_Size =} \FunctionTok{ifelse}\NormalTok{(covid\_results}\SpecialCharTok{$}\NormalTok{True\_Effect, }\DecValTok{1}\NormalTok{, }\DecValTok{0}\NormalTok{) }\SpecialCharTok{+} \FunctionTok{rnorm}\NormalTok{(n\_biomarkers, }\DecValTok{0}\NormalTok{, }\FloatTok{0.1}\NormalTok{),}
  \AttributeTok{Significant =}\NormalTok{ covid\_results}\SpecialCharTok{$}\NormalTok{FDR\_Sig,}
  \AttributeTok{True\_Effect =}\NormalTok{ covid\_results}\SpecialCharTok{$}\NormalTok{True\_Effect}
\NormalTok{)}

\FunctionTok{ggplot}\NormalTok{(volcano\_data, }\FunctionTok{aes}\NormalTok{(}\AttributeTok{x =}\NormalTok{ Effect\_Size, }\AttributeTok{y =}\NormalTok{ Log\_P)) }\SpecialCharTok{+}
  \FunctionTok{geom\_point}\NormalTok{(}\FunctionTok{aes}\NormalTok{(}\AttributeTok{color =} \FunctionTok{interaction}\NormalTok{(Significant, True\_Effect)), }\AttributeTok{alpha =} \FloatTok{0.6}\NormalTok{) }\SpecialCharTok{+}
  \FunctionTok{scale\_color\_manual}\NormalTok{(}
    \AttributeTok{values =} \FunctionTok{c}\NormalTok{(}\StringTok{"TRUE.FALSE"} \OtherTok{=} \StringTok{"red"}\NormalTok{, }\StringTok{"FALSE.FALSE"} \OtherTok{=} \StringTok{"gray"}\NormalTok{,}
               \StringTok{"TRUE.TRUE"} \OtherTok{=} \StringTok{"blue"}\NormalTok{, }\StringTok{"FALSE.TRUE"} \OtherTok{=} \StringTok{"orange"}\NormalTok{),}
    \AttributeTok{labels =} \FunctionTok{c}\NormalTok{(}\StringTok{"FDR Sig + True Effect"}\NormalTok{, }\StringTok{"Not Significant"}\NormalTok{,}
               \StringTok{"FDR Sig + No True Effect"}\NormalTok{, }\StringTok{"Not Sig + True Effect"}\NormalTok{)}
\NormalTok{  ) }\SpecialCharTok{+}
  \FunctionTok{geom\_hline}\NormalTok{(}\AttributeTok{yintercept =} \SpecialCharTok{{-}}\FunctionTok{log10}\NormalTok{(}\FloatTok{0.05}\NormalTok{), }\AttributeTok{linetype =} \StringTok{"dashed"}\NormalTok{, }\AttributeTok{color =} \StringTok{"red"}\NormalTok{) }\SpecialCharTok{+}
  \FunctionTok{labs}\NormalTok{(}
    \AttributeTok{title =} \StringTok{"COVID{-}19 Biomarker Discovery Results"}\NormalTok{,}
    \AttributeTok{x =} \StringTok{"Effect Size"}\NormalTok{,}
    \AttributeTok{y =} \StringTok{"{-}log10(p{-}value)"}\NormalTok{,}
    \AttributeTok{color =} \StringTok{"Classification"}
\NormalTok{  ) }\SpecialCharTok{+}
  \FunctionTok{theme\_minimal}\NormalTok{()}
\end{Highlighting}
\end{Shaded}

\section{Python Implementation}\label{python-implementation-1}

\begin{Shaded}
\begin{Highlighting}[]
\ImportTok{import}\NormalTok{ numpy }\ImportTok{as}\NormalTok{ np}
\ImportTok{import}\NormalTok{ pandas }\ImportTok{as}\NormalTok{ pd}
\ImportTok{import}\NormalTok{ matplotlib.pyplot }\ImportTok{as}\NormalTok{ plt}
\ImportTok{import}\NormalTok{ seaborn }\ImportTok{as}\NormalTok{ sns}
\ImportTok{from}\NormalTok{ scipy }\ImportTok{import}\NormalTok{ stats}
\ImportTok{from}\NormalTok{ statsmodels.stats.multitest }\ImportTok{import}\NormalTok{ multipletests}

\CommentTok{\# Simulate multiple testing scenario}
\NormalTok{np.random.seed(}\DecValTok{123}\NormalTok{)}
\NormalTok{n\_tests }\OperatorTok{=} \DecValTok{1000}
\NormalTok{n\_true\_effects }\OperatorTok{=} \DecValTok{50}

\CommentTok{\# Generate p{-}values}
\NormalTok{p\_values }\OperatorTok{=}\NormalTok{ []}
\NormalTok{true\_effects }\OperatorTok{=}\NormalTok{ np.array([}\VariableTok{False}\NormalTok{] }\OperatorTok{*}\NormalTok{ (n\_tests }\OperatorTok{{-}}\NormalTok{ n\_true\_effects) }\OperatorTok{+}\NormalTok{ [}\VariableTok{True}\NormalTok{] }\OperatorTok{*}\NormalTok{ n\_true\_effects)}

\ControlFlowTok{for}\NormalTok{ i }\KeywordTok{in} \BuiltInTok{range}\NormalTok{(n\_tests):}
    \CommentTok{\# Generate data}
    \ControlFlowTok{if}\NormalTok{ true\_effects[i]:}
        \CommentTok{\# True effect: different means}
\NormalTok{        group1 }\OperatorTok{=}\NormalTok{ np.random.normal(}\DecValTok{0}\NormalTok{, }\DecValTok{1}\NormalTok{, }\DecValTok{30}\NormalTok{)}
\NormalTok{        group2 }\OperatorTok{=}\NormalTok{ np.random.normal(}\DecValTok{1}\NormalTok{, }\DecValTok{1}\NormalTok{, }\DecValTok{30}\NormalTok{)  }\CommentTok{\# Effect size = 1}
    \ControlFlowTok{else}\NormalTok{:}
        \CommentTok{\# No effect: same means}
\NormalTok{        group1 }\OperatorTok{=}\NormalTok{ np.random.normal(}\DecValTok{0}\NormalTok{, }\DecValTok{1}\NormalTok{, }\DecValTok{30}\NormalTok{)}
\NormalTok{        group2 }\OperatorTok{=}\NormalTok{ np.random.normal(}\DecValTok{0}\NormalTok{, }\DecValTok{1}\NormalTok{, }\DecValTok{30}\NormalTok{)}

    \CommentTok{\# Perform t{-}test}
\NormalTok{    \_, p\_val }\OperatorTok{=}\NormalTok{ stats.ttest\_ind(group1, group2)}
\NormalTok{    p\_values.append(p\_val)}

\NormalTok{p\_values }\OperatorTok{=}\NormalTok{ np.array(p\_values)}

\CommentTok{\# Apply multiple testing corrections}
\NormalTok{rejected\_bonf, p\_adj\_bonf, \_, \_ }\OperatorTok{=}\NormalTok{ multipletests(p\_values, method}\OperatorTok{=}\StringTok{\textquotesingle{}bonferroni\textquotesingle{}}\NormalTok{)}
\NormalTok{rejected\_bh, p\_adj\_bh, \_, \_ }\OperatorTok{=}\NormalTok{ multipletests(p\_values, method}\OperatorTok{=}\StringTok{\textquotesingle{}fdr\_bh\textquotesingle{}}\NormalTok{)}

\CommentTok{\# Create results dataframe}
\NormalTok{results }\OperatorTok{=}\NormalTok{ pd.DataFrame(\{}
    \StringTok{\textquotesingle{}p\_value\textquotesingle{}}\NormalTok{: p\_values,}
    \StringTok{\textquotesingle{}true\_effect\textquotesingle{}}\NormalTok{: true\_effects,}
    \StringTok{\textquotesingle{}bonferroni\_adj\textquotesingle{}}\NormalTok{: p\_adj\_bonf,}
    \StringTok{\textquotesingle{}fdr\_adj\textquotesingle{}}\NormalTok{: p\_adj\_bh,}
    \StringTok{\textquotesingle{}bonferroni\_sig\textquotesingle{}}\NormalTok{: rejected\_bonf,}
    \StringTok{\textquotesingle{}fdr\_sig\textquotesingle{}}\NormalTok{: rejected\_bh}
\NormalTok{\})}

\CommentTok{\# Calculate performance metrics}
\KeywordTok{def}\NormalTok{ calculate\_performance(significant, true\_effects):}
\NormalTok{    tp }\OperatorTok{=}\NormalTok{ np.}\BuiltInTok{sum}\NormalTok{(significant }\OperatorTok{\&}\NormalTok{ true\_effects)}
\NormalTok{    fp }\OperatorTok{=}\NormalTok{ np.}\BuiltInTok{sum}\NormalTok{(significant }\OperatorTok{\&} \OperatorTok{\textasciitilde{}}\NormalTok{true\_effects)}
\NormalTok{    fn }\OperatorTok{=}\NormalTok{ np.}\BuiltInTok{sum}\NormalTok{(}\OperatorTok{\textasciitilde{}}\NormalTok{significant }\OperatorTok{\&}\NormalTok{ true\_effects)}
\NormalTok{    tn }\OperatorTok{=}\NormalTok{ np.}\BuiltInTok{sum}\NormalTok{(}\OperatorTok{\textasciitilde{}}\NormalTok{significant }\OperatorTok{\&} \OperatorTok{\textasciitilde{}}\NormalTok{true\_effects)}

\NormalTok{    sensitivity }\OperatorTok{=}\NormalTok{ tp }\OperatorTok{/}\NormalTok{ (tp }\OperatorTok{+}\NormalTok{ fn) }\ControlFlowTok{if}\NormalTok{ (tp }\OperatorTok{+}\NormalTok{ fn) }\OperatorTok{\textgreater{}} \DecValTok{0} \ControlFlowTok{else} \DecValTok{0}
\NormalTok{    fdr }\OperatorTok{=}\NormalTok{ fp }\OperatorTok{/}\NormalTok{ (tp }\OperatorTok{+}\NormalTok{ fp) }\ControlFlowTok{if}\NormalTok{ (tp }\OperatorTok{+}\NormalTok{ fp) }\OperatorTok{\textgreater{}} \DecValTok{0} \ControlFlowTok{else} \DecValTok{0}

    \ControlFlowTok{return}\NormalTok{ \{}
        \StringTok{\textquotesingle{}discoveries\textquotesingle{}}\NormalTok{: tp }\OperatorTok{+}\NormalTok{ fp,}
        \StringTok{\textquotesingle{}true\_positives\textquotesingle{}}\NormalTok{: tp,}
        \StringTok{\textquotesingle{}false\_positives\textquotesingle{}}\NormalTok{: fp,}
        \StringTok{\textquotesingle{}sensitivity\textquotesingle{}}\NormalTok{: sensitivity,}
        \StringTok{\textquotesingle{}fdr\textquotesingle{}}\NormalTok{: fdr}
\NormalTok{    \}}

\CommentTok{\# Performance comparison}
\NormalTok{raw\_sig }\OperatorTok{=}\NormalTok{ p\_values }\OperatorTok{\textless{}} \FloatTok{0.05}
\NormalTok{perf\_raw }\OperatorTok{=}\NormalTok{ calculate\_performance(raw\_sig, true\_effects)}
\NormalTok{perf\_bonf }\OperatorTok{=}\NormalTok{ calculate\_performance(rejected\_bonf, true\_effects)}
\NormalTok{perf\_fdr }\OperatorTok{=}\NormalTok{ calculate\_performance(rejected\_bh, true\_effects)}

\BuiltInTok{print}\NormalTok{(}\StringTok{"Performance Comparison:"}\NormalTok{)}
\BuiltInTok{print}\NormalTok{(}\StringTok{"======================="}\NormalTok{)}
\ControlFlowTok{for}\NormalTok{ method, perf }\KeywordTok{in}\NormalTok{ [(}\StringTok{"Raw"}\NormalTok{, perf\_raw), (}\StringTok{"Bonferroni"}\NormalTok{, perf\_bonf), (}\StringTok{"FDR"}\NormalTok{, perf\_fdr)]:}
    \BuiltInTok{print}\NormalTok{(}\SpecialStringTok{f"}\SpecialCharTok{\{}\NormalTok{method}\SpecialCharTok{\}}\SpecialStringTok{:"}\NormalTok{)}
    \BuiltInTok{print}\NormalTok{(}\SpecialStringTok{f"  Discoveries: }\SpecialCharTok{\{}\NormalTok{perf[}\StringTok{\textquotesingle{}discoveries\textquotesingle{}}\NormalTok{]}\SpecialCharTok{\}}\SpecialStringTok{"}\NormalTok{)}
    \BuiltInTok{print}\NormalTok{(}\SpecialStringTok{f"  Sensitivity: }\SpecialCharTok{\{}\NormalTok{perf[}\StringTok{\textquotesingle{}sensitivity\textquotesingle{}}\NormalTok{]}\SpecialCharTok{:.3f\}}\SpecialStringTok{"}\NormalTok{)}
    \BuiltInTok{print}\NormalTok{(}\SpecialStringTok{f"  FDR: }\SpecialCharTok{\{}\NormalTok{perf[}\StringTok{\textquotesingle{}fdr\textquotesingle{}}\NormalTok{]}\SpecialCharTok{:.3f\}}\SpecialStringTok{"}\NormalTok{)}
    \BuiltInTok{print}\NormalTok{()}

\CommentTok{\# Visualization}
\NormalTok{fig, (ax1, ax2) }\OperatorTok{=}\NormalTok{ plt.subplots(}\DecValTok{1}\NormalTok{, }\DecValTok{2}\NormalTok{, figsize}\OperatorTok{=}\NormalTok{(}\DecValTok{15}\NormalTok{, }\DecValTok{6}\NormalTok{))}

\CommentTok{\# Histogram of p{-}values}
\NormalTok{ax1.hist(p\_values[}\OperatorTok{\textasciitilde{}}\NormalTok{true\_effects], bins}\OperatorTok{=}\DecValTok{50}\NormalTok{, alpha}\OperatorTok{=}\FloatTok{0.7}\NormalTok{, label}\OperatorTok{=}\StringTok{\textquotesingle{}Null hypotheses\textquotesingle{}}\NormalTok{, density}\OperatorTok{=}\VariableTok{True}\NormalTok{)}
\NormalTok{ax1.hist(p\_values[true\_effects], bins}\OperatorTok{=}\DecValTok{50}\NormalTok{, alpha}\OperatorTok{=}\FloatTok{0.7}\NormalTok{, label}\OperatorTok{=}\StringTok{\textquotesingle{}Alternative hypotheses\textquotesingle{}}\NormalTok{, density}\OperatorTok{=}\VariableTok{True}\NormalTok{)}
\NormalTok{ax1.axhline(y}\OperatorTok{=}\DecValTok{1}\NormalTok{, color}\OperatorTok{=}\StringTok{\textquotesingle{}red\textquotesingle{}}\NormalTok{, linestyle}\OperatorTok{=}\StringTok{\textquotesingle{}{-}{-}\textquotesingle{}}\NormalTok{, label}\OperatorTok{=}\StringTok{\textquotesingle{}Uniform (null)\textquotesingle{}}\NormalTok{)}
\NormalTok{ax1.set\_xlabel(}\StringTok{\textquotesingle{}P{-}value\textquotesingle{}}\NormalTok{)}
\NormalTok{ax1.set\_ylabel(}\StringTok{\textquotesingle{}Density\textquotesingle{}}\NormalTok{)}
\NormalTok{ax1.set\_title(}\StringTok{\textquotesingle{}Distribution of P{-}values\textquotesingle{}}\NormalTok{)}
\NormalTok{ax1.legend()}

\CommentTok{\# Comparison of methods}
\NormalTok{methods }\OperatorTok{=}\NormalTok{ [}\StringTok{\textquotesingle{}Raw\textquotesingle{}}\NormalTok{, }\StringTok{\textquotesingle{}Bonferroni\textquotesingle{}}\NormalTok{, }\StringTok{\textquotesingle{}FDR\textquotesingle{}}\NormalTok{]}
\NormalTok{discoveries }\OperatorTok{=}\NormalTok{ [perf\_raw[}\StringTok{\textquotesingle{}discoveries\textquotesingle{}}\NormalTok{], perf\_bonf[}\StringTok{\textquotesingle{}discoveries\textquotesingle{}}\NormalTok{], perf\_fdr[}\StringTok{\textquotesingle{}discoveries\textquotesingle{}}\NormalTok{]]}
\NormalTok{true\_pos }\OperatorTok{=}\NormalTok{ [perf\_raw[}\StringTok{\textquotesingle{}true\_positives\textquotesingle{}}\NormalTok{], perf\_bonf[}\StringTok{\textquotesingle{}true\_positives\textquotesingle{}}\NormalTok{], perf\_fdr[}\StringTok{\textquotesingle{}true\_positives\textquotesingle{}}\NormalTok{]]}
\NormalTok{false\_pos }\OperatorTok{=}\NormalTok{ [perf\_raw[}\StringTok{\textquotesingle{}false\_positives\textquotesingle{}}\NormalTok{], perf\_bonf[}\StringTok{\textquotesingle{}false\_positives\textquotesingle{}}\NormalTok{], perf\_fdr[}\StringTok{\textquotesingle{}false\_positives\textquotesingle{}}\NormalTok{]]}

\NormalTok{x }\OperatorTok{=}\NormalTok{ np.arange(}\BuiltInTok{len}\NormalTok{(methods))}
\NormalTok{width }\OperatorTok{=} \FloatTok{0.35}

\NormalTok{ax2.bar(x }\OperatorTok{{-}}\NormalTok{ width}\OperatorTok{/}\DecValTok{2}\NormalTok{, true\_pos, width, label}\OperatorTok{=}\StringTok{\textquotesingle{}True Positives\textquotesingle{}}\NormalTok{, color}\OperatorTok{=}\StringTok{\textquotesingle{}green\textquotesingle{}}\NormalTok{, alpha}\OperatorTok{=}\FloatTok{0.7}\NormalTok{)}
\NormalTok{ax2.bar(x }\OperatorTok{+}\NormalTok{ width}\OperatorTok{/}\DecValTok{2}\NormalTok{, false\_pos, width, label}\OperatorTok{=}\StringTok{\textquotesingle{}False Positives\textquotesingle{}}\NormalTok{, color}\OperatorTok{=}\StringTok{\textquotesingle{}red\textquotesingle{}}\NormalTok{, alpha}\OperatorTok{=}\FloatTok{0.7}\NormalTok{)}
\NormalTok{ax2.set\_xlabel(}\StringTok{\textquotesingle{}Method\textquotesingle{}}\NormalTok{)}
\NormalTok{ax2.set\_ylabel(}\StringTok{\textquotesingle{}Count\textquotesingle{}}\NormalTok{)}
\NormalTok{ax2.set\_title(}\StringTok{\textquotesingle{}True vs False Positives by Method\textquotesingle{}}\NormalTok{)}
\NormalTok{ax2.set\_xticks(x)}
\NormalTok{ax2.set\_xticklabels(methods)}
\NormalTok{ax2.legend()}

\NormalTok{plt.tight\_layout()}
\NormalTok{plt.show()}
\end{Highlighting}
\end{Shaded}

\bookmarksetup{startatroot}

\chapter{Specialized \& Modern
Methods}\label{specialized-modern-methods}

Advanced techniques for complex data scenarios, from time-to-event
analysis to modern computational approaches in the era of big data.

\emph{Note: This chapter appears to be incomplete in the source HTML
file. Only the title and introduction were available. Additional content
for specialized and modern methods would typically include:}

\begin{itemize}
\tightlist
\item
  Survival analysis (Kaplan-Meier, Cox regression)
\item
  Non-parametric methods (Mann-Whitney, Kruskal-Wallis, etc.)
\item
  Bootstrap and resampling methods
\item
  Mixed-effects models for repeated measures
\item
  Time series analysis
\item
  Machine learning approaches in biostatistics
\item
  Bayesian methods
\item
  Meta-analysis techniques
\item
  Handling missing data (imputation methods)
\item
  High-dimensional data analysis
\end{itemize}

\emph{These topics would be covered in a complete version of this
chapter.}

\bookmarksetup{startatroot}

\chapter*{References}\label{references}
\addcontentsline{toc}{chapter}{References}

\markboth{References}{References}

\phantomsection\label{refs}
\begin{CSLReferences}{0}{1}
\end{CSLReferences}

\section*{Recommended Reading}\label{recommended-reading}
\addcontentsline{toc}{section}{Recommended Reading}

\markright{Recommended Reading}

\subsection*{Foundational Texts}\label{foundational-texts}
\addcontentsline{toc}{subsection}{Foundational Texts}

\begin{itemize}
\tightlist
\item
  \textbf{Rosner, B.} (2015). \emph{Fundamentals of Biostatistics} (8th
  ed.). Cengage Learning.
\item
  \textbf{Pagano, M., \& Gauvreau, K.} (2000). \emph{Principles of
  Biostatistics} (2nd ed.). Duxbury Press.
\item
  \textbf{Jewell, N. P.} (2003). \emph{Statistics for Epidemiology}.
  Chapman and Hall/CRC.
\end{itemize}

\subsection*{Advanced Methods}\label{advanced-methods}
\addcontentsline{toc}{subsection}{Advanced Methods}

\begin{itemize}
\tightlist
\item
  \textbf{Kleinbaum, D. G., \& Klein, M.} (2012). \emph{Survival
  Analysis: A Self-Learning Text} (3rd ed.). Springer.
\item
  \textbf{Hosmer, D. W., Lemeshow, S., \& Sturdivant, R. X.} (2013).
  \emph{Applied Logistic Regression} (3rd ed.). Wiley.
\item
  \textbf{Diggle, P., Heagerty, P., Liang, K. Y., \& Zeger, S.} (2002).
  \emph{Analysis of Longitudinal Data} (2nd ed.). Oxford University
  Press.
\end{itemize}

\subsection*{Statistical Software}\label{statistical-software}
\addcontentsline{toc}{subsection}{Statistical Software}

\begin{itemize}
\tightlist
\item
  \textbf{R Core Team} (2023). \emph{R: A language and environment for
  statistical computing}. R Foundation for Statistical Computing.
  https://www.R-project.org/
\item
  \textbf{Wickham, H., \& Grolemund, G.} (2017). \emph{R for Data
  Science}. O'Reilly Media. https://r4ds.had.co.nz/
\item
  \textbf{VanderPlas, J.} (2016). \emph{Python Data Science Handbook}.
  O'Reilly Media.
\end{itemize}

\subsection*{Methodological Resources}\label{methodological-resources}
\addcontentsline{toc}{subsection}{Methodological Resources}

\begin{itemize}
\tightlist
\item
  \textbf{Altman, D. G.} (1991). \emph{Practical Statistics for Medical
  Research}. Chapman and Hall/CRC.
\item
  \textbf{Bland, M.} (2015). \emph{An Introduction to Medical
  Statistics} (4th ed.). Oxford University Press.
\item
  \textbf{Zar, J. H.} (2013). \emph{Biostatistical Analysis} (5th ed.).
  Pearson.
\end{itemize}

\subsection*{Online Resources}\label{online-resources}
\addcontentsline{toc}{subsection}{Online Resources}

\begin{itemize}
\tightlist
\item
  \textbf{Khan Academy Statistics and Probability}:
  https://www.khanacademy.org/math/statistics-probability
\item
  \textbf{Coursera Biostatistics Courses}: Various universities offer
  comprehensive biostatistics courses
\item
  \textbf{edX Statistics Courses}: Free courses from top universities
\item
  \textbf{R Documentation}: https://www.rdocumentation.org/
\item
  \textbf{Python Statistics Libraries}: SciPy, StatsModels, Scikit-learn
  documentation
\end{itemize}


\backmatter


\end{document}
