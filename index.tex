% Options for packages loaded elsewhere
\PassOptionsToPackage{unicode}{hyperref}
\PassOptionsToPackage{hyphens}{url}
\PassOptionsToPackage{dvipsnames,svgnames,x11names}{xcolor}
%
\documentclass[
  11pt,
  letterpaper,
  oneside]{book}

\usepackage{amsmath,amssymb}
\usepackage{setspace}
\usepackage{iftex}
\ifPDFTeX
  \usepackage[T1]{fontenc}
  \usepackage[utf8]{inputenc}
  \usepackage{textcomp} % provide euro and other symbols
\else % if luatex or xetex
  \usepackage{unicode-math}
  \defaultfontfeatures{Scale=MatchLowercase}
  \defaultfontfeatures[\rmfamily]{Ligatures=TeX,Scale=1}
\fi
\usepackage[]{crimson}
\ifPDFTeX\else  
    % xetex/luatex font selection
\fi
% Use upquote if available, for straight quotes in verbatim environments
\IfFileExists{upquote.sty}{\usepackage{upquote}}{}
\IfFileExists{microtype.sty}{% use microtype if available
  \usepackage[]{microtype}
  \UseMicrotypeSet[protrusion]{basicmath} % disable protrusion for tt fonts
}{}
\makeatletter
\@ifundefined{KOMAClassName}{% if non-KOMA class
  \IfFileExists{parskip.sty}{%
    \usepackage{parskip}
  }{% else
    \setlength{\parindent}{0pt}
    \setlength{\parskip}{6pt plus 2pt minus 1pt}}
}{% if KOMA class
  \KOMAoptions{parskip=half}}
\makeatother
\usepackage{xcolor}
\usepackage[top=1in,bottom=1in,left=1.2in,right=1in]{geometry}
\setlength{\emergencystretch}{3em} % prevent overfull lines
\setcounter{secnumdepth}{5}
% Make \paragraph and \subparagraph free-standing
\makeatletter
\ifx\paragraph\undefined\else
  \let\oldparagraph\paragraph
  \renewcommand{\paragraph}{
    \@ifstar
      \xxxParagraphStar
      \xxxParagraphNoStar
  }
  \newcommand{\xxxParagraphStar}[1]{\oldparagraph*{#1}\mbox{}}
  \newcommand{\xxxParagraphNoStar}[1]{\oldparagraph{#1}\mbox{}}
\fi
\ifx\subparagraph\undefined\else
  \let\oldsubparagraph\subparagraph
  \renewcommand{\subparagraph}{
    \@ifstar
      \xxxSubParagraphStar
      \xxxSubParagraphNoStar
  }
  \newcommand{\xxxSubParagraphStar}[1]{\oldsubparagraph*{#1}\mbox{}}
  \newcommand{\xxxSubParagraphNoStar}[1]{\oldsubparagraph{#1}\mbox{}}
\fi
\makeatother


\providecommand{\tightlist}{%
  \setlength{\itemsep}{0pt}\setlength{\parskip}{0pt}}\usepackage{longtable,booktabs,array}
\usepackage{calc} % for calculating minipage widths
% Correct order of tables after \paragraph or \subparagraph
\usepackage{etoolbox}
\makeatletter
\patchcmd\longtable{\par}{\if@noskipsec\mbox{}\fi\par}{}{}
\makeatother
% Allow footnotes in longtable head/foot
\IfFileExists{footnotehyper.sty}{\usepackage{footnotehyper}}{\usepackage{footnote}}
\makesavenoteenv{longtable}
\usepackage{graphicx}
\makeatletter
\def\maxwidth{\ifdim\Gin@nat@width>\linewidth\linewidth\else\Gin@nat@width\fi}
\def\maxheight{\ifdim\Gin@nat@height>\textheight\textheight\else\Gin@nat@height\fi}
\makeatother
% Scale images if necessary, so that they will not overflow the page
% margins by default, and it is still possible to overwrite the defaults
% using explicit options in \includegraphics[width, height, ...]{}
\setkeys{Gin}{width=\maxwidth,height=\maxheight,keepaspectratio}
% Set default figure placement to htbp
\makeatletter
\def\fps@figure{htbp}
\makeatother
% definitions for citeproc citations
\NewDocumentCommand\citeproctext{}{}
\NewDocumentCommand\citeproc{mm}{%
  \begingroup\def\citeproctext{#2}\cite{#1}\endgroup}
\makeatletter
 % allow citations to break across lines
 \let\@cite@ofmt\@firstofone
 % avoid brackets around text for \cite:
 \def\@biblabel#1{}
 \def\@cite#1#2{{#1\if@tempswa , #2\fi}}
\makeatother
\newlength{\cslhangindent}
\setlength{\cslhangindent}{1.5em}
\newlength{\csllabelwidth}
\setlength{\csllabelwidth}{3em}
\newenvironment{CSLReferences}[2] % #1 hanging-indent, #2 entry-spacing
 {\begin{list}{}{%
  \setlength{\itemindent}{0pt}
  \setlength{\leftmargin}{0pt}
  \setlength{\parsep}{0pt}
  % turn on hanging indent if param 1 is 1
  \ifodd #1
   \setlength{\leftmargin}{\cslhangindent}
   \setlength{\itemindent}{-1\cslhangindent}
  \fi
  % set entry spacing
  \setlength{\itemsep}{#2\baselineskip}}}
 {\end{list}}
\usepackage{calc}
\newcommand{\CSLBlock}[1]{\hfill\break\parbox[t]{\linewidth}{\strut\ignorespaces#1\strut}}
\newcommand{\CSLLeftMargin}[1]{\parbox[t]{\csllabelwidth}{\strut#1\strut}}
\newcommand{\CSLRightInline}[1]{\parbox[t]{\linewidth - \csllabelwidth}{\strut#1\strut}}
\newcommand{\CSLIndent}[1]{\hspace{\cslhangindent}#1}

\makeatletter
\@ifpackageloaded{tcolorbox}{}{\usepackage[skins,breakable]{tcolorbox}}
\@ifpackageloaded{fontawesome5}{}{\usepackage{fontawesome5}}
\definecolor{quarto-callout-color}{HTML}{909090}
\definecolor{quarto-callout-note-color}{HTML}{0758E5}
\definecolor{quarto-callout-important-color}{HTML}{CC1914}
\definecolor{quarto-callout-warning-color}{HTML}{EB9113}
\definecolor{quarto-callout-tip-color}{HTML}{00A047}
\definecolor{quarto-callout-caution-color}{HTML}{FC5300}
\definecolor{quarto-callout-color-frame}{HTML}{acacac}
\definecolor{quarto-callout-note-color-frame}{HTML}{4582ec}
\definecolor{quarto-callout-important-color-frame}{HTML}{d9534f}
\definecolor{quarto-callout-warning-color-frame}{HTML}{f0ad4e}
\definecolor{quarto-callout-tip-color-frame}{HTML}{02b875}
\definecolor{quarto-callout-caution-color-frame}{HTML}{fd7e14}
\makeatother
\makeatletter
\@ifpackageloaded{bookmark}{}{\usepackage{bookmark}}
\makeatother
\makeatletter
\@ifpackageloaded{caption}{}{\usepackage{caption}}
\AtBeginDocument{%
\ifdefined\contentsname
  \renewcommand*\contentsname{Table of contents}
\else
  \newcommand\contentsname{Table of contents}
\fi
\ifdefined\listfigurename
  \renewcommand*\listfigurename{List of Figures}
\else
  \newcommand\listfigurename{List of Figures}
\fi
\ifdefined\listtablename
  \renewcommand*\listtablename{List of Tables}
\else
  \newcommand\listtablename{List of Tables}
\fi
\ifdefined\figurename
  \renewcommand*\figurename{Figure}
\else
  \newcommand\figurename{Figure}
\fi
\ifdefined\tablename
  \renewcommand*\tablename{Table}
\else
  \newcommand\tablename{Table}
\fi
}
\@ifpackageloaded{float}{}{\usepackage{float}}
\floatstyle{ruled}
\@ifundefined{c@chapter}{\newfloat{codelisting}{h}{lop}}{\newfloat{codelisting}{h}{lop}[chapter]}
\floatname{codelisting}{Listing}
\newcommand*\listoflistings{\listof{codelisting}{List of Listings}}
\makeatother
\makeatletter
\makeatother
\makeatletter
\@ifpackageloaded{caption}{}{\usepackage{caption}}
\@ifpackageloaded{subcaption}{}{\usepackage{subcaption}}
\makeatother

\ifLuaTeX
\usepackage[bidi=basic]{babel}
\else
\usepackage[bidi=default]{babel}
\fi
\babelprovide[main,import]{english}
% get rid of language-specific shorthands (see #6817):
\let\LanguageShortHands\languageshorthands
\def\languageshorthands#1{}
\ifLuaTeX
  \usepackage{selnolig}  % disable illegal ligatures
\fi
\usepackage{bookmark}

\IfFileExists{xurl.sty}{\usepackage{xurl}}{} % add URL line breaks if available
\urlstyle{same} % disable monospaced font for URLs
\hypersetup{
  pdftitle={BioStatica - Complete Guide to Biostatistics},
  pdfauthor={Pawan Rama Mali},
  pdflang={en},
  colorlinks=true,
  linkcolor={Maroon},
  filecolor={Maroon},
  citecolor={Blue},
  urlcolor={Blue},
  pdfcreator={LaTeX via pandoc}}


\title{BioStatica - Complete Guide to Biostatistics}
\usepackage{etoolbox}
\makeatletter
\providecommand{\subtitle}[1]{% add subtitle to \maketitle
  \apptocmd{\@title}{\par {\large #1 \par}}{}{}
}
\makeatother
\subtitle{From foundational principles to modern methodologies used in
cutting-edge research}
\author{Pawan Rama Mali}
\date{2025-09-17}

\begin{document}
\frontmatter
\maketitle

\renewcommand*\contentsname{Table of contents}
{
\hypersetup{linkcolor=}
\setcounter{tocdepth}{2}
\tableofcontents
}

\setstretch{1.2}
\mainmatter
\bookmarksetup{startatroot}

\chapter*{Preface}\label{preface}
\addcontentsline{toc}{chapter}{Preface}

\markboth{Preface}{Preface}

BioStatica is a comprehensive, beginner-friendly platform offering
tutorials, interactive modules, and real-world examples to master
biostatistics---from foundational principles to modern methodologies
used in cutting-edge research.

\section*{About This Book}\label{about-this-book}
\addcontentsline{toc}{section}{About This Book}

\markright{About This Book}

This book delivers a structured learning pathway---from statistical
fundamentals to advanced methods in biostatistics---blending clarity
with rigor for learners at all levels. Whether you're a student
beginning your journey in biostatistics or a researcher looking to
deepen your understanding, this guide provides practical insights and
real-world applications.

\section*{Prerequisites}\label{prerequisites}
\addcontentsline{toc}{section}{Prerequisites}

\markright{Prerequisites}

Before diving into this book, you should have:

\begin{itemize}
\tightlist
\item
  Basic understanding of mathematics and algebra
\item
  Familiarity with statistical software (R, Python, or similar)
\item
  Access to a computer with statistical software installed
\end{itemize}

\section*{How to Use This Book}\label{how-to-use-this-book}
\addcontentsline{toc}{section}{How to Use This Book}

\markright{How to Use This Book}

Each chapter builds upon previous concepts, so we recommend reading them
in order, especially if you're new to biostatistics. However,
experienced readers may jump to specific topics of interest.

\section*{Interactive Learning}\label{interactive-learning}
\addcontentsline{toc}{section}{Interactive Learning}

\markright{Interactive Learning}

This book includes:

\begin{itemize}
\tightlist
\item
  Interactive examples and exercises
\item
  Real-world datasets for practice
\item
  Code snippets in R and Python
\item
  Visual demonstrations of statistical concepts
\item
  Practical applications in biological and medical research
\end{itemize}

\section*{Contributing}\label{contributing}
\addcontentsline{toc}{section}{Contributing}

\markright{Contributing}

This is an open-source project. Contributions, suggestions, and feedback
are welcome. Please visit our
\href{https://github.com/PawanRamaMali/BioStatica}{GitHub repository} to
contribute.

\section*{Acknowledgments}\label{acknowledgments}
\addcontentsline{toc}{section}{Acknowledgments}

\markright{Acknowledgments}

We thank the biostatistics and open science communities for their
continued support and contributions to making statistical education
accessible to all.

\begin{center}\rule{0.5\linewidth}{0.5pt}\end{center}

\emph{Ready to begin your journey into biostatistics? Let's start with
the introduction.}

\bookmarksetup{startatroot}

\chapter{Introduction: Welcome to
Biostatistics}\label{introduction-welcome-to-biostatistics}

Biostatistics is the application of statistics to biological and
health-related data. This field bridges the gap between mathematical
theory and real-world medical research, helping us make sense of complex
biological phenomena and draw meaningful conclusions from data.

Whether you're a medical student beginning your research journey, a
healthcare professional seeking to understand clinical studies, or a
researcher diving into data analysis, this comprehensive guide will take
you from basic concepts to advanced methodologies used in cutting-edge
research.

\begin{tcolorbox}[enhanced jigsaw, breakable, toprule=.15mm, left=2mm, coltitle=black, colbacktitle=quarto-callout-note-color!10!white, opacityback=0, opacitybacktitle=0.6, arc=.35mm, titlerule=0mm, toptitle=1mm, colframe=quarto-callout-note-color-frame, colback=white, title=\textcolor{quarto-callout-note-color}{\faInfo}\hspace{0.5em}{Why Biostatistics Matters}, bottomtitle=1mm, leftrule=.75mm, rightrule=.15mm, bottomrule=.15mm]

Every major medical breakthrough---from vaccine development to treatment
protocols---relies on biostatistical methods. Understanding these
concepts enables you to:

\begin{itemize}
\tightlist
\item
  Critically evaluate medical literature
\item
  Design robust research studies
\item
  Analyze complex biological data
\item
  Make evidence-based decisions in healthcare
\end{itemize}

\end{tcolorbox}

This textbook is structured to provide a logical progression from
fundamental concepts to specialized applications. Each chapter builds
upon previous knowledge while remaining accessible to readers at
different levels of statistical background.

\bookmarksetup{startatroot}

\chapter{Foundational Concepts}\label{foundational-concepts}

Master the essential building blocks of statistical thinking that form
the foundation for all biostatistical analysis.

\section{Descriptive Statistics \&
Visualization}\label{descriptive-statistics-visualization}

Before diving into complex analyses, we must first understand how to
describe and visualize our data. Descriptive statistics provide the
first glimpse into what our data tells us.

\begin{tcolorbox}[enhanced jigsaw, breakable, toprule=.15mm, left=2mm, coltitle=black, colbacktitle=quarto-callout-note-color!10!white, opacityback=0, opacitybacktitle=0.6, arc=.35mm, titlerule=0mm, toptitle=1mm, colframe=quarto-callout-note-color-frame, colback=white, title=\textcolor{quarto-callout-note-color}{\faInfo}\hspace{0.5em}{Understanding Your Data}, bottomtitle=1mm, leftrule=.75mm, rightrule=.15mm, bottomrule=.15mm]

Every dataset tells a story. Descriptive statistics help us understand:

\begin{itemize}
\tightlist
\item
  \textbf{Central Tendency:} Where does the typical value lie?
\item
  \textbf{Variability:} How spread out are our observations?
\item
  \textbf{Shape:} Is the data symmetric or skewed?
\item
  \textbf{Outliers:} Are there unusual observations?
\end{itemize}

\end{tcolorbox}

\subsection{Measures of Central
Tendency}\label{measures-of-central-tendency}

\textbf{Mean (Average):} The arithmetic mean is the sum of all values
divided by the number of observations. While intuitive, it's sensitive
to extreme values.

\[\text{Mean} = \frac{\sum x}{n}\]

\begin{tcolorbox}[enhanced jigsaw, breakable, toprule=.15mm, left=2mm, coltitle=black, colbacktitle=quarto-callout-tip-color!10!white, opacityback=0, opacitybacktitle=0.6, arc=.35mm, titlerule=0mm, toptitle=1mm, colframe=quarto-callout-tip-color-frame, colback=white, title=\textcolor{quarto-callout-tip-color}{\faLightbulb}\hspace{0.5em}{Example: Blood Pressure Study}, bottomtitle=1mm, leftrule=.75mm, rightrule=.15mm, bottomrule=.15mm]

Consider systolic blood pressure measurements from 7 patients: 120, 125,
118, 140, 122, 135, 180

Mean = (120 + 125 + 118 + 140 + 122 + 135 + 180) / 7 = 134.3 mmHg

Notice how the outlier (180) pulls the mean upward.

\end{tcolorbox}

\textbf{Median:} The middle value when data is arranged in order. More
robust to outliers than the mean.

\textbf{Mode:} The most frequently occurring value. Useful for
categorical data or to identify peaks in distributions.

\subsection{Measures of Variability}\label{measures-of-variability}

\textbf{Range:} Simply the difference between maximum and minimum
values. Easy to calculate but affected by outliers.

\textbf{Standard Deviation:} Measures how much individual observations
deviate from the mean. This is perhaps the most important measure of
variability in statistics.

\[\text{Standard Deviation} = \sqrt{\frac{\sum(x - \text{mean})^2}{n-1}}\]

\begin{tcolorbox}[enhanced jigsaw, breakable, toprule=.15mm, left=2mm, coltitle=black, colbacktitle=quarto-callout-warning-color!10!white, opacityback=0, opacitybacktitle=0.6, arc=.35mm, titlerule=0mm, toptitle=1mm, colframe=quarto-callout-warning-color-frame, colback=white, title=\textcolor{quarto-callout-warning-color}{\faExclamationTriangle}\hspace{0.5em}{Common Mistake}, bottomtitle=1mm, leftrule=.75mm, rightrule=.15mm, bottomrule=.15mm]

Always use n-1 (not n) in the denominator when calculating sample
standard deviation. This provides an unbiased estimate of population
variability.

\end{tcolorbox}

\textbf{Variance:} The square of standard deviation. While less
intuitive (different units), it's mathematically convenient for many
statistical procedures.

\subsection{Data Visualization}\label{data-visualization}

Visual representation of data often reveals patterns invisible in raw
numbers:

\textbf{Histograms:} Show the distribution shape and identify skewness,
multiple peaks, or outliers.

\textbf{Box Plots:} Provide a five-number summary (minimum, Q1, median,
Q3, maximum) and clearly highlight outliers.

\begin{tcolorbox}[enhanced jigsaw, breakable, toprule=.15mm, left=2mm, coltitle=black, colbacktitle=quarto-callout-tip-color!10!white, opacityback=0, opacitybacktitle=0.6, arc=.35mm, titlerule=0mm, toptitle=1mm, colframe=quarto-callout-tip-color-frame, colback=white, title=\textcolor{quarto-callout-tip-color}{\faLightbulb}\hspace{0.5em}{Interpreting Box Plots in Medical Research}, bottomtitle=1mm, leftrule=.75mm, rightrule=.15mm, bottomrule=.15mm]

A box plot of recovery times after surgery might show:

\begin{itemize}
\tightlist
\item
  Median recovery: 7 days
\item
  Most patients recover between 5-10 days (IQR)
\item
  Some outliers taking 20+ days
\item
  Distribution skewed toward longer recovery times
\end{itemize}

\end{tcolorbox}

\textbf{Scatter Plots:} Essential for examining relationships between
two continuous variables.

\section{Probability \& Distributions}\label{probability-distributions}

Probability theory provides the mathematical foundation for statistical
inference. Understanding distributions helps us model real-world
phenomena and calculate the likelihood of different outcomes.

\begin{tcolorbox}[enhanced jigsaw, breakable, toprule=.15mm, left=2mm, coltitle=black, colbacktitle=quarto-callout-note-color!10!white, opacityback=0, opacitybacktitle=0.6, arc=.35mm, titlerule=0mm, toptitle=1mm, colframe=quarto-callout-note-color-frame, colback=white, title=\textcolor{quarto-callout-note-color}{\faInfo}\hspace{0.5em}{What is Probability?}, bottomtitle=1mm, leftrule=.75mm, rightrule=.15mm, bottomrule=.15mm]

Probability quantifies uncertainty. In medical research, we use
probability to:

\begin{itemize}
\tightlist
\item
  Assess the likelihood of treatment success
\item
  Calculate confidence in our estimates
\item
  Determine sample sizes needed for studies
\item
  Evaluate the strength of evidence against hypotheses
\end{itemize}

\end{tcolorbox}

\subsection{Basic Probability Rules}\label{basic-probability-rules}

\begin{enumerate}
\def\labelenumi{\arabic{enumi}.}
\tightlist
\item
  \textbf{Addition Rule:} P(A or B) = P(A) + P(B) - P(A and B)
\item
  \textbf{Multiplication Rule:} P(A and B) = P(A) × P(B\textbar A)
\item
  \textbf{Complement Rule:} P(not A) = 1 - P(A)
\end{enumerate}

\begin{tcolorbox}[enhanced jigsaw, breakable, toprule=.15mm, left=2mm, coltitle=black, colbacktitle=quarto-callout-tip-color!10!white, opacityback=0, opacitybacktitle=0.6, arc=.35mm, titlerule=0mm, toptitle=1mm, colframe=quarto-callout-tip-color-frame, colback=white, title=\textcolor{quarto-callout-tip-color}{\faLightbulb}\hspace{0.5em}{Medical Example: Disease Testing}, bottomtitle=1mm, leftrule=.75mm, rightrule=.15mm, bottomrule=.15mm]

Consider a diagnostic test for a rare disease:

\begin{itemize}
\tightlist
\item
  Disease prevalence: 1\% (P(Disease) = 0.01)
\item
  Test sensitivity: 95\% (P(Positive\textbar Disease) = 0.95)
\item
  Test specificity: 90\% (P(Negative\textbar No Disease) = 0.90)
\end{itemize}

What's the probability someone actually has the disease if they test
positive?

This requires Bayes' theorem and often surprises medical professionals!

\end{tcolorbox}

\subsection{The Normal Distribution}\label{the-normal-distribution}

The normal (Gaussian) distribution is the cornerstone of statistical
analysis. Many biological measurements naturally follow this bell-shaped
curve.

\begin{tcolorbox}[enhanced jigsaw, breakable, toprule=.15mm, left=2mm, coltitle=black, colbacktitle=quarto-callout-note-color!10!white, opacityback=0, opacitybacktitle=0.6, arc=.35mm, titlerule=0mm, toptitle=1mm, colframe=quarto-callout-note-color-frame, colback=white, title=\textcolor{quarto-callout-note-color}{\faInfo}\hspace{0.5em}{Properties of Normal Distribution}, bottomtitle=1mm, leftrule=.75mm, rightrule=.15mm, bottomrule=.15mm]

\begin{itemize}
\tightlist
\item
  Symmetric and bell-shaped
\item
  Mean = Median = Mode
\item
  68\% of data within 1 standard deviation
\item
  95\% of data within 2 standard deviations
\item
  99.7\% of data within 3 standard deviations
\end{itemize}

\end{tcolorbox}

Examples of normally distributed biological variables:

\begin{itemize}
\tightlist
\item
  Height and weight in populations
\item
  Blood pressure readings
\item
  IQ scores
\item
  Many laboratory test results
\end{itemize}

\subsection{The t-Distribution}\label{the-t-distribution}

When working with small samples (n \textless{} 30), the t-distribution
becomes crucial. It's similar to the normal distribution but with
heavier tails, accounting for additional uncertainty from small sample
sizes.

\begin{tcolorbox}[enhanced jigsaw, breakable, toprule=.15mm, left=2mm, coltitle=black, colbacktitle=quarto-callout-warning-color!10!white, opacityback=0, opacitybacktitle=0.6, arc=.35mm, titlerule=0mm, toptitle=1mm, colframe=quarto-callout-warning-color-frame, colback=white, title=\textcolor{quarto-callout-warning-color}{\faExclamationTriangle}\hspace{0.5em}{When to Use t vs Normal}, bottomtitle=1mm, leftrule=.75mm, rightrule=.15mm, bottomrule=.15mm]

Use t-distribution when:

\begin{itemize}
\tightlist
\item
  Sample size is small (n \textless{} 30)
\item
  Population standard deviation is unknown
\item
  Data is approximately normally distributed
\end{itemize}

\end{tcolorbox}

\section{Sampling \& Sampling
Distributions}\label{sampling-sampling-distributions}

Understanding how samples relate to populations is fundamental to
statistical inference. The Central Limit Theorem bridges individual
observations to population-level conclusions.

\begin{tcolorbox}[enhanced jigsaw, breakable, toprule=.15mm, left=2mm, coltitle=black, colbacktitle=quarto-callout-note-color!10!white, opacityback=0, opacitybacktitle=0.6, arc=.35mm, titlerule=0mm, toptitle=1mm, colframe=quarto-callout-note-color-frame, colback=white, title=\textcolor{quarto-callout-note-color}{\faInfo}\hspace{0.5em}{Population vs Sample}, bottomtitle=1mm, leftrule=.75mm, rightrule=.15mm, bottomrule=.15mm]

\textbf{Population:} All possible subjects of interest (e.g., all
patients with diabetes)

\textbf{Sample:} A subset of the population we actually observe (e.g.,
100 diabetic patients in our study)

We use sample statistics to estimate population parameters.

\end{tcolorbox}

\subsection{Sampling Methods}\label{sampling-methods}

\textbf{Simple Random Sampling:} Every member has equal chance of
selection.

\textbf{Stratified Sampling:} Population divided into groups (strata),
then random sampling within each group.

\begin{tcolorbox}[enhanced jigsaw, breakable, toprule=.15mm, left=2mm, coltitle=black, colbacktitle=quarto-callout-tip-color!10!white, opacityback=0, opacitybacktitle=0.6, arc=.35mm, titlerule=0mm, toptitle=1mm, colframe=quarto-callout-tip-color-frame, colback=white, title=\textcolor{quarto-callout-tip-color}{\faLightbulb}\hspace{0.5em}{Stratified Sampling in Clinical Trials}, bottomtitle=1mm, leftrule=.75mm, rightrule=.15mm, bottomrule=.15mm]

When studying a new cardiac medication, researchers might stratify by:

\begin{itemize}
\tightlist
\item
  Age groups (18-40, 41-65, 65+)
\item
  Gender
\item
  Severity of condition
\end{itemize}

This ensures adequate representation across important subgroups.

\end{tcolorbox}

\subsection{The Central Limit Theorem}\label{the-central-limit-theorem}

This fundamental theorem states that sample means will be approximately
normally distributed, regardless of the population distribution, as
sample size increases (typically n ≥ 30).

\begin{tcolorbox}[enhanced jigsaw, breakable, toprule=.15mm, left=2mm, coltitle=black, colbacktitle=quarto-callout-note-color!10!white, opacityback=0, opacitybacktitle=0.6, arc=.35mm, titlerule=0mm, toptitle=1mm, colframe=quarto-callout-note-color-frame, colback=white, title=\textcolor{quarto-callout-note-color}{\faInfo}\hspace{0.5em}{Implications of Central Limit Theorem}, bottomtitle=1mm, leftrule=.75mm, rightrule=.15mm, bottomrule=.15mm]

\begin{itemize}
\tightlist
\item
  We can use normal distribution for inference about means
\item
  Larger samples give more precise estimates
\item
  We can calculate confidence intervals and p-values
\item
  Many statistical tests are based on this principle
\end{itemize}

\end{tcolorbox}

\[\text{Standard Error of Mean} = \frac{\sigma}{\sqrt{n}}\]

The standard error quantifies how much sample means vary from the true
population mean.

\section{Point \& Interval Estimation}\label{point-interval-estimation}

Estimation allows us to use sample data to make educated guesses about
population parameters, with quantified uncertainty.

\subsection{Point Estimation}\label{point-estimation}

A point estimate is a single value that serves as our ``best guess'' for
a population parameter.

\begin{tcolorbox}[enhanced jigsaw, breakable, toprule=.15mm, left=2mm, coltitle=black, colbacktitle=quarto-callout-tip-color!10!white, opacityback=0, opacitybacktitle=0.6, arc=.35mm, titlerule=0mm, toptitle=1mm, colframe=quarto-callout-tip-color-frame, colback=white, title=\textcolor{quarto-callout-tip-color}{\faLightbulb}\hspace{0.5em}{Point Estimates in Medicine}, bottomtitle=1mm, leftrule=.75mm, rightrule=.15mm, bottomrule=.15mm]

\begin{itemize}
\tightlist
\item
  Sample mean blood pressure → Population mean blood pressure
\item
  Sample proportion cured → Population cure rate
\item
  Sample correlation → Population correlation
\end{itemize}

\end{tcolorbox}

\subsection{Interval Estimation (Confidence
Intervals)}\label{interval-estimation-confidence-intervals}

While point estimates provide a single value, confidence intervals
provide a range of plausible values for the population parameter.

\begin{tcolorbox}[enhanced jigsaw, breakable, toprule=.15mm, left=2mm, coltitle=black, colbacktitle=quarto-callout-note-color!10!white, opacityback=0, opacitybacktitle=0.6, arc=.35mm, titlerule=0mm, toptitle=1mm, colframe=quarto-callout-note-color-frame, colback=white, title=\textcolor{quarto-callout-note-color}{\faInfo}\hspace{0.5em}{Interpreting Confidence Intervals}, bottomtitle=1mm, leftrule=.75mm, rightrule=.15mm, bottomrule=.15mm]

A 95\% confidence interval means:

``If we repeated this study 100 times, approximately 95 of the resulting
confidence intervals would contain the true population parameter.''

\end{tcolorbox}

\[95\% \text{ CI for mean} = \bar{x} \pm t_{(0.025, df)} \times \frac{s}{\sqrt{n}}\]

\begin{tcolorbox}[enhanced jigsaw, breakable, toprule=.15mm, left=2mm, coltitle=black, colbacktitle=quarto-callout-tip-color!10!white, opacityback=0, opacitybacktitle=0.6, arc=.35mm, titlerule=0mm, toptitle=1mm, colframe=quarto-callout-tip-color-frame, colback=white, title=\textcolor{quarto-callout-tip-color}{\faLightbulb}\hspace{0.5em}{Clinical Example}, bottomtitle=1mm, leftrule=.75mm, rightrule=.15mm, bottomrule=.15mm]

A new pain medication reduces pain scores by an average of 3.2 points
(95\% CI: 2.1 to 4.3).

\textbf{Interpretation:} We're 95\% confident the true average reduction
is between 2.1 and 4.3 points.

\end{tcolorbox}

\begin{tcolorbox}[enhanced jigsaw, breakable, toprule=.15mm, left=2mm, coltitle=black, colbacktitle=quarto-callout-warning-color!10!white, opacityback=0, opacitybacktitle=0.6, arc=.35mm, titlerule=0mm, toptitle=1mm, colframe=quarto-callout-warning-color-frame, colback=white, title=\textcolor{quarto-callout-warning-color}{\faExclamationTriangle}\hspace{0.5em}{Common Misinterpretation}, bottomtitle=1mm, leftrule=.75mm, rightrule=.15mm, bottomrule=.15mm]

❌ ``There's a 95\% chance the true mean is in this interval''

✅ ``95\% of such intervals would contain the true mean''

\end{tcolorbox}

\section{Hypothesis Testing
Framework}\label{hypothesis-testing-framework}

Hypothesis testing provides a structured approach to making decisions
under uncertainty, fundamental to scientific research.

\begin{tcolorbox}[enhanced jigsaw, breakable, toprule=.15mm, left=2mm, coltitle=black, colbacktitle=quarto-callout-note-color!10!white, opacityback=0, opacitybacktitle=0.6, arc=.35mm, titlerule=0mm, toptitle=1mm, colframe=quarto-callout-note-color-frame, colback=white, title=\textcolor{quarto-callout-note-color}{\faInfo}\hspace{0.5em}{The Logic of Hypothesis Testing}, bottomtitle=1mm, leftrule=.75mm, rightrule=.15mm, bottomrule=.15mm]

We start with a skeptical position (null hypothesis) and ask: ``Is our
observed data surprising enough to abandon this position?''

\end{tcolorbox}

\subsection{Setting Up Hypotheses}\label{setting-up-hypotheses}

\textbf{Null Hypothesis (H₀):} The ``no effect'' or ``status quo''
hypothesis. What we assume is true until proven otherwise.

\textbf{Alternative Hypothesis (H₁ or Hₐ):} The research hypothesis
we're trying to establish.

\begin{tcolorbox}[enhanced jigsaw, breakable, toprule=.15mm, left=2mm, coltitle=black, colbacktitle=quarto-callout-tip-color!10!white, opacityback=0, opacitybacktitle=0.6, arc=.35mm, titlerule=0mm, toptitle=1mm, colframe=quarto-callout-tip-color-frame, colback=white, title=\textcolor{quarto-callout-tip-color}{\faLightbulb}\hspace{0.5em}{Hypothesis Example: New Treatment}, bottomtitle=1mm, leftrule=.75mm, rightrule=.15mm, bottomrule=.15mm]

Research question: ``Does new drug X reduce blood pressure more than
placebo?''

\begin{itemize}
\tightlist
\item
  \textbf{H₀:} Drug X has no effect (μ\_drug = μ\_placebo)
\item
  \textbf{H₁:} Drug X reduces blood pressure (μ\_drug \textless{}
  μ\_placebo)
\end{itemize}

\end{tcolorbox}

\subsection{Types of Errors in Statistical
Testing}\label{types-of-errors-in-statistical-testing}

Statistical testing is not perfect---we can make two types of errors
when making decisions about hypotheses. Understanding these errors is
crucial for interpreting research results and making informed decisions.

\begin{tcolorbox}[enhanced jigsaw, breakable, toprule=.15mm, left=2mm, coltitle=black, colbacktitle=quarto-callout-note-color!10!white, opacityback=0, opacitybacktitle=0.6, arc=.35mm, titlerule=0mm, toptitle=1mm, colframe=quarto-callout-note-color-frame, colback=white, title=\textcolor{quarto-callout-note-color}{\faInfo}\hspace{0.5em}{The Truth Table of Statistical Decisions}, bottomtitle=1mm, leftrule=.75mm, rightrule=.15mm, bottomrule=.15mm]

When conducting a hypothesis test, there are four possible outcomes:

\begin{longtable}[]{@{}lll@{}}
\toprule\noalign{}
& H₀ is True & H₀ is False \\
\midrule\noalign{}
\endhead
\bottomrule\noalign{}
\endlastfoot
Reject H₀ & Type I Error (α) & Correct Decision \\
Fail to Reject H₀ & Correct Decision & Type II Error (β) \\
\end{longtable}

\end{tcolorbox}

\subsection{Type I Error (α): False
Positive}\label{type-i-error-ux3b1-false-positive}

\textbf{Definition:} Rejecting a true null hypothesis---concluding there
is an effect when there really isn't one.

\textbf{Probability:} The significance level (α) represents the maximum
Type I error rate we're willing to accept, typically set at 0.05 (5\%).

\begin{tcolorbox}[enhanced jigsaw, breakable, toprule=.15mm, left=2mm, coltitle=black, colbacktitle=quarto-callout-tip-color!10!white, opacityback=0, opacitybacktitle=0.6, arc=.35mm, titlerule=0mm, toptitle=1mm, colframe=quarto-callout-tip-color-frame, colback=white, title=\textcolor{quarto-callout-tip-color}{\faLightbulb}\hspace{0.5em}{Type I Error Examples}, bottomtitle=1mm, leftrule=.75mm, rightrule=.15mm, bottomrule=.15mm]

\textbf{Example 1: Drug Testing} - \textbf{H₀:} New drug has no effect
(same as placebo) - \textbf{Truth:} Drug actually has no effect -
\textbf{Type I Error:} Study concludes drug is effective -
\textbf{Consequences:} Ineffective drug approved, wasting resources and
potentially exposing patients to unnecessary side effects

\textbf{Example 2: Diagnostic Testing} - \textbf{H₀:} Patient does not
have disease - \textbf{Truth:} Patient is actually healthy -
\textbf{Type I Error:} Test indicates patient has disease (false
positive) - \textbf{Consequences:} Unnecessary anxiety, additional
testing, and potentially harmful treatments

\textbf{Example 3: Quality Control} - \textbf{H₀:} Manufacturing process
is working correctly - \textbf{Truth:} Process is actually working fine
- \textbf{Type I Error:} Conclude process is faulty -
\textbf{Consequences:} Unnecessary production shutdown, wasted time and
money investigating non-existent problems

\end{tcolorbox}

\begin{tcolorbox}[enhanced jigsaw, breakable, toprule=.15mm, left=2mm, coltitle=black, colbacktitle=quarto-callout-warning-color!10!white, opacityback=0, opacitybacktitle=0.6, arc=.35mm, titlerule=0mm, toptitle=1mm, colframe=quarto-callout-warning-color-frame, colback=white, title=\textcolor{quarto-callout-warning-color}{\faExclamationTriangle}\hspace{0.5em}{Controlling Type I Error}, bottomtitle=1mm, leftrule=.75mm, rightrule=.15mm, bottomrule=.15mm]

We control Type I error by setting the significance level (α) before
conducting our test:

\begin{itemize}
\tightlist
\item
  \textbf{α = 0.05:} 5\% chance of Type I error (most common)
\item
  \textbf{α = 0.01:} 1\% chance of Type I error (more stringent)
\item
  \textbf{α = 0.10:} 10\% chance of Type I error (more lenient)
\end{itemize}

\emph{Lowering α reduces Type I errors but increases Type II errors.}

\end{tcolorbox}

\subsection{Type II Error (β): False
Negative}\label{type-ii-error-ux3b2-false-negative}

\textbf{Definition:} Failing to reject a false null hypothesis---missing
a real effect that actually exists.

\textbf{Statistical Power:} Power = 1 - β, represents the probability of
correctly detecting an effect when it truly exists.

\begin{tcolorbox}[enhanced jigsaw, breakable, toprule=.15mm, left=2mm, coltitle=black, colbacktitle=quarto-callout-tip-color!10!white, opacityback=0, opacitybacktitle=0.6, arc=.35mm, titlerule=0mm, toptitle=1mm, colframe=quarto-callout-tip-color-frame, colback=white, title=\textcolor{quarto-callout-tip-color}{\faLightbulb}\hspace{0.5em}{Type II Error Examples}, bottomtitle=1mm, leftrule=.75mm, rightrule=.15mm, bottomrule=.15mm]

\textbf{Example 1: Drug Testing} - \textbf{H₀:} New drug has no effect -
\textbf{Truth:} Drug is actually effective - \textbf{Type II Error:}
Study fails to detect drug's effectiveness - \textbf{Consequences:}
Effective treatment not approved, patients continue suffering from
treatable condition

\textbf{Example 2: Diagnostic Testing} - \textbf{H₀:} Patient does not
have disease - \textbf{Truth:} Patient actually has the disease -
\textbf{Type II Error:} Test fails to detect disease (false negative) -
\textbf{Consequences:} Delayed treatment, disease progression,
potentially fatal outcomes

\textbf{Example 3: Environmental Monitoring} - \textbf{H₀:} Pollution
levels are safe - \textbf{Truth:} Pollution levels are actually
dangerous - \textbf{Type II Error:} Fail to detect dangerous pollution -
\textbf{Consequences:} Continued exposure to harmful substances, public
health risks

\end{tcolorbox}

\begin{tcolorbox}[enhanced jigsaw, breakable, toprule=.15mm, left=2mm, coltitle=black, colbacktitle=quarto-callout-note-color!10!white, opacityback=0, opacitybacktitle=0.6, arc=.35mm, titlerule=0mm, toptitle=1mm, colframe=quarto-callout-note-color-frame, colback=white, title=\textcolor{quarto-callout-note-color}{\faInfo}\hspace{0.5em}{Factors Affecting Type II Error (β)}, bottomtitle=1mm, leftrule=.75mm, rightrule=.15mm, bottomrule=.15mm]

Several factors influence the probability of Type II error:

\begin{itemize}
\tightlist
\item
  \textbf{Effect Size:} Larger true effects are easier to detect (lower
  β)
\item
  \textbf{Sample Size:} Larger samples reduce β
\item
  \textbf{Significance Level (α):} Higher α reduces β
\item
  \textbf{Variability:} Less noisy data reduces β
\item
  \textbf{Study Design:} Better designs reduce β
\end{itemize}

\end{tcolorbox}

\subsection{Real-World Clinical Example: COVID-19
Testing}\label{real-world-clinical-example-covid-19-testing}

\begin{tcolorbox}[enhanced jigsaw, breakable, toprule=.15mm, left=2mm, coltitle=black, colbacktitle=quarto-callout-tip-color!10!white, opacityback=0, opacitybacktitle=0.6, arc=.35mm, titlerule=0mm, toptitle=1mm, colframe=quarto-callout-tip-color-frame, colback=white, title=\textcolor{quarto-callout-tip-color}{\faLightbulb}\hspace{0.5em}{Understanding Both Error Types in Practice}, bottomtitle=1mm, leftrule=.75mm, rightrule=.15mm, bottomrule=.15mm]

\textbf{Scenario:} Testing for COVID-19 infection \textbf{H₀:} Person is
not infected with COVID-19 \textbf{H₁:} Person is infected with COVID-19

\textbf{Type I Error (False Positive):} - Test says ``infected'' when
person is actually healthy - Rate: Depends on test specificity (1 -
specificity) - Consequences: Unnecessary quarantine, contact tracing,
anxiety - Public health impact: Resource waste, reduced public
confidence

\textbf{Type II Error (False Negative):} - Test says ``not infected''
when person is actually infected - Rate: Depends on test sensitivity (1
- sensitivity) - Consequences: Continued spread, no treatment, false
security - Public health impact: Disease transmission, outbreak
expansion

\textbf{The Trade-off:} During a pandemic, false negatives (Type II
errors) are often considered more dangerous than false positives (Type I
errors) because missing infected individuals leads to continued disease
spread.

\end{tcolorbox}

\subsection{The Relationship Between Error
Types}\label{the-relationship-between-error-types}

\begin{tcolorbox}[enhanced jigsaw, breakable, toprule=.15mm, left=2mm, coltitle=black, colbacktitle=quarto-callout-note-color!10!white, opacityback=0, opacitybacktitle=0.6, arc=.35mm, titlerule=0mm, toptitle=1mm, colframe=quarto-callout-note-color-frame, colback=white, title=\textcolor{quarto-callout-note-color}{\faInfo}\hspace{0.5em}{Key Insights About Statistical Errors}, bottomtitle=1mm, leftrule=.75mm, rightrule=.15mm, bottomrule=.15mm]

\begin{itemize}
\tightlist
\item
  \textbf{Inverse Relationship:} Generally, as Type I error decreases,
  Type II error increases
\item
  \textbf{Sample Size Effect:} Larger samples reduce both types of
  errors
\item
  \textbf{Effect Size Matters:} Larger true effects make Type II errors
  less likely
\item
  \textbf{Context Dependent:} The relative cost of each error type
  varies by situation
\item
  \textbf{Power Analysis:} Helps balance these errors when designing
  studies
\end{itemize}

\end{tcolorbox}

\textbf{Key Relationships:} - Power = 1 - β - P(Type I Error) = α -
P(Type II Error) = β - As n ↑, both α and β ↓ (for fixed effect size)

\subsection{Practical Guidelines for
Researchers}\label{practical-guidelines-for-researchers}

\begin{tcolorbox}[enhanced jigsaw, breakable, toprule=.15mm, left=2mm, coltitle=black, colbacktitle=quarto-callout-warning-color!10!white, opacityback=0, opacitybacktitle=0.6, arc=.35mm, titlerule=0mm, toptitle=1mm, colframe=quarto-callout-warning-color-frame, colback=white, title=\textcolor{quarto-callout-warning-color}{\faExclamationTriangle}\hspace{0.5em}{Choosing Error Tolerances}, bottomtitle=1mm, leftrule=.75mm, rightrule=.15mm, bottomrule=.15mm]

\textbf{When Type I errors are more serious:} - Drug approval studies
(don't approve ineffective drugs) - Diagnostic tests for serious
conditions - Use lower α (0.01 or 0.001)

\textbf{When Type II errors are more serious:} - Screening for treatable
conditions - Safety monitoring studies - Use higher α (0.10) or ensure
high power (0.90+)

\end{tcolorbox}

\begin{tcolorbox}[enhanced jigsaw, breakable, toprule=.15mm, left=2mm, coltitle=black, colbacktitle=quarto-callout-note-color!10!white, opacityback=0, opacitybacktitle=0.6, arc=.35mm, titlerule=0mm, toptitle=1mm, colframe=quarto-callout-note-color-frame, colback=white, title=\textcolor{quarto-callout-note-color}{\faInfo}\hspace{0.5em}{Medical Context Summary}, bottomtitle=1mm, leftrule=.75mm, rightrule=.15mm, bottomrule=.15mm]

\textbf{Type I Error (False Positive):} Concluding a treatment works
when it doesn't, a diagnostic test is positive when disease is absent,
or an intervention has an effect when it doesn't.

\textbf{Type II Error (False Negative):} Failing to detect a treatment
that actually works, missing a disease that is present, or failing to
identify a real intervention effect.

\textbf{Clinical Impact:} Both errors have serious consequences in
healthcare---unnecessary treatments and missed diagnoses can both harm
patients.

\end{tcolorbox}

\subsection{P-values}\label{p-values}

The p-value answers: ``If the null hypothesis were true, what's the
probability of observing data as extreme or more extreme than what we
actually observed?''

\begin{tcolorbox}[enhanced jigsaw, breakable, toprule=.15mm, left=2mm, coltitle=black, colbacktitle=quarto-callout-warning-color!10!white, opacityback=0, opacitybacktitle=0.6, arc=.35mm, titlerule=0mm, toptitle=1mm, colframe=quarto-callout-warning-color-frame, colback=white, title=\textcolor{quarto-callout-warning-color}{\faExclamationTriangle}\hspace{0.5em}{P-value Misconceptions}, bottomtitle=1mm, leftrule=.75mm, rightrule=.15mm, bottomrule=.15mm]

❌ ``P-value is the probability the null hypothesis is true''

❌ ``P-value is the probability of making a mistake''

✅ ``P-value is the probability of observing such data if null
hypothesis is true''

\end{tcolorbox}

\textbf{Statistical Significance:} Typically, p \textless{} 0.05 is
considered statistically significant, meaning we reject the null
hypothesis.

\section{One-Sample \& Two-Sample
t-Tests}\label{one-sample-two-sample-t-tests}

T-tests are among the most commonly used statistical tests in biomedical
research, allowing us to compare means between groups or against known
values.

\subsection{One-Sample t-Test}\label{one-sample-t-test}

Compares a sample mean to a known population value or theoretical
expectation.

\begin{tcolorbox}[enhanced jigsaw, breakable, toprule=.15mm, left=2mm, coltitle=black, colbacktitle=quarto-callout-tip-color!10!white, opacityback=0, opacitybacktitle=0.6, arc=.35mm, titlerule=0mm, toptitle=1mm, colframe=quarto-callout-tip-color-frame, colback=white, title=\textcolor{quarto-callout-tip-color}{\faLightbulb}\hspace{0.5em}{One-Sample Example}, bottomtitle=1mm, leftrule=.75mm, rightrule=.15mm, bottomrule=.15mm]

Normal body temperature is supposedly 98.6°F. We measure 25 healthy
adults and want to test if the population mean differs from 98.6°F.

\begin{itemize}
\tightlist
\item
  \textbf{H₀:} μ = 98.6°F
\item
  \textbf{H₁:} μ ≠ 98.6°F
\end{itemize}

\end{tcolorbox}

\[t = \frac{\bar{x} - \mu_0}{s/\sqrt{n}}\] \[df = n - 1\]

\subsection{Two-Sample t-Test (Equal
Variances)}\label{two-sample-t-test-equal-variances}

Compares means between two independent groups, assuming equal population
variances.

\begin{tcolorbox}[enhanced jigsaw, breakable, toprule=.15mm, left=2mm, coltitle=black, colbacktitle=quarto-callout-note-color!10!white, opacityback=0, opacitybacktitle=0.6, arc=.35mm, titlerule=0mm, toptitle=1mm, colframe=quarto-callout-note-color-frame, colback=white, title=\textcolor{quarto-callout-note-color}{\faInfo}\hspace{0.5em}{Assumptions for Two-Sample t-Test}, bottomtitle=1mm, leftrule=.75mm, rightrule=.15mm, bottomrule=.15mm]

\begin{itemize}
\tightlist
\item
  Both samples from normal distributions
\item
  Independent observations
\item
  Equal population variances (homoscedasticity)
\end{itemize}

\end{tcolorbox}

The \textbf{pooled standard deviation} combines information from both
samples:

\[s_{pooled} = \sqrt{\frac{(n_1-1)s_1^2 + (n_2-1)s_2^2}{n_1+n_2-2}}\]

\[t = \frac{\bar{x}_1 - \bar{x}_2}{s_{pooled} \times \sqrt{\frac{1}{n_1} + \frac{1}{n_2}}}\]

\begin{tcolorbox}[enhanced jigsaw, breakable, toprule=.15mm, left=2mm, coltitle=black, colbacktitle=quarto-callout-tip-color!10!white, opacityback=0, opacitybacktitle=0.6, arc=.35mm, titlerule=0mm, toptitle=1mm, colframe=quarto-callout-tip-color-frame, colback=white, title=\textcolor{quarto-callout-tip-color}{\faLightbulb}\hspace{0.5em}{Clinical Trial Example}, bottomtitle=1mm, leftrule=.75mm, rightrule=.15mm, bottomrule=.15mm]

Comparing pain scores between treatment and control groups:

\begin{itemize}
\tightlist
\item
  \textbf{Treatment group:} n=20, mean=4.2, sd=1.8
\item
  \textbf{Control group:} n=18, mean=6.1, sd=2.1
\item
  \textbf{Question:} Does treatment reduce pain scores?
\end{itemize}

\end{tcolorbox}

\subsection{When to Use Pooled vs Unpooled (Welch's)
t-Test}\label{when-to-use-pooled-vs-unpooled-welchs-t-test}

\begin{tcolorbox}[enhanced jigsaw, breakable, toprule=.15mm, left=2mm, coltitle=black, colbacktitle=quarto-callout-warning-color!10!white, opacityback=0, opacitybacktitle=0.6, arc=.35mm, titlerule=0mm, toptitle=1mm, colframe=quarto-callout-warning-color-frame, colback=white, title=\textcolor{quarto-callout-warning-color}{\faExclamationTriangle}\hspace{0.5em}{Testing Equal Variances}, bottomtitle=1mm, leftrule=.75mm, rightrule=.15mm, bottomrule=.15mm]

Use F-test or Levene's test to check equal variance assumption. If p
\textgreater{} 0.05, variances are likely equal and pooled t-test is
appropriate.

\end{tcolorbox}

The pooled t-test is more powerful when variances are truly equal, but
Welch's t-test is more robust when they're not.

\bookmarksetup{startatroot}

\chapter{Comparative Inference}\label{comparative-inference}

Advanced techniques for comparing groups when traditional assumptions
don't hold, ensuring robust statistical inference across diverse
research scenarios.

\section{Welch's t-Test \& Paired
t-Test}\label{welchs-t-test-paired-t-test}

When the equal variance assumption is violated or when data comes in
natural pairs, we need specialized approaches to maintain valid
statistical inference.

\subsection{Welch's t-Test (Unequal
Variances)}\label{welchs-t-test-unequal-variances}

When group variances differ substantially, the pooled t-test can give
misleading results. Welch's t-test adjusts for unequal variances.

\[t = \frac{\bar{x}_1 - \bar{x}_2}{\sqrt{\frac{s_1^2}{n_1} + \frac{s_2^2}{n_2}}}\]

\[df = \frac{\left(\frac{s_1^2}{n_1} + \frac{s_2^2}{n_2}\right)^2}{\frac{\left(\frac{s_1^2}{n_1}\right)^2}{n_1-1} + \frac{\left(\frac{s_2^2}{n_2}\right)^2}{n_2-1}}\]

\begin{tcolorbox}[enhanced jigsaw, breakable, toprule=.15mm, left=2mm, coltitle=black, colbacktitle=quarto-callout-tip-color!10!white, opacityback=0, opacitybacktitle=0.6, arc=.35mm, titlerule=0mm, toptitle=1mm, colframe=quarto-callout-tip-color-frame, colback=white, title=\textcolor{quarto-callout-tip-color}{\faLightbulb}\hspace{0.5em}{When Variances Differ}, bottomtitle=1mm, leftrule=.75mm, rightrule=.15mm, bottomrule=.15mm]

Comparing recovery times between two surgical procedures:

\begin{itemize}
\tightlist
\item
  \textbf{Procedure A:} mean=7 days, sd=2 days (low variability)
\item
  \textbf{Procedure B:} mean=8 days, sd=5 days (high variability)
\end{itemize}

The large difference in standard deviations suggests using Welch's
t-test.

\end{tcolorbox}

\emph{Note: This chapter appears to be incomplete in the source HTML
file. Additional content for paired t-tests and other comparative
inference methods would typically be included here.}

\bookmarksetup{startatroot}

\chapter{Multi-Group Analysis}\label{multi-group-analysis}

When research involves comparing three or more groups, specialized
techniques prevent inflated error rates while maintaining statistical
power.

\section{Introduction to ANOVA (Analysis of
Variance)}\label{introduction-to-anova-analysis-of-variance}

When comparing means across three or more groups, using multiple t-tests
creates a serious statistical problem: \textbf{multiple comparisons}.
ANOVA provides an elegant solution by testing all groups simultaneously
while controlling error rates.

\begin{tcolorbox}[enhanced jigsaw, breakable, toprule=.15mm, left=2mm, coltitle=black, colbacktitle=quarto-callout-warning-color!10!white, opacityback=0, opacitybacktitle=0.6, arc=.35mm, titlerule=0mm, toptitle=1mm, colframe=quarto-callout-warning-color-frame, colback=white, title=\textcolor{quarto-callout-warning-color}{\faExclamationTriangle}\hspace{0.5em}{Why Not Multiple t-Tests?}, bottomtitle=1mm, leftrule=.75mm, rightrule=.15mm, bottomrule=.15mm]

With 3 groups, you'd need 3 t-tests. With α = 0.05 for each test:

\begin{itemize}
\tightlist
\item
  \textbf{Individual error rate:} 5\% per test
\item
  \textbf{Family-wise error rate:} 1 - (0.95)³ = 14.3\%
\item
  \textbf{Problem:} Much higher chance of false positives!
\end{itemize}

With 5 groups, you'd need 10 tests, inflating error rate to
\textasciitilde40\%!

\end{tcolorbox}

\begin{tcolorbox}[enhanced jigsaw, breakable, toprule=.15mm, left=2mm, coltitle=black, colbacktitle=quarto-callout-note-color!10!white, opacityback=0, opacitybacktitle=0.6, arc=.35mm, titlerule=0mm, toptitle=1mm, colframe=quarto-callout-note-color-frame, colback=white, title=\textcolor{quarto-callout-note-color}{\faInfo}\hspace{0.5em}{The Logic of ANOVA}, bottomtitle=1mm, leftrule=.75mm, rightrule=.15mm, bottomrule=.15mm]

ANOVA compares two sources of variation:

\begin{itemize}
\tightlist
\item
  \textbf{Between-groups variation:} Differences due to treatment
  effects
\item
  \textbf{Within-groups variation:} Random variation (noise)
\end{itemize}

\textbf{Key Insight:} If treatment has no effect, both sources should be
similar. If there's a real effect, between-groups variation will be much
larger.

\end{tcolorbox}

\subsection{ANOVA Hypotheses}\label{anova-hypotheses}

\textbf{Null Hypothesis (H₀):} All group means are equal H₀: μ₁ = μ₂ =
μ₃ = \ldots{} = μₖ

\textbf{Alternative Hypothesis (H₁):} At least one group mean differs
H₁: Not all μᵢ are equal

\begin{tcolorbox}[enhanced jigsaw, breakable, toprule=.15mm, left=2mm, coltitle=black, colbacktitle=quarto-callout-tip-color!10!white, opacityback=0, opacitybacktitle=0.6, arc=.35mm, titlerule=0mm, toptitle=1mm, colframe=quarto-callout-tip-color-frame, colback=white, title=\textcolor{quarto-callout-tip-color}{\faLightbulb}\hspace{0.5em}{Medical Research Example}, bottomtitle=1mm, leftrule=.75mm, rightrule=.15mm, bottomrule=.15mm]

\textbf{Research Question:} Do three different pain medications have
different effectiveness?

\begin{itemize}
\tightlist
\item
  \textbf{Group 1:} Medication A (n=25)
\item
  \textbf{Group 2:} Medication B (n=23)
\item
  \textbf{Group 3:} Placebo (n=27)
\item
  \textbf{Outcome:} Pain reduction score (0-10 scale)
\end{itemize}

\textbf{H₀:} μₐ = μᵦ = μₚₗₐᶜₑᵦₒ (all medications equally effective)
\textbf{H₁:} At least one medication differs in effectiveness

\end{tcolorbox}

\subsection{The F-Statistic}\label{the-f-statistic}

ANOVA uses the F-statistic to compare variances:

\[F = \frac{\text{MSB}}{\text{MSW}}\]

where: - MSB = Mean Square Between groups - MSW = Mean Square Within
groups

\begin{tcolorbox}[enhanced jigsaw, breakable, toprule=.15mm, left=2mm, coltitle=black, colbacktitle=quarto-callout-note-color!10!white, opacityback=0, opacitybacktitle=0.6, arc=.35mm, titlerule=0mm, toptitle=1mm, colframe=quarto-callout-note-color-frame, colback=white, title=\textcolor{quarto-callout-note-color}{\faInfo}\hspace{0.5em}{Understanding the F-Ratio}, bottomtitle=1mm, leftrule=.75mm, rightrule=.15mm, bottomrule=.15mm]

\begin{itemize}
\tightlist
\item
  \textbf{F ≈ 1:} Between-group variation similar to within-group
  variation → No treatment effect
\item
  \textbf{F \textgreater\textgreater{} 1:} Between-group variation much
  larger → Likely treatment effect
\item
  \textbf{F \textless{} 1:} Rare, suggests less variation between groups
  than within (unusual)
\end{itemize}

\end{tcolorbox}

\subsection{ANOVA Assumptions}\label{anova-assumptions}

\begin{tcolorbox}[enhanced jigsaw, breakable, toprule=.15mm, left=2mm, coltitle=black, colbacktitle=quarto-callout-warning-color!10!white, opacityback=0, opacitybacktitle=0.6, arc=.35mm, titlerule=0mm, toptitle=1mm, colframe=quarto-callout-warning-color-frame, colback=white, title=\textcolor{quarto-callout-warning-color}{\faExclamationTriangle}\hspace{0.5em}{Critical Assumptions}, bottomtitle=1mm, leftrule=.75mm, rightrule=.15mm, bottomrule=.15mm]

\begin{itemize}
\tightlist
\item
  \textbf{Independence:} Observations must be independent
\item
  \textbf{Normality:} Data in each group should be approximately normal
\item
  \textbf{Homoscedasticity:} Equal variances across all groups
\end{itemize}

\textbf{Violation Consequences:} Increased Type I error, reduced power,
invalid conclusions

\end{tcolorbox}

\begin{tcolorbox}[enhanced jigsaw, breakable, toprule=.15mm, left=2mm, coltitle=black, colbacktitle=quarto-callout-tip-color!10!white, opacityback=0, opacitybacktitle=0.6, arc=.35mm, titlerule=0mm, toptitle=1mm, colframe=quarto-callout-tip-color-frame, colback=white, title=\textcolor{quarto-callout-tip-color}{\faLightbulb}\hspace{0.5em}{Checking Assumptions in Practice}, bottomtitle=1mm, leftrule=.75mm, rightrule=.15mm, bottomrule=.15mm]

\textbf{Independence:} Ensure proper randomization and no clustering

\textbf{Normality:} - Visual: Q-Q plots, histograms for each group -
Statistical: Shapiro-Wilk test (if n \textless{} 50) - Robust: ANOVA
reasonably robust with n \textgreater{} 30 per group

\textbf{Equal Variances:} - Visual: Box plots, side-by-side -
Statistical: Levene's test (preferred), Bartlett's test - Rule of thumb:
Largest SD / Smallest SD \textless{} 2

\end{tcolorbox}

\section{One-Way ANOVA: Step-by-Step}\label{one-way-anova-step-by-step}

One-way ANOVA compares means across groups defined by a single
categorical variable (factor).

\subsection{ANOVA Table Components}\label{anova-table-components}

\begin{tcolorbox}[enhanced jigsaw, breakable, toprule=.15mm, left=2mm, coltitle=black, colbacktitle=quarto-callout-note-color!10!white, opacityback=0, opacitybacktitle=0.6, arc=.35mm, titlerule=0mm, toptitle=1mm, colframe=quarto-callout-note-color-frame, colback=white, title=\textcolor{quarto-callout-note-color}{\faInfo}\hspace{0.5em}{Understanding ANOVA Table}, bottomtitle=1mm, leftrule=.75mm, rightrule=.15mm, bottomrule=.15mm]

\begin{longtable}[]{@{}lllll@{}}
\toprule\noalign{}
Source & df & SS & MS & F \\
\midrule\noalign{}
\endhead
\bottomrule\noalign{}
\endlastfoot
Between & k-1 & SSB & MSB & MSB/MSW \\
Within & N-k & SSW & MSW & - \\
Total & N-1 & SST & - & - \\
\end{longtable}

\textbf{Where:} k = number of groups, N = total sample size

\end{tcolorbox}

\subsection{Calculating Sum of
Squares}\label{calculating-sum-of-squares}

\textbf{Total Sum of Squares (SST):}
\[\text{SST} = \sum_i(x_i - \bar{x})^2\]

\textbf{Between Groups Sum of Squares (SSB):}
\[\text{SSB} = \sum_j n_j(\bar{x}_j - \bar{x})^2\]

\textbf{Within Groups Sum of Squares (SSW):}
\[\text{SSW} = \text{SST} - \text{SSB}\]

\begin{tcolorbox}[enhanced jigsaw, breakable, toprule=.15mm, left=2mm, coltitle=black, colbacktitle=quarto-callout-tip-color!10!white, opacityback=0, opacitybacktitle=0.6, arc=.35mm, titlerule=0mm, toptitle=1mm, colframe=quarto-callout-tip-color-frame, colback=white, title=\textcolor{quarto-callout-tip-color}{\faLightbulb}\hspace{0.5em}{Worked Example: Pain Medication Study}, bottomtitle=1mm, leftrule=.75mm, rightrule=.15mm, bottomrule=.15mm]

\textbf{Data:} - \textbf{Medication A:} 7, 8, 6, 9, 7 (n₁=5, x̄₁=7.4) -
\textbf{Medication B:} 5, 6, 4, 7, 5 (n₂=5, x̄₂=5.4) - \textbf{Placebo:}
3, 4, 2, 5, 3 (n₃=5, x̄₃=3.4)

\textbf{Step 1: Calculate overall mean} x̄ = (7.4×5 + 5.4×5 + 3.4×5) / 15
= 81/15 = 5.4

\textbf{Step 2: Calculate SSB} SSB = 5×(7.4-5.4)² + 5×(5.4-5.4)² +
5×(3.4-5.4)² SSB = 5×4 + 5×0 + 5×4 = 40

\textbf{Step 3: Calculate SST} SST = (7-5.4)² + (8-5.4)² + \ldots{} +
(3-5.4)² = 56

\textbf{Step 4: Calculate SSW} SSW = SST - SSB = 56 - 40 = 16

\textbf{Step 5: Calculate Mean Squares} MSB = SSB/(k-1) = 40/(3-1) = 20
MSW = SSW/(N-k) = 16/(15-3) = 1.33

\textbf{Step 6: Calculate F-statistic} F = MSB/MSW = 20/1.33 = 15.0

\textbf{Step 7: Compare to critical value} F₀.₀₅,₂,₁₂ = 3.89. Since 15.0
\textgreater{} 3.89, p \textless{} 0.05 \textbf{Conclusion:} Reject H₀.
At least one medication differs significantly.

\end{tcolorbox}

\section{Post-Hoc Tests and Multiple
Comparisons}\label{post-hoc-tests-and-multiple-comparisons}

When ANOVA shows significant differences, \textbf{post-hoc tests}
determine which specific groups differ from each other.

\begin{tcolorbox}[enhanced jigsaw, breakable, toprule=.15mm, left=2mm, coltitle=black, colbacktitle=quarto-callout-note-color!10!white, opacityback=0, opacitybacktitle=0.6, arc=.35mm, titlerule=0mm, toptitle=1mm, colframe=quarto-callout-note-color-frame, colback=white, title=\textcolor{quarto-callout-note-color}{\faInfo}\hspace{0.5em}{Why Post-Hoc Tests?}, bottomtitle=1mm, leftrule=.75mm, rightrule=.15mm, bottomrule=.15mm]

ANOVA only tells us that ``at least one group differs'' but not:

\begin{itemize}
\tightlist
\item
  Which specific groups are different?
\item
  How many groups are different?
\item
  The magnitude of differences?
\end{itemize}

Post-hoc tests provide pairwise comparisons while controlling
family-wise error rate.

\end{tcolorbox}

\subsection{Common Post-Hoc Tests}\label{common-post-hoc-tests}

\begin{tcolorbox}[enhanced jigsaw, breakable, toprule=.15mm, left=2mm, coltitle=black, colbacktitle=quarto-callout-tip-color!10!white, opacityback=0, opacitybacktitle=0.6, arc=.35mm, titlerule=0mm, toptitle=1mm, colframe=quarto-callout-tip-color-frame, colback=white, title=\textcolor{quarto-callout-tip-color}{\faLightbulb}\hspace{0.5em}{Tukey's HSD (Honestly Significant Difference)}, bottomtitle=1mm, leftrule=.75mm, rightrule=.15mm, bottomrule=.15mm]

\textbf{When to use:} Most common, good balance of power and control
\textbf{Formula:} HSD = q × √(MSW/n) \textbf{Advantage:} Controls
family-wise error rate at α \textbf{Disadvantage:} Requires equal sample
sizes (or harmonic mean)

\textbf{Interpretation:} If \textbar x̄ᵢ - x̄ⱼ\textbar{} \textgreater{}
HSD, groups i and j differ significantly

\end{tcolorbox}

\begin{tcolorbox}[enhanced jigsaw, breakable, toprule=.15mm, left=2mm, coltitle=black, colbacktitle=quarto-callout-tip-color!10!white, opacityback=0, opacitybacktitle=0.6, arc=.35mm, titlerule=0mm, toptitle=1mm, colframe=quarto-callout-tip-color-frame, colback=white, title=\textcolor{quarto-callout-tip-color}{\faLightbulb}\hspace{0.5em}{Bonferroni Correction}, bottomtitle=1mm, leftrule=.75mm, rightrule=.15mm, bottomrule=.15mm]

\textbf{When to use:} Conservative approach, few planned comparisons
\textbf{Method:} Use α/c for each comparison (c = number of comparisons)
\textbf{Example:} With 3 groups, 3 comparisons, use α = 0.05/3 = 0.017
\textbf{Advantage:} Simple, very conservative \textbf{Disadvantage:} Can
be overly conservative, reduced power

\end{tcolorbox}

\begin{tcolorbox}[enhanced jigsaw, breakable, toprule=.15mm, left=2mm, coltitle=black, colbacktitle=quarto-callout-tip-color!10!white, opacityback=0, opacitybacktitle=0.6, arc=.35mm, titlerule=0mm, toptitle=1mm, colframe=quarto-callout-tip-color-frame, colback=white, title=\textcolor{quarto-callout-tip-color}{\faLightbulb}\hspace{0.5em}{Scheffé's Test}, bottomtitle=1mm, leftrule=.75mm, rightrule=.15mm, bottomrule=.15mm]

\textbf{When to use:} Any possible comparison, not just pairwise
\textbf{Advantage:} Most flexible, allows complex contrasts
\textbf{Disadvantage:} Most conservative for simple pairwise comparisons

\end{tcolorbox}

\subsection{Continuing Our Pain Medication
Example}\label{continuing-our-pain-medication-example}

\begin{tcolorbox}[enhanced jigsaw, breakable, toprule=.15mm, left=2mm, coltitle=black, colbacktitle=quarto-callout-tip-color!10!white, opacityback=0, opacitybacktitle=0.6, arc=.35mm, titlerule=0mm, toptitle=1mm, colframe=quarto-callout-tip-color-frame, colback=white, title=\textcolor{quarto-callout-tip-color}{\faLightbulb}\hspace{0.5em}{Post-Hoc Analysis}, bottomtitle=1mm, leftrule=.75mm, rightrule=.15mm, bottomrule=.15mm]

We found F = 15.0, p \textless{} 0.05. Now let's determine which
medications differ:

\textbf{Tukey's HSD Calculation:} HSD = q₀.₀₅,₃,₁₂ × √(MSW/n) = 3.77 ×
√(1.33/5) = 1.94

\textbf{Pairwise Comparisons:} - \textbar Med A - Med B\textbar{} =
\textbar7.4 - 5.4\textbar{} = 2.0 \textgreater{} 1.94 ✓
\textbf{Significant} - \textbar Med A - Placebo\textbar{} = \textbar7.4
- 3.4\textbar{} = 4.0 \textgreater{} 1.94 ✓ \textbf{Significant} -
\textbar Med B - Placebo\textbar{} = \textbar5.4 - 3.4\textbar{} = 2.0
\textgreater{} 1.94 ✓ \textbf{Significant}

\textbf{Conclusion:} All three treatments differ significantly from each
other. \textbf{Clinical Interpretation:} Med A \textgreater{} Med B
\textgreater{} Placebo for pain reduction.

\end{tcolorbox}

\begin{tcolorbox}[enhanced jigsaw, breakable, toprule=.15mm, left=2mm, coltitle=black, colbacktitle=quarto-callout-warning-color!10!white, opacityback=0, opacitybacktitle=0.6, arc=.35mm, titlerule=0mm, toptitle=1mm, colframe=quarto-callout-warning-color-frame, colback=white, title=\textcolor{quarto-callout-warning-color}{\faExclamationTriangle}\hspace{0.5em}{Common Mistakes in Post-Hoc Testing}, bottomtitle=1mm, leftrule=.75mm, rightrule=.15mm, bottomrule=.15mm]

\begin{itemize}
\tightlist
\item
  \textbf{❌ Running post-hocs when ANOVA is not significant} → Fishing
  for significance
\item
  \textbf{❌ Using multiple different post-hoc tests} → Inflated error
  rates
\item
  \textbf{❌ Ignoring practical significance} → Statistical ≠ clinical
  significance
\item
  \textbf{✅ Choose post-hoc test before analysis} → Avoid bias
\item
  \textbf{✅ Report effect sizes} → Practical importance
\end{itemize}

\end{tcolorbox}

\section{Two-Way ANOVA}\label{two-way-anova}

Two-way ANOVA examines the effects of two categorical variables
(factors) simultaneously, including their potential interaction.

\begin{tcolorbox}[enhanced jigsaw, breakable, toprule=.15mm, left=2mm, coltitle=black, colbacktitle=quarto-callout-note-color!10!white, opacityback=0, opacitybacktitle=0.6, arc=.35mm, titlerule=0mm, toptitle=1mm, colframe=quarto-callout-note-color-frame, colback=white, title=\textcolor{quarto-callout-note-color}{\faInfo}\hspace{0.5em}{Advantages of Two-Way ANOVA}, bottomtitle=1mm, leftrule=.75mm, rightrule=.15mm, bottomrule=.15mm]

\begin{itemize}
\tightlist
\item
  \textbf{Efficiency:} Tests multiple factors in one analysis
\item
  \textbf{Interaction Detection:} Identifies when factors work together
\item
  \textbf{Control:} Accounts for additional sources of variation
\item
  \textbf{Power:} Often more powerful than separate one-way ANOVAs
\end{itemize}

\end{tcolorbox}

\subsection{Three Research Questions}\label{three-research-questions}

Two-way ANOVA addresses three hypotheses simultaneously:

\textbf{Main Effect of Factor A:} H₀: μ₁• = μ₂• = \ldots{} = μₐ•
\textbf{Main Effect of Factor B:} H₀: μ•₁ = μ•₂ = \ldots{} = μ•ᵦ
\textbf{Interaction Effect:} H₀: No A×B interaction

\begin{tcolorbox}[enhanced jigsaw, breakable, toprule=.15mm, left=2mm, coltitle=black, colbacktitle=quarto-callout-tip-color!10!white, opacityback=0, opacitybacktitle=0.6, arc=.35mm, titlerule=0mm, toptitle=1mm, colframe=quarto-callout-tip-color-frame, colback=white, title=\textcolor{quarto-callout-tip-color}{\faLightbulb}\hspace{0.5em}{Clinical Trial Example: Drug and Exercise}, bottomtitle=1mm, leftrule=.75mm, rightrule=.15mm, bottomrule=.15mm]

\textbf{Research Question:} Do cholesterol-lowering drugs and exercise
programs interact?

\textbf{Factor A (Drug):} New drug vs.~Placebo \textbf{Factor B
(Exercise):} High intensity vs.~Low intensity \textbf{Outcome:}
Cholesterol reduction (mg/dL)

\textbf{Design:} 2×2 factorial design - Group 1: New drug + High
exercise (n=20) - Group 2: New drug + Low exercise (n=20) - Group 3:
Placebo + High exercise (n=20) - Group 4: Placebo + Low exercise (n=20)

\end{tcolorbox}

\subsection{Understanding
Interactions}\label{understanding-interactions}

\begin{tcolorbox}[enhanced jigsaw, breakable, toprule=.15mm, left=2mm, coltitle=black, colbacktitle=quarto-callout-note-color!10!white, opacityback=0, opacitybacktitle=0.6, arc=.35mm, titlerule=0mm, toptitle=1mm, colframe=quarto-callout-note-color-frame, colback=white, title=\textcolor{quarto-callout-note-color}{\faInfo}\hspace{0.5em}{Types of Interaction Effects}, bottomtitle=1mm, leftrule=.75mm, rightrule=.15mm, bottomrule=.15mm]

\textbf{No Interaction:} Effect of Factor A is the same at all levels of
Factor B \textbf{Ordinal Interaction:} Effect of Factor A varies in
magnitude but not direction \textbf{Disordinal Interaction:} Effect of
Factor A changes direction at different levels of Factor B

\end{tcolorbox}

\begin{tcolorbox}[enhanced jigsaw, breakable, toprule=.15mm, left=2mm, coltitle=black, colbacktitle=quarto-callout-tip-color!10!white, opacityback=0, opacitybacktitle=0.6, arc=.35mm, titlerule=0mm, toptitle=1mm, colframe=quarto-callout-tip-color-frame, colback=white, title=\textcolor{quarto-callout-tip-color}{\faLightbulb}\hspace{0.5em}{Interpreting Interaction: Hypothetical Results}, bottomtitle=1mm, leftrule=.75mm, rightrule=.15mm, bottomrule=.15mm]

\textbf{Scenario 1: No Interaction}

\begin{longtable}[]{@{}llll@{}}
\toprule\noalign{}
& High Exercise & Low Exercise & Difference \\
\midrule\noalign{}
\endhead
\bottomrule\noalign{}
\endlastfoot
New Drug & 40 mg/dL & 30 mg/dL & 10 mg/dL \\
Placebo & 20 mg/dL & 10 mg/dL & 10 mg/dL \\
\end{longtable}

\textbf{Interpretation:} Exercise always improves cholesterol by 10
mg/dL, regardless of drug

\textbf{Scenario 2: Significant Interaction}

\begin{longtable}[]{@{}llll@{}}
\toprule\noalign{}
& High Exercise & Low Exercise & Difference \\
\midrule\noalign{}
\endhead
\bottomrule\noalign{}
\endlastfoot
New Drug & 50 mg/dL & 25 mg/dL & 25 mg/dL \\
Placebo & 15 mg/dL & 10 mg/dL & 5 mg/dL \\
\end{longtable}

\textbf{Interpretation:} The drug works much better with high exercise
(synergistic effect)

\end{tcolorbox}

\begin{tcolorbox}[enhanced jigsaw, breakable, toprule=.15mm, left=2mm, coltitle=black, colbacktitle=quarto-callout-warning-color!10!white, opacityback=0, opacitybacktitle=0.6, arc=.35mm, titlerule=0mm, toptitle=1mm, colframe=quarto-callout-warning-color-frame, colback=white, title=\textcolor{quarto-callout-warning-color}{\faExclamationTriangle}\hspace{0.5em}{Interpreting Two-Way ANOVA Results}, bottomtitle=1mm, leftrule=.75mm, rightrule=.15mm, bottomrule=.15mm]

\textbf{If interaction is significant:} - Focus primarily on interaction
interpretation - Main effects may be misleading - Use simple effects
analysis or interaction contrasts

\textbf{If interaction is not significant:} - Interpret main effects
independently - Each factor's effect is consistent across levels of the
other

\end{tcolorbox}

\section{ANOVA in Practice: Assumptions and
Alternatives}\label{anova-in-practice-assumptions-and-alternatives}

\subsection{When ANOVA Assumptions
Fail}\label{when-anova-assumptions-fail}

\begin{tcolorbox}[enhanced jigsaw, breakable, toprule=.15mm, left=2mm, coltitle=black, colbacktitle=quarto-callout-warning-color!10!white, opacityback=0, opacitybacktitle=0.6, arc=.35mm, titlerule=0mm, toptitle=1mm, colframe=quarto-callout-warning-color-frame, colback=white, title=\textcolor{quarto-callout-warning-color}{\faExclamationTriangle}\hspace{0.5em}{Assumption Violations and Solutions}, bottomtitle=1mm, leftrule=.75mm, rightrule=.15mm, bottomrule=.15mm]

\textbf{Non-normality:} - \textbf{Mild violation:} ANOVA is robust
(central limit theorem) - \textbf{Severe violation:} Use Kruskal-Wallis
test (non-parametric) - \textbf{Transformation:} Log, square root, or
Box-Cox transformations

\textbf{Unequal variances:} - \textbf{Welch's ANOVA:} Doesn't assume
equal variances - \textbf{Brown-Forsythe test:} Robust to variance
differences - \textbf{Transformation:} Often stabilizes variances

\textbf{Non-independence:} - \textbf{Most serious violation} → Inflated
Type I error - \textbf{Solutions:} Mixed-effects models, cluster-robust
standard errors - \textbf{Design fix:} Proper randomization, account for
clustering

\end{tcolorbox}

\subsection{Effect Size and Practical
Significance}\label{effect-size-and-practical-significance}

\begin{tcolorbox}[enhanced jigsaw, breakable, toprule=.15mm, left=2mm, coltitle=black, colbacktitle=quarto-callout-note-color!10!white, opacityback=0, opacitybacktitle=0.6, arc=.35mm, titlerule=0mm, toptitle=1mm, colframe=quarto-callout-note-color-frame, colback=white, title=\textcolor{quarto-callout-note-color}{\faInfo}\hspace{0.5em}{Effect Size Measures for ANOVA}, bottomtitle=1mm, leftrule=.75mm, rightrule=.15mm, bottomrule=.15mm]

\textbf{Eta-squared (η²):} Proportion of total variance explained η² =
SSB / SST

\textbf{Partial Eta-squared:} More common in complex designs ηₚ² = SSB /
(SSB + SSW)

\textbf{Cohen's Guidelines:} - Small effect: η² = 0.01 - Medium effect:
η² = 0.06 - Large effect: η² = 0.14

\end{tcolorbox}

\begin{tcolorbox}[enhanced jigsaw, breakable, toprule=.15mm, left=2mm, coltitle=black, colbacktitle=quarto-callout-tip-color!10!white, opacityback=0, opacitybacktitle=0.6, arc=.35mm, titlerule=0mm, toptitle=1mm, colframe=quarto-callout-tip-color-frame, colback=white, title=\textcolor{quarto-callout-tip-color}{\faLightbulb}\hspace{0.5em}{Reporting ANOVA Results}, bottomtitle=1mm, leftrule=.75mm, rightrule=.15mm, bottomrule=.15mm]

\textbf{Complete ANOVA Report:}

``A one-way ANOVA was conducted to compare the effectiveness of three
pain medications. The analysis revealed a statistically significant
difference between groups, F(2, 12) = 15.0, p \textless{} 0.001, η² =
0.71, indicating a large effect size.

Post-hoc comparisons using Tukey's HSD test indicated that all pairwise
comparisons were statistically significant (p \textless{} 0.05).
Medication A (M = 7.4, SD = 1.1) was more effective than Medication B (M
= 5.4, SD = 1.1), which was more effective than Placebo (M = 3.4, SD =
1.1).''

\end{tcolorbox}

\subsection{Power Analysis for ANOVA}\label{power-analysis-for-anova}

\begin{tcolorbox}[enhanced jigsaw, breakable, toprule=.15mm, left=2mm, coltitle=black, colbacktitle=quarto-callout-note-color!10!white, opacityback=0, opacitybacktitle=0.6, arc=.35mm, titlerule=0mm, toptitle=1mm, colframe=quarto-callout-note-color-frame, colback=white, title=\textcolor{quarto-callout-note-color}{\faInfo}\hspace{0.5em}{Sample Size Planning}, bottomtitle=1mm, leftrule=.75mm, rightrule=.15mm, bottomrule=.15mm]

Power analysis helps determine appropriate sample sizes before
conducting research:

\begin{itemize}
\tightlist
\item
  \textbf{Specify:} Effect size, significance level (α), desired power
\item
  \textbf{Effect size:} Based on pilot data or literature
\item
  \textbf{Common power:} 0.80 (80\% chance of detecting true effect)
\item
  \textbf{Balance:} Equal sample sizes maximize power
\end{itemize}

\end{tcolorbox}

\begin{tcolorbox}[enhanced jigsaw, breakable, toprule=.15mm, left=2mm, coltitle=black, colbacktitle=quarto-callout-tip-color!10!white, opacityback=0, opacitybacktitle=0.6, arc=.35mm, titlerule=0mm, toptitle=1mm, colframe=quarto-callout-tip-color-frame, colback=white, title=\textcolor{quarto-callout-tip-color}{\faLightbulb}\hspace{0.5em}{Power Analysis Example}, bottomtitle=1mm, leftrule=.75mm, rightrule=.15mm, bottomrule=.15mm]

\textbf{Research Planning:} Comparing 4 treatments for depression

\begin{itemize}
\tightlist
\item
  \textbf{Expected effect size:} f = 0.25 (medium effect)
\item
  \textbf{Significance level:} α = 0.05
\item
  \textbf{Desired power:} 0.80
\item
  \textbf{Result:} Need n = 45 per group (180 total)
\end{itemize}

\textbf{Interpretation:} With 45 participants per group, we have an 80\%
chance of detecting a medium-sized difference if it truly exists.

\end{tcolorbox}

\bookmarksetup{startatroot}

\chapter{Interrelationships \&
Modeling}\label{interrelationships-modeling}

Move beyond simple group comparisons to understand relationships between
variables and build predictive models.

\emph{Note: This chapter appears to be incomplete in the source HTML
file. Only the title and introduction were available. Additional content
for interrelationships and modeling methods would typically include:}

\begin{itemize}
\tightlist
\item
  Correlation analysis
\item
  Simple and multiple linear regression
\item
  Logistic regression
\item
  Model diagnostics and assumptions
\item
  Variable selection techniques
\item
  Model interpretation and inference
\end{itemize}

\emph{These topics would be covered in a complete version of this
chapter.}

\bookmarksetup{startatroot}

\chapter{Design \& Power
Considerations}\label{design-power-considerations}

Plan studies effectively by determining appropriate sample sizes and
control error rates when testing multiple hypotheses.

\emph{Note: This chapter appears to be incomplete in the source HTML
file. Only the title and introduction were available. Additional content
for design and power considerations would typically include:}

\begin{itemize}
\tightlist
\item
  Study design principles
\item
  Sample size calculation methods
\item
  Power analysis techniques
\item
  Multiple testing corrections (Bonferroni, FDR, etc.)
\item
  Type I and Type II error control
\item
  Interim analysis and stopping rules
\item
  Adaptive trial designs
\end{itemize}

\emph{These topics would be covered in a complete version of this
chapter.}

\bookmarksetup{startatroot}

\chapter{Specialized \& Modern
Methods}\label{specialized-modern-methods}

Advanced techniques for complex data scenarios, from time-to-event
analysis to modern computational approaches in the era of big data.

\emph{Note: This chapter appears to be incomplete in the source HTML
file. Only the title and introduction were available. Additional content
for specialized and modern methods would typically include:}

\begin{itemize}
\tightlist
\item
  Survival analysis (Kaplan-Meier, Cox regression)
\item
  Non-parametric methods (Mann-Whitney, Kruskal-Wallis, etc.)
\item
  Bootstrap and resampling methods
\item
  Mixed-effects models for repeated measures
\item
  Time series analysis
\item
  Machine learning approaches in biostatistics
\item
  Bayesian methods
\item
  Meta-analysis techniques
\item
  Handling missing data (imputation methods)
\item
  High-dimensional data analysis
\end{itemize}

\emph{These topics would be covered in a complete version of this
chapter.}

\bookmarksetup{startatroot}

\chapter*{References}\label{references}
\addcontentsline{toc}{chapter}{References}

\markboth{References}{References}

\phantomsection\label{refs}
\begin{CSLReferences}{0}{1}
\end{CSLReferences}

\section*{Recommended Reading}\label{recommended-reading}
\addcontentsline{toc}{section}{Recommended Reading}

\markright{Recommended Reading}

\subsection*{Foundational Texts}\label{foundational-texts}
\addcontentsline{toc}{subsection}{Foundational Texts}

\begin{itemize}
\tightlist
\item
  \textbf{Rosner, B.} (2015). \emph{Fundamentals of Biostatistics} (8th
  ed.). Cengage Learning.
\item
  \textbf{Pagano, M., \& Gauvreau, K.} (2000). \emph{Principles of
  Biostatistics} (2nd ed.). Duxbury Press.
\item
  \textbf{Jewell, N. P.} (2003). \emph{Statistics for Epidemiology}.
  Chapman and Hall/CRC.
\end{itemize}

\subsection*{Advanced Methods}\label{advanced-methods}
\addcontentsline{toc}{subsection}{Advanced Methods}

\begin{itemize}
\tightlist
\item
  \textbf{Kleinbaum, D. G., \& Klein, M.} (2012). \emph{Survival
  Analysis: A Self-Learning Text} (3rd ed.). Springer.
\item
  \textbf{Hosmer, D. W., Lemeshow, S., \& Sturdivant, R. X.} (2013).
  \emph{Applied Logistic Regression} (3rd ed.). Wiley.
\item
  \textbf{Diggle, P., Heagerty, P., Liang, K. Y., \& Zeger, S.} (2002).
  \emph{Analysis of Longitudinal Data} (2nd ed.). Oxford University
  Press.
\end{itemize}

\subsection*{Statistical Software}\label{statistical-software}
\addcontentsline{toc}{subsection}{Statistical Software}

\begin{itemize}
\tightlist
\item
  \textbf{R Core Team} (2023). \emph{R: A language and environment for
  statistical computing}. R Foundation for Statistical Computing.
  https://www.R-project.org/
\item
  \textbf{Wickham, H., \& Grolemund, G.} (2017). \emph{R for Data
  Science}. O'Reilly Media. https://r4ds.had.co.nz/
\item
  \textbf{VanderPlas, J.} (2016). \emph{Python Data Science Handbook}.
  O'Reilly Media.
\end{itemize}

\subsection*{Methodological Resources}\label{methodological-resources}
\addcontentsline{toc}{subsection}{Methodological Resources}

\begin{itemize}
\tightlist
\item
  \textbf{Altman, D. G.} (1991). \emph{Practical Statistics for Medical
  Research}. Chapman and Hall/CRC.
\item
  \textbf{Bland, M.} (2015). \emph{An Introduction to Medical
  Statistics} (4th ed.). Oxford University Press.
\item
  \textbf{Zar, J. H.} (2013). \emph{Biostatistical Analysis} (5th ed.).
  Pearson.
\end{itemize}

\subsection*{Online Resources}\label{online-resources}
\addcontentsline{toc}{subsection}{Online Resources}

\begin{itemize}
\tightlist
\item
  \textbf{Khan Academy Statistics and Probability}:
  https://www.khanacademy.org/math/statistics-probability
\item
  \textbf{Coursera Biostatistics Courses}: Various universities offer
  comprehensive biostatistics courses
\item
  \textbf{edX Statistics Courses}: Free courses from top universities
\item
  \textbf{R Documentation}: https://www.rdocumentation.org/
\item
  \textbf{Python Statistics Libraries}: SciPy, StatsModels, Scikit-learn
  documentation
\end{itemize}


\backmatter


\end{document}
